\chapter*{Remerciements}

Tout d'abord, je tiens à remercier mon directeur de thèse, Guy Gogniat, Professeur, ainsi que mon codirecteur de thèse, Vianney Lapôtre, Maitre de Conférences HDR, tous les deux à l'Université Bretagne Sud, à Lorient. Leur accompagnement, expertise et soutien ont été plus que précieux durant cette thèse.

Je remercie également Lejla Batina, Nele Mentens et Vincent Beroulle, respectivement Professeurs des Universités à Radboub (Pays-Bas), KU leuven (Belgique), et à Grenoble, pour avoir accepté de rapporter ma thèse. Leurs remarques ont été pertinentes et m'ont permis d'améliorer mon manuscrit.

Je souhaite également remercier Jean-Max Dutertre, Professeur à l'\'Ecole des Mines de Saint-\'Etienne, à Gardanne, qui a accepté de participer à mon Comité de Suivi de thèse (CSI) ainsi que d'avoir accepté de faire partie de mon jury de thèse. Je remercie également Karine Heydemann pour avoir fait partie de mon CSI.

Je remercie grandement Francesco Regazzoni, Professeur à l'Università della Svizzera Italiana, à Lugano (Suisse) et à l'Université d'Amsterdam (Pays-Bas) pour avoir accepté de faire partie de mon jury de thèse et pour m'avoir guidé durant ma mobilité. J'ai beaucoup apprécié les échanges que nous avons pu avoir. Cela a contribué positivement à mon travail, la preuve étant avec les contributions scientifiques que cela a amené. Cela m'a permis d'étendre mes connaissances en sécurité ainsi que d'améliorer mon anglais. J'ai pu rencontrer de nouveaux chercheurs, doctorants et postdoctorants au laboratoire et d'échanger avec eux, ce qui m'a enrichi personnellement et professionnellement. J'ai eu la chance de découvrir un endroit formidable durant ces cinq mois et d'en garder un souvenir incroyable. J'espère pouvoir y retourner bientôt.

De plus, je souhaite remercier mes collègues au Lab-STICC, Nicolas, Mohamed, Noura, Hongwei, Tianxu, Clément, et tous les autres, ainsi que les personnes que j'ai pu rencontrer durant ma thèse. Un grand merci à Tom et Chiara, rencontrés lors de mon séjour à Lugano. Ils m'ont tous deux aidé à m'intégrer dans cette nouvelle ville. Nous avons également eu des discussions très intéressantes. Je vous dis à bientôt, j'espère. Je souhaite enfin remercier mes enseignants de Licence et Master (particulièrement Catherine Dezan et David Espès) à l'Université de Bretagne Occidentale d'avoir cru en moi et m'avoir offert la possibilité de réaliser des stages en recherche.

Finalement, je conclurai en remerciant mes parents, mon frère, ma copine Hellen, ma famille, ainsi que tous mes amis pour leur support et leur accompagnement pendant toutes ces années. Merci pour vos remarques, vos conseils et votre écoute. Vous m'avez tous permis de mener à bien ce travail jusqu'au bout en me permettant de me sentir toujours bien.
Merci pour tout.

\chapter*{Acknowledgments}

Firstly, I would like to thank my thesis supervisor, Professor Guy Gogniat, and my co-supervisor, Vianney Lapôtre, Associate Professor, both at the Université Bretagne Sud in Lorient. Their guidance, expertise, and support have been invaluable during this thesis.

Secondly, I would like to thank Lejla Batina, Nele Mentens and Vincent Beroulle, respectively Full Professors at Radboub University (Netherlands), KU Leuven (Belgium) and Grenoble, for agreeing to review my PhD thesis. Their comments were pertinent and helped me to improve my manuscript.

I would also like to thank Jean-Max Dutertre, Professor at the Ecole des Mines de Saint-Étienne, Gardanne, who agreed to take part in my thesis monitoring committee (Comité de Suivi de thèse - CSI) and to sit on my thesis jury. I would also like to thank Karine Heydemann for being part of my CSI.

I would like to thank Francesco Regazzoni, Senior Researcher at the Università della Svizzera Italiana in Lugano (Switzerland) and at the University of Amsterdam (Netherlands), for agreeing to sit on my thesis jury and for guiding me during my mobility. I very much appreciated the exchanges we had. It made a positive contribution to my work, as evidenced by the scientific contributions it brought. It enabled me to broaden my knowledge of safety and improve my English. I was able to meet new researchers, PhD students and post-docs in the laboratory and exchange ideas with them, which enriched me personally and professionally. I've been lucky enough to discover a wonderful place during these five months and to have incredible memories of it. I hope to be able to return there very soon.

I would also like to thank my colleagues at the Lab-STICC, Nicolas, Mohamed, Noura, Hongwei, Tianxu, Clément, and all the others, as well as the people I met during my thesis. A big thanks to Tom and Chiara, whom I met during my stay in Lugano. They both helped me at the time to integrate myself in this new city. We also had some very interesting and rewarding discussions. I hope to see you soon. Finally, I'd like to thank my undergraduate and Master's teachers (especially Catherine Dezan and David Espès) at the Université de Bretagne Occidentale for believing in me and giving me the opportunity to do research internships.

Finally, I would like to conclude by thanking my parents, my brother, my girlfriend Hellen, my family and all my friends for their support and guidance over the years. Thank you for your comments, advice, and attentiveness. You have all enabled me to see this work through to the end, making me always feel good.

Thank you very much for everything.