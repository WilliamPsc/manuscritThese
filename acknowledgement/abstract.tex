\chapter*{Résumé}
\label{chapter:resume}
% \addcontentsline{toc}{chapter}{Résumé}

L'augmentation de l'IoT, dans des domaines tels que la santé ou l'industrie, favorise l'augmentation de la surface d'attaque, ce qui soulève d'importantes préoccupations en termes de sécurité. Ces systèmes, traitant des données sensibles, sont vulnérables aux attaques logicielles et physiques en raison de leur connectivité réseau et de leur proximité avec les attaquants.
Le suivi dynamique des flux d'informations (DIFT) détecte les attaques logicielles, comme les maliciels, en étiquetant et en suivant les données au moment de l'exécution. Les attaques par injection de fautes (FIA) induisent des erreurs (par exemple, via la tension ou des lasers) perturbant le comportement et contournant les mécanismes de sécurité. Les FIA sont critiques dans les systèmes embarqués et cryptographiques, où les vulnérabilités peuvent compromettre les données. Bien que de nombreuses études aient exploré les vulnérabilités des FIA, aucune n'a ciblé les mécanismes DIFT.
Nous travaillons sur le processeur D-RI5CY, implémentant un DIFT matériel in-core. Nous évaluons l'impact des FIA sur l'efficacité du DIFT. Grâce à des simulations d'injection de fautes, en utilisant FISSA, un outil conçu pour l'évaluation des fautes, nous identifions les registres vulnérables et implémentons trois protections : la parité simple pour la détection, le code de Hamming pour la correction d'erreurs sur un bit, et SECDED pour détecter les erreurs sur deux bits. Ces protections ont été optimisées en regroupant les registres afin de minimiser le coût. Nous avons ensuite évalué d'autres compositions de groupes pour améliorer la protection contre des modèles plus complexes, en développant quatre stratégies pour améliorer la détection et la correction des erreurs.

%%%%%%%%%%%%%%%%%%%%%%%%%%%%%%%%%%%%%%%%%%%%%%%%%%%%%%%%%%%%%%%%%%%%%%%%%%%%%%%%%%%%%%%%%%%%%%%
\chapter*{Abstract}
\label{chapter:abstract}
% \addcontentsline{toc}{chapter}{Abstract}

Embedded security is more and more crucial with the huge increase of IoT devices, enhancing efficiency and addressing challenges like industrial change and health. However, their widespread use also increases the attack surface, raising significant security concerns. These systems, handling sensitive data, are vulnerable to both software and physical attacks due to their network connectivity and proximity to attackers.
Dynamic Information Flow Tracking (DIFT) detects software attacks, such as buffer overflows, by tagging and tracking data at runtime. Fault Injection Attacks (FIA) deliberately introduce hardware errors to disrupt normal operation and bypass security mechanisms. These faults can be introduced physically (e.g., via voltage or lasers). FIAs are concerning in embedded and cryptographic systems, where low-level faults can compromise sensitive data. Although many studies have explored FIA vulnerabilities, none have targeted DIFT mechanisms.
Our research focuses on the D-RI5CY processor, which implements a hardware in-core DIFT. We assess the impact of FIAs on DIFT's effectiveness in this processor. Through fault injection simulations, using FISSA, a tool developed to facilitate fault evaluation, we identify vulnerable hardware registers and implement three countermeasures: simple parity for error detection, Hamming Code for single-bit error correction, and SECDED to detect double-bit errors. These were optimised by grouping registers to minimise redundancy overhead. We further evaluated multiple register group compositions to enhance countermeasures against complex fault models, developing four strategies to improve error detection and correction efficiency.


%%%%%%%%%%%%%%%%%%%%%%%%%%%%%%%%%%%%%%%%%%%%%%%%%%%%%%%%%%%%%%%%%%%%%%%%%%%%%%%%%%%%%%%%%%%%%%%
\chapter*{Résumé étendu}
\label{chapter:extended_resume}
\addcontentsline{toc}{chapter}{Résumé \'Etendu}

L'internet des objets (IoT) a révolutionné la façon dont nous interagissons avec la technologie, en permettant une connectivité et une communication entre une myriade d'appareils. Ces appareils font partie de notre vie quotidienne, de l'ampoule connectée à la voiture autonome. Ils collectent et partagent des données sur la manière dont ils sont utilisés et sur l'environnement dans lequel ils fonctionnent. Les systèmes embarqués, qui constituent l'épine dorsale des appareils IoT, sont de plus en plus vulnérables aux menaces logicielles et matérielles, ainsi qu'aux menaces basées sur le réseau, pouvant entraîner des fuites de données ou un accès non autorisé à des composants essentiels du système. Ces systèmes sont fréquemment déployés dans des environnements dans lesquels ils sont exposés à des attaquants potentiels, ce qui en fait des cibles attrayantes pour divers types d'attaques.

La sécurité des logiciels est un aspect essentiel du développement et du déploiement des systèmes logiciels. Elle englobe les mesures et les pratiques destinées à protéger les applications contre les attaques malveillantes, les vulnérabilités et les autres risques de sécurité.
En ce qui concerne le matériel, les attaques physiques font référence à différentes techniques et méthodes visant à compromettre la sécurité des systèmes embarqués. Ces attaques exploitent les vulnérabilités de la couche physique ou de la mise en œuvre du matériel de l'appareil pour supprimer, modifier, obtenir ou empêcher l'accès à des données confidentielles.
Les attaques physiques les plus courantes sont l'ingénierie inversée, les attaques par canaux auxiliaires et les attaques par injection de fautes (FIA).

Tout d'abord, nous décrivons le processeur sur lequel nous nous concentrons, avec sa mise en œuvre d'un DIFT matériel dans le cœur. Nous décrivons son fonctionnement et la manière d'utiliser le mécanisme DIFT avec la configuration par défaut. Ensuite, nous décrivons les différents cas d'utilisation avec lesquels nous avons choisi de travailler, afin d'analyser le comportement du DIFT et de l'évaluer contre les attaques par injection de fautes. Enfin, nous présentons l'évaluation de la vulnérabilité de ces cas d'utilisation à l'aide du mécanisme de sécurité D-RI5CY. Nous montrons que cette implémentation DIFT est vulnérable aux attaques par injection de fautes dans différents registres en fonction du modèle de fautes et en fonction de l'application, car différents chemins sont utilisés et donc différents registres seront critiques.