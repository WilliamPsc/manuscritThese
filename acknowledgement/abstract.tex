\chapter*{Résumé}
\label{chapter:resume}
% \addcontentsline{toc}{chapter}{Résumé}

L'augmentation de l'IoT, dans des domaines tels que la santé ou l'industrie, favorise la hausse de la surface d'attaque, ce qui soulève d'importantes préoccupations en termes de sécurité. Ces systèmes, traitant des données sensibles, sont vulnérables aux attaques logicielles et physiques en raison de leur connectivité réseau et de leur proximité avec les attaquants.
Le suivi dynamique des flux d'informations (DIFT) détecte les attaques logicielles, comme les maliciels, en étiquetant et en suivant les données au moment de l'exécution. Les attaques par injection de fautes (FIA) induisent des erreurs (par exemple, via la tension ou des lasers) perturbant le comportement et contournant les mécanismes de sécurité. Les FIA sont critiques dans les systèmes embarqués et cryptographiques, où les vulnérabilités peuvent compromettre les données. Bien que de nombreuses études aient exploré les vulnérabilités des FIA, aucune n'a ciblé les mécanismes DIFT.
Nous travaillons sur le processeur D-RI5CY, implémentant un DIFT matériel in-core. Nous évaluons l'impact des FIA sur l'efficacité du DIFT. Grâce à des simulations d'injection de fautes, en utilisant FISSA, un outil conçu pour l'évaluation des fautes, nous identifions les registres vulnérables et implémentons trois protections : la parité simple pour la détection, le code de Hamming pour la correction d'erreurs sur un bit, et SECDED pour détecter les erreurs sur deux bits. Ces protections ont été optimisées en regroupant les registres afin de minimiser le coût. Nous avons ensuite évalué d'autres compositions de groupes pour améliorer la protection contre des modèles plus complexes, en développant quatre stratégies pour améliorer la détection et la correction des erreurs.

%%%%%%%%%%%%%%%%%%%%%%%%%%%%%%%%%%%%%%%%%%%%%%%%%%%%%%%%%%%%%%%%%%%%%%%%%%%%%%%%%%%%%%%%%%%%%%%
\chapter*{Abstract}
\label{chapter:abstract}
% \addcontentsline{toc}{chapter}{Abstract}

Embedded security is more and more crucial with the huge increase of IoT devices, enhancing efficiency and addressing challenges like industrial change and health. However, their widespread use also increases the attack surface, raising significant security concerns. These systems, handling sensitive data, are vulnerable to both software and physical attacks due to their network connectivity and proximity to attackers.
Dynamic Information Flow Tracking (DIFT) detects software attacks, such as buffer overflows, by tagging and tracking data at runtime. Fault Injection Attacks (FIA) deliberately introduce hardware errors to disrupt normal operation and bypass security mechanisms. These faults can be introduced physically (e.g., via voltage or lasers). FIAs are concerning in embedded and cryptographic systems, where low-level faults can compromise sensitive data. Although many studies have explored FIA vulnerabilities, none have targeted DIFT mechanisms.
Our research focuses on the D-RI5CY processor, which implements a hardware in-core DIFT. We assess the impact of FIAs on DIFT's effectiveness in this processor. Through fault injection simulations, using FISSA, a tool developed to facilitate fault evaluation, we identify vulnerable hardware registers and implement three countermeasures: simple parity for error detection, Hamming Code for single-bit error correction, and SECDED to detect double-bit errors. These were optimised by grouping registers to minimise redundancy overhead. We further evaluated multiple register group compositions to enhance countermeasures against complex fault models, developing four strategies to improve error detection and correction efficiency.


%%%%%%%%%%%%%%%%%%%%%%%%%%%%%%%%%%%%%%%%%%%%%%%%%%%%%%%%%%%%%%%%%%%%%%%%%%%%%%%%%%%%%%%%%%%%%%%
\chapter*{Résumé étendu}
\label{chapter:extended_resume}
\addcontentsline{toc}{chapter}{Résumé \'Etendu}

L'Internet des objets (IoT) a transformé notre interaction avec la technologie en rendant possible la connectivité et la communication entre une multitude de dispositifs. Ces derniers, intégrés dans notre quotidien, allant des ampoules connectées aux véhicules autonomes, collectent et échangent des données relatives à leur utilisation et à leur environnement. Toutefois, les systèmes embarqués, qui constituent le cœur des dispositifs IoT, sont de plus en plus vulnérables aux attaques logicielles, matérielles et réseaux, pouvant entraîner des fuites de données ou l’accès non autorisé à des composants critiques. Ces systèmes sont souvent déployés dans des environnements dans lesquels ils sont exposés à des adversaires potentiels, les rendant ainsi des cibles privilégiées pour différentes formes d'attaques.

La sécurité des logiciels constitue un pilier fondamental dans le développement et le déploiement des systèmes embarqués, intégrant des pratiques et des mesures visant à protéger les applications contre les attaques malveillantes et autres vulnérabilités. Concernant la sécurité matérielle, les attaques physiques englobent une gamme de techniques visant à compromettre les systèmes embarqués en exploitant des failles dans la couche physique ou dans la mise en œuvre matérielle. Parmi les attaques les plus courantes, on retrouve l’ingénierie inverse, les attaques par canaux auxiliaires et les attaques par injection de fautes.

Dans cette thèse, nous proposons tout d’abord une revue de l’état de l’art couvrant les trois axes principaux de nos travaux. Nous introduisons dans un premier temps les mécanismes de suivi de flux d’informations, en présentant succinctement les solutions existantes et leur rôle dans la détection des attaques logicielles. Ensuite, nous abordons les attaques physiques, avec un accent particulier sur les attaques par injection de fautes. Enfin, nous concluons en examinant et en discutant les contre-mesures existantes face à ces attaques.

Deuxièmement, nous présentons le processeur étudié, intégrant un mécanisme de suivi de flux d’informations dynamiques matériel in-core, et décrivons son architecture et son fonctionnement en configuration par défaut. Nous détaillons ensuite les cas d'utilisation retenus pour évaluer le DIFT face aux attaques par injection de fautes. Enfin, nous réalisons une évaluation approfondie de la vulnérabilité de ces cas avec le mécanisme de sécurité D-RI5CY, montrant que l'implémentation DIFT est vulnérable à ces attaques, les registres critiques variant selon le modèle de fautes et l'application. En effet, l’utilisation de différents chemins d’exécution dans les applications entraîne une criticité variable des registres concernés.

Troisièmement, nous avons développé un outil open source, nommé FISSA, pour automatiser les campagnes d’injection de fautes en simulation à partir d'outils de développement HDL tels que Questasim. Cet outil génère des scripts TCL exécutables par un simulateur HDL et produit des logs permettant d’analyser la vulnérabilité d’un modèle face à un modèle de fautes spécifique. FISSA est utilisé tout au long de cette thèse pour l’évaluation de la sécurité, en s’inscrivant dans la démarche de \textit{Sécurité dès la Conception}.

Quatrièmement, nous avons démontré que le processeur D-RI5CY est vulnérable aux attaques par injection de fautes. En réponse, nous avons proposé trois contre-mesures basées sur des codes détecteurs et correcteurs d’erreurs : la simple parité pour la détection, le code de Hamming pour la détection et correction d’une faute, et SECDED (Single Error Correction, Double Error Detection), une extension du code de Hamming avec un bit supplémentaire pour détecter deux fautes et en corriger une. Ces contre-mesures offrent d'excellents résultats, avec une détection et correction de 100~\% des fautes injectées dans les modèles de fautes 1-bit et sur un modèle de fautes dans lequel on injecte deux fautes sur deux cycles différents.

Cinquièmement, nous avons exploré des modèles de fautes plus réalistes et démontrés que les contre-mesures initiales deviennent insuffisantes face aux attaques par injection multi-bits. Pour pallier ces faiblesses, nous avons proposé quatre stratégies d'implémentation visant à réduire le taux de succès des attaques, tout en maintenant un faible coût en termes de performances et de surface. Parmi ces stratégies, les quatrième et cinquième se sont révélées les plus efficaces, bien qu’elles engendrent les coûts les plus élevés en termes de ressources.

En conclusion de ce travail de recherche, nous avons évalué la robustesse du mécanisme DIFT contre les attaques par injection de fautes. Nous avons montré que ces mécanismes nécessitent des protections supplémentaires pour renforcer la sécurité des systèmes. Nous avons proposé trois contre-mesures et cinq stratégies d’implémentation adaptées à différents modèles de fautes complexes, par exemple des modèles de fautes multi-bits. Cependant, des vulnérabilités subsistent toujours face aux attaques les plus complexes. L’adoption de codes correcteurs d’erreurs plus puissants, tels que les codes LDPC ou BCH, bien qu’efficaces, impliquerait une augmentation considérable de la surface et une baisse des performances, rendant leur mise en œuvre coûteuse, voire impossible sur des systèmes à très hautes contraintes.