% La page de garde est en français
% The front cover is in French
\selectlanguage{french}

% Inclure les infos de chaque établissement
% Include each institution data

%%% Switch case in latex
%%% https://tex.stackexchange.com/a/343306
\makeatletter
\newcommand\addcase[3]{\expandafter\def\csname\string#1@case@#2\endcsname{#3}}
\newcommand\makeswitch[2][]{%
  \newcommand#2[1]{%
    \ifcsname\string#2@case@##1\endcsname\csname\string#2@case@##1\endcsname\else#1\fi%
  }%
}
\makeatother

%%%% Il faut adapter la taille des logos dans certains cas (e.g., EGAAL, 2 etablissements)
\newcommand\hauteurlogos[3]{
    \hauteurlogoecole{#1}
    \hauteurlogoetablissementA{#2}
    \hauteurlogoetablissementB{#3}
}

%%%%%%%%%%%%%%%%%%%%%%%%%%%%%%%%%%%%%%%%%%%%%%%%%%%
%%%%%%%%%%%%%%%% ECOLES DOCTORALES %%%%%%%%%%%%%%%%

%%%% #1: dossier des images, #2: numero ED, #3: couleur ED, #4-#5: nom complet sur plusieurs lignes
\newcommand\addecoledoctorale[5]{\direcole{#1}\numeroecole{#2}\definecolor{color-ecole}{RGB}{#3}\nomecoleA{#4}\nomecoleB{#5}}

\makeswitch[default]\ecoledoctorale{}

\addcase\ecoledoctorale{MathSTIC}{\addecoledoctorale
    {MathSTIC}
    {644}
    {159,182,217} %{236,115,127}
    {Math\'{e}matiques et Sciences et Technologies}
    {de l'Information et de la Communication en Bretagne Océane}
}

%%%%%%%%%%%%%%%%%%%%%%%%%%%%%%%%%%%%%%%%%%%%%%%%
%%%%%%%%%%%%%%%% ETABLISSEMENTS %%%%%%%%%%%%%%%%

%%%% #1 nom du logo, #2-#4: nom complet sur plusieurs lignes
\newcommand\addetablissement[4]{\logoetablissementB{#1}\nometablissementC{#2}\nometablissementD{#3}\nometablissementE{#4}}

\makeswitch[default]\etablissement{}

\addcase\etablissement{ENIB}{\addetablissement
    {ENIB}
    {}
    {L'\'{E}COLE NATIONALE}
    {D'ING\'{E}NIEURS DE BREST}
}

\addcase\etablissement{UBO}{\addetablissement
    {UBO}
    {}
    {}
    {L'UNIVERSIT\'{E} DE BRETAGNE OCCIDENTALE}
}
\addcase\etablissement{UBS}{\addetablissement
    {UBS}
    {}
    {}
    {L'UNIVERSIT\'{E} DE BRETAGNE SUD}
}



% Inclure infos de l'école doctorale
% Include doctoral school data
\ecoledoctorale{MathSTIC}

% Inclure infos de l'établissement
% Include institution data
\etablissement{UBS}

%Inscrivez ici votre sp\'{e}cialit\'{e} (voir liste des sp\'{e}cialit\'{e}s sur le site de votre \'{e}cole doctorale)
%Indicate the domain (see list of domains in your ecole doctorale)
\spec{Informatique et Architectures Numériques}

%Attention : le pr\'{e}nom doit être en minuscules (Jean) et le NOM en majuscules (BRITTEF) 
%Attention : the first name in small letters and the name in Capital letters 
\author{William PENSEC}

% Donner le titre complet de la th\`{e}se, \'{e}ventuellement le sous titre, si n\'{e}cessaire sur plusieurs lignes 
%Give the complete title of the thesis, if necessary on several lines
\title{Extension de la Protection des Processeurs Contre les Menaces Physiques et Logicielles par la Sécurisation du Mécanisme DIFT Contre les Attaques par Injections de Fautes}
\lesoustitre{Enhanced Processor Defence Against Physical and Software Threats by Securing DIFT Against Fault Injection Attacks}

%Indiquer la date et le lieu de soutenance de la th\`{e}se 
%indicates the date and the place of the defense 
\date{19/12/2024}
\lieu{Lorient}

%Indiquer le nom du (ou des) laboratoire (s) dans le(s)quel(s) le travail de th\`{e}se a \'{e}t\'{e} effectu\'{e}, indiquer aussi si souhait\'{e} le nom de la (les) facult\'{e}(s) (UFR, \'{e}cole(s), Institut(s), Centre(s)...), son (leurs) adresse(s)... 
%Indicates the name (or names) of research laboratories where the work has been done as well as (if desired) the names of faculties (UFR, Schools, institution...
\uniterecherche{Université Bretagne Sud, UMR CNRS 6285, Lab-STICC}

%Indiquer le Numero de th\`{e}se, si cela est opportun, ou laisser vide pour faire disparaitre cet ligne de la couverture
%Indicate the number of the thesis if there is one. otherwise leave empty so the line disappeurs on the cover
\numthese{« si pertinent »} % \numthese{}

%Indiquer le Pr\'{e}nom en minuscules et le Nom en majuscules, le titre de la personne et l’\'{e}tablissement dans lequel il effectue sa recherche  
%Indicates the first name on small letters and the Names on capital letters, the person's title and the institution where he/she belongs to.
%Exemples :  Examples :
%%%- Professeur, Universit\'{e} d’Angers 
%%%- Chercheur, CNRS, \'{e}cole Centrale de Nantes 
%%%-  Professeur d’universit\'{e} – Praticien Hospitalier, Universit\'{e} Paris V  
%%%-  Maitre de conf\'{e}rences, Oniris 
%%%- Charg\'{e} de recherche, INSERM, HDR, Universit\'{e} de Tours  
 %S’il n’y a pas de co-direction, faire disparaitre cet item de la couverture  
 %In there is no co-director, remove the item from the cover
\jury{
{\normalTwelve \textbf{Rapporteurs avant soutenance :}}\\ \newline
\footnotesizeTwelve
\begin{tabular}{@{}ll}
Pr\'{e}nom NOM & Fonction et \'{e}tablissement d'exercice \\
Pr\'{e}nom NOM & Fonction et \'{e}tablissement d'exercice \\
Pr\'{e}nom NOM & Fonction et \'{e}tablissement d'exercice \\
\end{tabular}

\vspace{\baselineskip}
{\normalTwelve \textbf{Composition du Jury :}}\\
{\fontsize{9.5}{11}\selectfont {\textcolor{red}{\textit{Attention, en cas d’absence d’un des membres du Jury le jour de la soutenance, la composition du jury doit être revue pour s’assurer qu’elle est conforme et devra être répercutée sur la couverture de thèse}}}}\\ \newline
\footnotesizeTwelve
\begin{tabular}{@{}lll}

Pr\'{e}sident :        & Pr\'{e}nom NOM & Fonction et \'{e}tablissement d'exercice \textit{(à préciser après la soutenance)} \\
Examinateurs :         & Pr\'{e}nom NOM & Fonction et \'{e}tablissement d'exercice \\
Dir. de th\`{e}se :    & Guy GOGNIAT & Professeur des Universités (Lab-STICC, Université Bretagne Sud) \\
Co-dir. de th\`{e}se : & Vianney LAP\^OTRE & Maitre de Conférence HDR (Lab-STICC, Université Bretagne Sud) \\
\end{tabular}


% {\normalTwelve \textbf{Rapporteurs avant soutenance :}}\\ \newline
% \footnotesizeTwelve
% \begin{tabular}{@{}ll}
% Lejla BATINA & Professeur des Universités (Radboud University) \\
% Nele MENTENS & Professeur des Universités (Leiden University et KU Leuven) \\
% Vincent BEROULLE & Professeur des Universités (LCIS, Université Grenoble Alpes) \\
% \end{tabular}

% \vspace{\baselineskip}
% {\normalTwelve \textbf{Composition du Jury :}}\\
% {\fontsize{9.5}{11}\selectfont {\textcolor{red}{\textit{Attention, en cas d’absence d’un des membres du Jury le jour de la soutenance, la composition du jury doit être revue pour s’assurer qu’elle est conforme et devra être répercutée sur la couverture de thèse}}}}\\ \newline
% \footnotesizeTwelve
% \begin{tabular}{@{}lll}

% Pr\'{e}sident :        & Pr\'{e}nom NOM & Fonction et \'{e}tablissement d'exercice \textit{(à préciser après la soutenance)} \\
% Examinateurs :         & Jean-Max DUTERTRE & Professeur des Universités (Ecole des Mines de Saint-Etienne) \\
%                        & Francesco REGAZZONI & Professeur des Universités (University of Amsterdam et\\ && Università della Svizzera italiana) \\
% Dir. de th\`{e}se :    & Guy GOGNIAT & Professeur des Universités (Lab-STICC, Université Bretagne Sud) \\
% Co-dir. de th\`{e}se : & Vianney LAP\^OTRE & Maitre de Conférence HDR (Lab-STICC, Université Bretagne Sud) \\
% \end{tabular}

\vspace{\baselineskip}
{\normalTwelve \textbf{Invit\'{e}(s) :}}\\ \newline
\footnotesizeTwelve
\begin{tabular}{@{}ll}
Pr\'{e}nom NOM & Fonction et \'{e}tablissement d'exercice \\
\end{tabular}
}

\maketitle
