% Ce fichier main.tex est le fichier principal \`{a} partir duquel tout est g\'{e}n\'{e}r\'{e}
% This file is the main file where the final document is generated
\documentclass{these-dbl}

% Remplir les metadonnees du pdf
% Fill the pdf metadata
\hypersetup{
   pdfauthor   = {William PENSEC},
   pdftitle    = {Protection d'un processeur avec DIFT contre des attaques physiques},
   pdfsubject  = {Th\`{e}se de doctorat de William PENSEC},
   pdfkeywords = {},
   pdfstartview= {FitV} % ajuste la page à la largueur de l'écran
}

\geometry{vmargin=4.0cm}
\input{listings}
\newboolean{showcomments}
% \setboolean{showcomments}{false}
\setboolean{showcomments}{true}
\ifthenelse{\boolean{showcomments}}
{ \newcommand{\mynote}[2]{
    \fbox{\bfseries\sffamily\normalsize#1}
    {\normalsize$\blacktriangleright$\textsf{\emph{#2}}$\blacktriangleleft$}}}
{ \newcommand{\mynote}[2]{}}

\newcommand{\wip}[1]{\mynote{William}{\textcolor{red}{#1}}}
\newcommand{\vl}[1]{\mynote{Vianney}{\textcolor{blue}{#1}}}
\newcommand{\gug}[1]{\mynote{Guy}{\textcolor{ForestGreen}{#1}}}

%%%%%%%%%%%%%%%%%%%%%%%%%%%%%%%%%%%%%%%%%%%%%%%%%%%%%%%%%%%%%%%%%%%%%%%%%%%%%%%%%%%%%%%%%%%%%%%%%%%%%%%%%%%%%%%%%%%%%%%%%%%%%%%
\makenomenclature
\nomenclature{FIA}{Fault Injection Attack}
\nomenclature{TPR}{Tag Propagation Register}
\nomenclature{TCR}{Tag Check Register}
\nomenclature{CSR}{Control and Status Registers}
\nomenclature{DIFT}{Dynamic Information Flow Tracking}
\nomenclature{RA}{Return Address}
\nomenclature{PC}{Program Counter}
\nomenclature{ISA}{Instruction Set Architecture}
\renewcommand\_{\textunderscore\allowbreak}

% Spécifier vos fichiers de bibliographie
% Specify you bibliography files here
\addbibresource{./biblio/biblio.bib}

\setlength{\headheight}{14pt}

%%%%%%%%%%%%%%%%%%%%% Minitoc configuration %%%%%%%%%%%%%%%%%%%%%
\setcounter{minitocdepth}{2}
\usepackage{lipsum}

%%%%%%%%%%%%%%%%%%%%% FP calculations %%%%%%%%%%%%%%%%%%%%%
\sisetup{
    round-mode = places,
    round-precision = 2,
    group-separator = {,},
    zero-decimal-to-integer = true
}
\newcommand{\compute}[2]{\FPeval{\result}{round(#1, #2)}\num{\result}}
\newcommand{\powerTwo}[2]{\FPeval{\result}{round(#1, #2)}\num{\result}}
\newcommand{\tableTwoLines}[2]{\begin{tabular}[c]{@{}c@{}}#1\\ #2\end{tabular}}
\newcommand{\tableCentered}[1]{\begin{tabular}[c]{@{}c@{}}#1\end{tabular}}

%%%%%%%%%%%%%%%%%%%%% Epigraph configuration %%%%%%%%%%%%%%%%%%%%%
\usepackage{calrsfs}
\usepackage{dutchcal}
% Set the desired width for the epigraph
\setlength{\epigraphwidth}{0.8\textwidth}

\begin{document}
% \addtocontents{toc}{\protect\hypertarget{toc}{}}

% Page de garde avec commande \maketitle
% Front cover calling \maketitle
% La page de garde est en français
% The front cover is in French
\selectlanguage{french}

% Inclure les infos de chaque établissement
% Include each institution data
\input{./cover/affiliations.tex}

% Inclure infos de l'école doctorale
% Include doctoral school data
\ecoledoctorale{MathSTIC}

% Inclure infos de l'établissement
% Include institution data
\etablissement{UBS}

%Inscrivez ici votre sp\'{e}cialit\'{e} (voir liste des sp\'{e}cialit\'{e}s sur le site de votre \'{e}cole doctorale)
%Indicate the domain (see list of domains in your ecole doctorale)
\spec{Informatique et Architectures Numériques}

%Attention : le pr\'{e}nom doit être en minuscules (Jean) et le NOM en majuscules (BRITTEF) 
%Attention : the first name in small letters and the name in Capital letters 
\author{William PENSEC}

% Donner le titre complet de la th\`{e}se, \'{e}ventuellement le sous titre, si n\'{e}cessaire sur plusieurs lignes 
%Give the complete title of the thesis, if necessary on several lines
\title{Extension de la Protection des Processeurs Contre les Menaces Physiques et Logicielles par la Sécurisation du Mécanisme DIFT Contre les Attaques par Injections de Fautes}
\lesoustitre{Enhanced Processor Defence Against Physical and Software Threats by Securing DIFT Against Fault Injection Attacks}

%Indiquer la date et le lieu de soutenance de la th\`{e}se 
%indicates the date and the place of the defense 
\date{19/12/2024}
\lieu{Lorient}

%Indiquer le nom du (ou des) laboratoire (s) dans le(s)quel(s) le travail de th\`{e}se a \'{e}t\'{e} effectu\'{e}, indiquer aussi si souhait\'{e} le nom de la (les) facult\'{e}(s) (UFR, \'{e}cole(s), Institut(s), Centre(s)...), son (leurs) adresse(s)... 
%Indicates the name (or names) of research laboratories where the work has been done as well as (if desired) the names of faculties (UFR, Schools, institution...
\uniterecherche{UMR CNRS 6285, Lab-STICC, Université Bretagne Sud}

%Indiquer le Numero de th\`{e}se, si cela est opportun, ou laisser vide pour faire disparaitre cet ligne de la couverture
%Indicate the number of the thesis if there is one. otherwise leave empty so the line disappeurs on the cover
\numthese{715} % \numthese{}

\jury{
{\normalTwelve \textbf{Rapporteurs avant soutenance :}}\\ \newline
\footnotesizeTwelve
\begin{tabular}{@{}ll}
Lejla BATINA & Professeur des Universités (Radboud University, Pays-Bas) \\
Nele MENTENS & Professeur des Universités (Leiden University, Pays-Bas et KU Leuven, Belgique) \\
Vincent BEROULLE & Professeur des Universités (INP - Université Grenoble Alpes, France) \\
\end{tabular}

\vspace{\baselineskip}
{\normalTwelve \textbf{Composition du Jury :}}\\ \newline
\footnotesizeTwelve
\begin{tabular}{@{}lll}
Pr\'{e}sident :        & Jean-Max DUTERTRE & Professeur des Universités (Ecole des Mines de Saint-Etienne) \\
Examinateurs :         & Francesco REGAZZONI & Professeur des Universités (University of Amsterdam et\\ & & Università della Svizzera italiana) \\
Dir. de th\`{e}se :    & Guy GOGNIAT & Professeur des Universités (Lab-STICC, Université Bretagne Sud) \\
Co-dir. de th\`{e}se : & Vianney LAP\^OTRE & Maitre de Conférence HDR (Lab-STICC, Université Bretagne Sud) \\
\end{tabular}
}

\maketitle


% Sélectionner la langue du contenu suivant cette ligne
% Select the content language following this line
\selectlanguage{english}

% \setlength{\headheight}{14pt}
% \addtolength{\topmargin}{-1.6pt}

\thispagestyle{empty}
\pagenumbering{roman}

% Inclusion du chapitre remerciement
% Input acknowledgement chapter
\clearemptydoublepage
\vspace*{\fill}

\epigraph{{\large $\mathcal{A}$d mentes inquisitivas quae lucem futuri Scientiae accendunt.}\\
{\footnotesize \textit{Aux esprits curieux qui illuminent l'avenir de la Connaissance.}}\\
{\footnotesize \textit{To the inquisitive minds that are lighting up the future of Knowledge.}}}{}

\vspace*{\fill}

% Inclusion du chapitre remerciement
% Input acknowledgement chapter
\clearemptydoublepage
\chapter*{Remerciements}

Tout d'abord, je tiens à remercier mon directeur de thèse, Guy Gogniat, Professeur, ainsi que mon codirecteur de thèse, Vianney Lapôtre, Maitre de Conférences HDR, tous les deux à l'Université Bretagne Sud, à Lorient. Leur accompagnement, expertise et soutien ont été plus que précieux durant cette thèse.

Je remercie également Lejla Batina, Nele Mentens et Vincent Beroulle, respectivement Professeurs des Universités à Radboub (Pays-Bas), KU leuven (Belgique), et à Grenoble, pour avoir accepté de rapporter ma thèse. Leurs remarques ont été pertinentes et m'ont permis d'améliorer mon manuscrit.

Je souhaite également remercier Jean-Max Dutertre, Professeur à l'\'Ecole des Mines de Saint-\'Etienne, à Gardanne, qui a accepté de participer à mon Comité de Suivi de thèse (CSI) ainsi que d'avoir accepté de faire partie de mon jury de thèse. Je remercie également Karine Heydemann pour avoir fait partie de mon CSI.

Je remercie grandement Francesco Regazzoni, Professeur à l'Università della Svizzera Italiana, à Lugano (Suisse) et à l'Université d'Amsterdam (Pays-Bas) pour avoir accepté de faire partie de mon jury de thèse et pour m'avoir guidé durant ma mobilité. J'ai beaucoup apprécié les échanges que nous avons pu avoir. Cela a contribué positivement à mon travail, la preuve étant avec les contributions scientifiques que cela a amené. Cela m'a permis d'étendre mes connaissances en sécurité ainsi que d'améliorer mon anglais. J'ai pu rencontrer de nouveaux chercheurs, doctorants et postdoctorants au laboratoire et d'échanger avec eux, ce qui m'a enrichi personnellement et professionnellement. J'ai eu la chance de découvrir un endroit formidable durant ces cinq mois et d'en garder un souvenir incroyable. J'espère pouvoir y retourner bientôt.

De plus, je souhaite remercier mes collègues au Lab-STICC, Nicolas, Mohamed, Noura, Hongwei, Tianxu, Clément, et tous les autres, ainsi que les personnes que j'ai pu rencontrer durant ma thèse. Un grand merci à Tom et Chiara, rencontrés lors de mon séjour à Lugano. Ils m'ont tous deux aidé à m'intégrer dans cette nouvelle ville. Nous avons également eu des discussions très intéressantes. Je vous dis à bientôt, j'espère. Je souhaite enfin remercier mes enseignants de Licence et Master (particulièrement Catherine Dezan et David Espès) à l'Université de Bretagne Occidentale d'avoir cru en moi et m'avoir offert la possibilité de réaliser des stages en recherche.

Finalement, je conclurai en remerciant mes parents, mon frère, ma copine Hellen, ma famille, ainsi que tous mes amis pour leur support et leur accompagnement pendant toutes ces années. Merci pour vos remarques, vos conseils et votre écoute. Vous m'avez tous permis de mener à bien ce travail jusqu'au bout en me permettant de me sentir toujours bien.
Merci pour tout.

\chapter*{Acknowledgments}

Firstly, I would like to thank my thesis supervisor, Professor Guy Gogniat, and my co-supervisor, Vianney Lapôtre, Associate Professor, both at the Université Bretagne Sud in Lorient. Their guidance, expertise, and support have been invaluable during this thesis.

Secondly, I would like to thank Lejla Batina, Nele Mentens and Vincent Beroulle, respectively Full Professors at Radboub University (Netherlands), KU Leuven (Belgium) and Grenoble, for agreeing to review my PhD thesis. Their comments were pertinent and helped me to improve my manuscript.

I would also like to thank Jean-Max Dutertre, Professor at the Ecole des Mines de Saint-Étienne, Gardanne, who agreed to take part in my thesis monitoring committee (Comité de Suivi de thèse - CSI) and to sit on my thesis jury. I would also like to thank Karine Heydemann for being part of my CSI.

I would like to thank Francesco Regazzoni, Senior Researcher at the Università della Svizzera Italiana in Lugano (Switzerland) and at the University of Amsterdam (Netherlands), for agreeing to sit on my thesis jury and for guiding me during my mobility. I very much appreciated the exchanges we had. It made a positive contribution to my work, as evidenced by the scientific contributions it brought. It enabled me to broaden my knowledge of safety and improve my English. I was able to meet new researchers, PhD students and post-docs in the laboratory and exchange ideas with them, which enriched me personally and professionally. I've been lucky enough to discover a wonderful place during these five months and to have incredible memories of it. I hope to be able to return there very soon.

I would also like to thank my colleagues at the Lab-STICC, Nicolas, Mohamed, Noura, Hongwei, Tianxu, Clément, and all the others, as well as the people I met during my thesis. A big thanks to Tom and Chiara, whom I met during my stay in Lugano. They both helped me at the time to integrate myself in this new city. We also had some very interesting and rewarding discussions. I hope to see you soon. Finally, I'd like to thank my undergraduate and Master's teachers (especially Catherine Dezan and David Espès) at the Université de Bretagne Occidentale for believing in me and giving me the opportunity to do research internships.

Finally, I would like to conclude by thanking my parents, my brother, my girlfriend Hellen, my family and all my friends for their support and guidance over the years. Thank you for your comments, advice, and attentiveness. You have all enabled me to see this work through to the end, making me always feel good.

Thank you very much for everything.

% Inclusion du chapitre remerciement
% Input acknowledgement chapter
\clearemptydoublepage
\chapter*{Abstract}
\label{chapter:abstract}
\minitoc

%%%%%%%%%%%%%%%%%%%%%%%%%%%%%%%%%%%%%%%%%%%%%%%%%%%%%%%%%%%%%%%%%%%%%%%%%%%%%%%%%%%%%%%%%%%%%%%
Embedded systems are increasingly prevalent in critical infrastructures such as industries, smart cities, and biomedical devices, improving efficiency and addressing challenges like climate change and health. However, their widespread use also expands the attack surface, creating significant security risks. These systems, typically powered by low-energy processors handling sensitive data, are vulnerable to both software and physical attacks due to their network connectivity and proximity to potential attackers. Hence, addressing both threats during processor designing is essential.

Dynamic Information Flow Tracking (DIFT) techniques, which detect software attacks like buffer overflow and malware by attaching and propagating tags to data at runtime, are a key defence.
Fault Injection Attacks (FIA) deliberately induce errors in a system's hardware to alter its normal operation, often bypassing security mechanisms. These faults can be introduced via physical methods (e.g., voltage, lasers), leading to potential data breaches or system disruptions. FIAs are particularly concerning in embedded systems and cryptographic devices, where low-level faults can compromise sensitive information. Many studies have shown different vulnerabilities due to FIAs on critical systems but none of them targetted a DIFT mechanism.

We focus on the D-RI5CY~\cite{PDGLC-18-hpec} processor, which implements a hardware-based in-core DIFT. Our primary objective is to assess the impact of FIA on the effectiveness of DIFT in the D-RI5CY processor. Through fault injection simulations, we evaluate the vulnerability of DIFT and identify critical hardware components requiring protection~\cite{PLG-22-SensorsSP}. 
As a result of this evaluation, we implemented two lightweight countermeasures, considering constraints like area and performance: simple parity for error detection and Hamming Code for single-bit error detection and correction~\cite{PRLG-24-isvlsi}. These were optimised by grouping registers to reduce parity/redundancy overhead. The sensitivity evaluation was conducted using FISSA, a tool developed to facilitate fault evaluation at the conceptual stage~\cite{PLG-24-dsd}. This tool allows the enabling of the principle of \textit{Security by Design}.
Finally, we evaluated the security of multiple register group compositions to enhance countermeasure effectiveness against complex fault models. We tested Hamming Code with five group configurations and developed a new version of the code capable of detecting two errors and correcting one (SECDED). This was compared across the same groups in terms of efficiency and area to find the optimal trade-off for embedded systems with strict energy and performance constraints.

% Ne pas oublier cette commande qui g\'{e}n\`{e}re la page de couverture avant
% This command will generate the front cover
\frontmatter
\clearemptydoublepage
\dominitoc % Initialization

\newpage
\renewcommand{\contentsname}{Table of Contents}
\addcontentsline{toc}{chapter}{Table of Contents}
\tableofcontents

\newpage
\renewcommand{\nomname}{Acronyms}
\printnomenclature

\newpage
\addcontentsline{toc}{chapter}{List of Figures}
\listoffigures

\newpage
\addcontentsline{toc}{chapter}{List of Tables}
\listoftables

\newpage
\renewcommand{\lstlistlistingname}{List of Listings}
\addcontentsline{toc}{chapter}{List of Listings}
\lstlistoflistings

%%%%%%%%%%%%%%%%%%%%%%%%%%%%%%%%%%%%%%%%%%%%%%%%%%%%%%%%%%%%%%%%%%%%%%%%%%%%%%%%%%%%%%%%%%%%%%%%%%%%%%%%%%%%%%%%%%%%%%%%%%%%%%%%%%%%%%%%%%%%

\clearemptydoublepage
\pagenumbering{arabic}
\setcounter{page}{1}
\setcounter{mtc}{8}
\mainmatter
\chapter{Introduction}
\chaptermark{Introduction}
\label{chapter:introduction}

\epigraph{\textit{IoT without security means Internet of Threats}}{Stéphane Nappo}

\minitoc

%%%%%%%%%%%%%%%%%%%%%%%%%%%%%%%%%%%%%%%%%%%%%%%%%%%%%%%%%%%%%%%%%%%%%%%%%%%%%%%%%%%%%%%%%%%%%%%
\section{Context}
An embedded system is a specialised computing system designed to perform dedicated functions or tasks within a larger mechanical or electrical system. Unlike general-purpose computers, embedded systems are optimised for specific control operations and are typically integrated into the hardware they manage. These systems are characterised by their compact size, low power consumption, and real-time performance constraints. They consist of microcontrollers or microprocessors, along with memory and input/output interfaces, tailored to meet the precise requirements of the application they serve. Embedded systems are ubiquitous in modern technology, powering a wide range of devices from household appliances and medical equipment to industrial machines and automotive systems, ensuring efficiency, reliability, and functionality in their operations.

\begin{figure}[ht]
    \centering
    \includegraphics[width=\textwidth]{c1_intro/img/iot_forecasts.pdf}
    \caption{Number of IoT (IoT) devices worldwide from 2022 to 2033 (from~\cite{statista_iot})}
    \label{fig:iot_forecasts}
\end{figure}

The Internet of Things (IoT) has revolutionised the way we interact with technology, enabling seamless connectivity and communication between a myriad of devices. These devices are part of our daily lives, from the connected light bulb to autonomous cars. These devices collect and share data about how they are used and the environment in which they operate. Immense amounts of data are also being generated by connected cars, production, and transport applications. Today, Industrial IoT (IIoT) represents the largest and fastest-growing volume of data.
To capture data, they rely on sensors embedded in every physical device, such as mobile phones, smartwatches, medical devices (pacemakers, cardiac defibrillators, etc.), but also in recent cars, or in agriculture to monitor humidity, temperature, or automate the irrigation system. These sensors generate data that can be critical, and as these data exist, they are subjects of cyber-attacks.
According to forecasts, the number of IoT devices in use worldwide is estimated to reach approximatively 40 billion in 2033~\cite{statista_iot}, as shown in Figure~\ref{fig:iot_forecasts}, while, today, in 2024, we count around 18 billion. The economic impact of IoT is substantial, with worldwide consumer IoT revenue expected to rise from \$181.5 billion in 2020 to \$621.6 billion by 2030~\cite{statista_iot_revenu} as shown in Figure~\ref{fig:iot_revenue}.
As IoT continues to expand its reach, the importance of ensuring robust security in these systems becomes increasingly critical. IoT devices, often characterised by limited resources and large-scale deployment, present unique security and privacy challenges.

\begin{figure}[ht]
    \centering
    \includegraphics[width=\linewidth]{c1_intro/img/iot_revenue.pdf}
    \caption{Internet of Things total annual revenue worldwide from 2020 to 2030 (from~\cite{statista_iot_revenu})}
    \label{fig:iot_revenue}
\end{figure}


Embedded systems, which form the backbone of IoT devices, are increasingly vulnerable to both software and hardware threats, as well as network-based threats, which can lead to data leaks or unauthorised access to essential system components. These systems are frequently deployed in environments where they are exposed to potential adversaries, making them attractive targets for various types of attack~\cite{MW-19-compnet, EAJJMB-22-compscrev}.

Software security is a critical aspect of the development and deployment of software systems, encompassing measures and practices designed to protect applications from malicious attacks, vulnerabilities, and other security risks. It involves the implementation of protocols to ensure the confidentiality, integrity, and availability of software and data. This field addresses a wide range of threats, including but not limited to, malware~\cite{FIMI-23-access}, memory overflow attacks~\cite{CWCBW-00-discex}, SQL injection~\cite{DGWKU-16-notere}, and Cross-site scripting (XSS)~\cite{ST-12-computer}. Effective software security practices include rigorous code reviews, the use of secure coding standards, regular vulnerability assessments, and the deployment of encryption and authentication mechanisms. As software becomes increasingly integral to various aspects of daily life and business operations, ensuring its security is paramount to safeguarding sensitive information, maintaining user trust, and preventing financial and reputational damage.

Network attacks, such as Distributed Denial of Service (DDoS) attacks, can overwhelm an embedded system's network interface, rendering it inoperative, while man-in-the-middle attacks~\cite{CDL-16-commsurtuto} intercept and potentially alter communication between devices, Internet Protocol spoofing~\cite{MRT-10-isf}, jamming~\cite{PZ-22-commsurtuto}, and many others. These vulnerabilities can be exploited to leak confidential data, corrupt system functionality, or gain control over critical system operations, underscoring the urgent need for robust security mechanisms in embedded systems.

On the hardware front, physical attacks refer to different techniques and methods aimed at compromising the security of embedded systems. These attacks exploit vulnerabilities in the physical layer or implementation of the device’s hardware to delete, modify, gain or prevent access to confidential data.
The most common physical attacks are Side-Channel Attacks (SCA) and Fault Injection Attacks (FIA).

Side-channel attacks~\cite{DM-21-appiot} are passive physical attacks that primarily aim to exploit leakages of information from a device, such as power consumption, electromagnetic emissions, or timing information. By capturing and analysing these side-channel data, attackers can infer sensitive information, such as cryptographic keys~\cite{K-96-crypto}.

Fault injection attacks~\cite{BCNTW-06-procieee, BBKN-12-procieee, YSW-18-hss} are active physical attacks, noninvasive or invasive, transient or permanent, where the attacker intentionally try to change the normal behaviour of a device during program execution by injecting one or more faults, then observing the erroneous behaviour that could be further exploited as a vulnerability. Boneh et al.~\cite{BDL-97-eurocrypt} introduced fault injection attacks. They were able to break some cryptographic protocols by inducing faults into the computations.

In this dissertation, we only study and present fault injection attacks. Nowadays, these attacks are more and more easier to make. For example, NetSPI introduced, in the Black Hat conference in Las Vegas, in August 2024, a new laser hacking device called the RayV Lite~\cite{rayvlite_wired}. The authors, Sam Beaumont and Larry "Patch" Trowell, presented their open-source tool that aims to let anyone achieve laser-based tricks to reverse engineer chips and trigger their vulnerabilities. There are already some tools such as Riscure Laser Station~\cite{riscure_station} who costs between \$10,000 and \$150,000. In the same way as NewAE~\cite{chipwhisperer, chipshouter} with their ChipWhisperer or ChipShouter that allow to realise clock glitching, voltage glitching or even electromagnetic injection at a lower cost and more accessible, RayV Lite allows people to perform laser-based attacks for only \$500 which is more accessible and cheaper than any other tools available. M. S. Kelly and K. Mayes~\cite{KM-20-host} have shown that with cheap components they are able to make a laser setup for around \$500. The low cost and relative ease of construction of their laser environment suggests that developers of IoT devices need to seriously consider this threat on their devices, because it must be assumed that these attack techniques are readily available to malicious attackers.

Many studies have shown the vulnerabilities of critical systems against FIAs.
\cite{LBDP-19-date} demonstrates that it is possible to recover computed secret data using FIA in hidden registers on the RISC-V Rocket processor. 
Electromagnetic fault injection (EMFI) attack can be used to recover an AES key by targeting the cache hierarchy and the MMU, as shown in~\cite{TBELB-21-jce}.
Laser fault injections (LFI) can allow the replay of instructions~\cite{KDD-21-dsd}, that can lead to the overwriting of an entire section of a program.
\cite{TSW-16-fdtc} shows the use of glitch injections on the power supply to control the program counter (PC). Voltage glitches can also lead to glitch TrustZone mechanisms, as shown in~\cite{SMS-23-usenix}.
Finally, authors of~\cite{NSUH-21-tches} have shown that one can combine side-channel attacks (SCA) and FIAs to bypass the PMP mechanism in a RISC-V processor.

Thus, the main research question of this work is how can we maintain maximum protection against software attacks in the presence of physical attacks ?

%%%%%%%%%%%%%%%%%%%%%%%%%%%%%%%%%%%%%%%%%%%%%%%%%%%%%%%%%%%%%%%%%%%%%%%%%%%%%%%%%%%%%%%%%%%%%%%
\section{Objectives}

In this dissertation, we address a part of the threats that IoT devices faces, with a particular emphasis on security threats affecting the software and hardware layers of a device. The main objective is to provide a robust security mechanism against both software and physical threats, where the attacker performs a fault injection attack to bypass a software security mechanism in order to realise a software attack.
We rely on a security mechanism called Dynamic Information Flow Tracking (DIFT) to protect the system against software attacks. This mechanism is presented in Chapter~\ref{section:ift}.

The first contribution of this dissertation is to show that this mechanism is vulnerable to fault injection attacks, using an HDL simulator tool to simulate the behaviour of a processor in the presence of fault injections targeting the DIFT mechanism at runtime.

The second contribution is the development of a tool for automating the simulation process on a given processor design. This open-source tool is available on GitHub and can be used during the development process to find the vulnerabilities of an HDL design. Thanks to this tool, the designer is able to check his design right from the conceptual phase in order to have a robust design against fault injection attacks, enabling the notion of \textit{Security by Design}.

The third contribution is the implementation of two lightweight countermeasures inside the DIFT mechanism to protect it against fault injection attacks. For the countermeasures, we take into account various constraints such as area, and performance overhead.

Finally, in our last contribution, we evaluate different implementations of lightweight countermeasures to protect the mechanism against stronger fault model.


%%%%%%%%%%%%%%%%%%%%%%%%%%%%%%%%%%%%%%%%%%%%%%%%%%%%%%%%%%%%%%%%%%%%%%%%%%%%%%%%%%%%%%%%%%%%%%%
\section{Manuscript outline}

This work is segmented in seven chapters, the first being this introduction.

Chapter~\ref{chapter:soa} presents the state of the art of this dissertation and define the different technical terms. Firstly, it presents Information Flow Tracking (IFT), and its different types.
Secondly, this chapter presents physical attacks, focusing on the two mains types: Side-Channel Attacks and Fault Injection Attacks.
Finally, the chapter presents an overview of the literature about countermeasures against Fault Injection Attacks, and provides a small discussion on their advantage and disadvantages.

Chapter~\ref{chapter:dift_assessment} presents the background of this work with the presentation of the RISC-V Instruction Set Architecture (ISA), the architecture of the D-RI5CY core in detail. Then, the different use cases are presented in details, highlighting their software vulnerability which can be detected by a DIFT mechanism. Finally, a vulnerability assessment is done to show how the considered DIFT mechanism is vulnerable against FIA in these examples and where.

Chapter~\ref{chapter:fissa} introduces a new tool, FISSA, to automatise fault injection campaigns in simulation. This tool allows a designer to assess his design during the conception phase. This chapter presents its software architecture and how to use it, and compares it to others tools available in the literature.

Chapter~\ref{chapter:countermeasures} details the different implementation of lightweight countermeasures to protect the D-RI5CY core against FIA, taking into account common fault models. Then we evaluate these protections in terms of area, performance, and efficiency.

Chapter~\ref{chapter:exp_setup_results} evaluate the countermeasures performances against more complex fault models, and we introduce a third countermeasure against more complex fault models. Then we evaluate these protections in terms of area, performance, and efficiency.

Chapter~\ref{chapter:conclusion} is dedicated to the summary of this dissertation with a short discussion on the obtained results, identifying limitations, and discussing the challenges encountered in this thesis.
We also explore future research perspectives at short and long terms, and suggest potential improvements.



%%%%%%%%%%%%%%%%%%%%%%%%%%%%%%%%%%%%%%%%%%%%%%%%%%%%%%%%%%%%%%%%%%%%%%%%%%%%%%%%%%%%%%%%%%%%%%%

\clearemptydoublepage
\chapter{State of the Art}
\chaptermark{State of the Art}
\label{chapter:soa}
\minitoc

%%%%%%%%%%%%%%%%%%%%%%%%%%%%%%%%%%%%%%%%%%%%%%%%%%%%%%%%%%%%%%%%%%%%%%%%%%%%%%%%%%%%%%%%%%%%%%%
\section{Introduction}
This chapter provides an overview of related work to contextualize the primary objectives of this thesis. Firstly, in Section~\ref{section:ift}, Information Flow Tracking (IFT) is introduced, detailing the different types and their respective purposes. We will discuss the various levels of monitoring, from program behaviour to the detection of hardware trojans.
Then in Section~\ref{section:physicalAttacks}, Physical Attacks are examined, focusing on two main types: Side-Channel Attacks (SCA) and Fault Injection Attacks (FIA).
Finally in Section~\ref{section:countermeasuresAgainstFIA}, as this work will concentrate on FIA, we will exclusively present countermeasures against Fault Injection Attacks.

%%%%%%%%%%%%%%%%%%%%%%%%%%%%%%%%%%%%%%%%%%%%%%%%%%%%%%%%%%%%%%%%%%%%%%%%%%%%%%%%%%%%%%%%%%%%%%%
\section{Information Flow Tracking}
\label{section:ift}
This section introduces Information Flow Tracking mechanisms, explains how they work, and presents the various types of IFT with their different functional levels.
% We mainly focus on presenting hardware IFT architectures.
    
%%%%%%%%%%%%%%%%%%%%%%%%%%%%%%%%
\subsection{Different types of IFT}
There are two distinct types of IFT approaches: static and dynamic, each with its own specific objectives.

\subsubsection{Static IFT}
Static Information Flow Tracking (SIFT) is a security technique used to analyse and control the flow of information within a program or system without executing it, by examining the source code or compiled binary~\cite{HAK-21-acmcsur}. This method is particularly useful for identifying theoretical vulnerabilities, ensuring compliance with design principles, and preventing unauthorised information leaks before deployment. SIFT is comprehensive, covering all possible execution paths and detecting both explicit information flows (direct data assignments) and implicit flows (leaks through control flow structures). By performing checks at compile-time, SIFT helps developers address potential security issues early, enforcing principles like non-interference and data confidentiality through security policies. However, static analysis may generate false positives by flagging theoretical flows that might not occur in practice and may struggle with certain dynamic language features or runtime-dependent behaviours. SIFT is employed in various contexts, such as verifying secure information flow in operating systems, programming languages with built-in information flow controls, and hardware design for secure systems.

\subsubsection{Dynamic IFT}
Dynamic Information Flow Tracking (DIFT) is a powerful security technique that monitors and analyses, in real-time, the flow of information within a program during its execution~\cite{CGDJ-21-micromac}. DIFT operates by tagging or labelling input data from potentially untrusted sources and tracking how this data propagates through the system~\cite{SLD-04-sigplan}. As the program executes, DIFT maintains metadata about the tagged information, updating it as operations are performed on the data. This allows the system to detect when tainted data is used in security-critical operations, such as modifying control flow or accessing sensitive resources. DIFT can be implemented at various levels, including hardware, software, or a combination of both. Hardware-based implementations often offer better performance but require specialized processor modifications, while software-based approaches provide more flexibility but may incur higher overhead~\cite{CGDJ-21-micromac}. DIFT has proven effective in detecting and preventing a wide range of security vulnerabilities, including buffer overflows, format string attacks, and code injection attacks~\cite{SLD-04-sigplan}. However, DIFT also faces challenges, such as handling implicit information flows, managing performance overhead, and addressing over-tainting issues.
This approach might not cover all potential data paths, as it is dependent on the specific conditions and inputs provided during the monitoring period.
Despite these challenges, DIFT remains a valuable tool for software security, particularly for runtime attack detection in modern systems.

%%%%%%%%%%%%%%%%%%%%%%%%%%%%%%%%    
\subsection{Different levels of DIFT}
IFT can be implemented at various levels of abstraction in computing systems. Each level presents unique trade-offs between precision, performance overhead, and ease of implementation, allowing designers to choose the most appropriate approach for their security requirements.

Software-based DIFT mechanisms benefit from close integration with the software context via binary code instrumentation and source code modifications, offering better flexibility, customisation, and scalability without altering hardware components. However, these software solutions often incur high performance overheads due to the extra instructions required. They operate at either the system level, monitoring OS-wide information flows, or the program level, focusing on specific applications.
On the other hand, hardware-assisted DIFT designs can efficiently enforce security rules by implementing DIFT-related operations as hardware logic, reducing performance overhead but at the expense of flexibility and scalability, making them challenging to deploy in modern commercial systems. They can be implemented within processor cores or as off-core designs. But they can also be at the lowest level such as Gate-Level IFT who tracks information flow through logic gates.
A hybrid hardware and software co-design offers a promising alternative, enabling fine-grained security checks by associating software context with hardware data, though it faces challenges such as balancing flexibility with hardware overhead and designing appropriate tags that support rule updates post-deployment.


Figure~\ref{fig:levels_system} represents the different levels of a simplified embedded system: application layer, system service layer, OS layer, and hardware layer. This figure is inspired by Figure 1.9 of~\cite{ebrary}. Software-based IFTs work in the first three levels.

Positioned at the highest level of the software hierarchy, \textit{the application layer} is responsible for implementing system functionalities and business logic. Functionally, all modules within this layer work together to execute the required system operations. Applications generally run in a less-privileged mode on the processor and utilise the OS-provided API scheduling to communicate with the operating system.
\textit{The system service layer} serves as the intermediary service interface offered by the OS to the application layer. This interface allows applications to access a variety of OS-provided services, essentially bridging the gap between the OS and applications. Typically, this layer encompasses components like the file system, Graphical User Interface (GUI), task manager.
An Operating System (OS) is a software framework designed to manage hardware resources uniformly. It abstracts numerous hardware functions and offers them to applications as services. Common services provided by an OS include scheduling, file synchronisation, and networking. Operating systems are prevalent in both desktop and embedded systems. In the context of embedded systems, OSs possess distinct characteristics such as stability, customisability, modularity, and real-time processing capabilities.
\textit{The hardware layer} refers to the physical components and circuitry, including the microprocessor or microcontroller, memory, sensors, and input/output interfaces. This layer encompasses all the tangible electronic elements that interact directly with each other to perform the device's functions. It provides the essential infrastructure that supports and drives the embedded system’s operations and connectivity.

\begin{figure}[ht]
    \centering
    \includegraphics{c2_soa/img/system_layer.pdf}
    \caption{Simplified representation of the different layers in an embedded system}
    \label{fig:levels_system}
\end{figure}

Tracking information can be performed at various levels, from the application level to the hardware level. Each level offers distinct advantages and disadvantages.
%For instance, application-level tracking might provide detailed insights and user-friendly interfaces, while hardware-level tracking offers more granular data and real-time monitoring but can be more complex and costly.
The following subsections will explore these different levels, highlighting their respective benefits and limitations.


\subsubsection{Software-based DIFT}
\paragraph{Application level DIFT} tracks information flows between application variables. The programmer has to integrate data tagging inside his program and use a modified compiler or analyse his program to check if no security violation happened.

One kind of application DIFT is language-based IFT.

\paragraph{OS level and System-based DIFT}
\paragraph*{}
\cite{CQBH-06-iscc}


\subsubsection{Hardware-based DIFT}
This subsection discusses the hardware-based DIFT designs including gate-level DIFT designs and micro-architecture-level DIFT designs. Survey~\cite{HAK-21-acmcsur} presents an overview on all hardware DIFT techniques. They developed a taxonomy for them and use it to classify and differentiate hardware DIFT tools and techniques.


\paragraph{Gate-Level DIFT} include gate-level netlist and also RTL designs.
GLIFT~\cite{TWMMCS-09-asplos} is a well-established IFT technique. The goal is to protect against hardware trojans and unauthorized behaviors. All information flows, both explicit and implicit, are unified at the gate level. GLIFT employs a detailed initialisation and propagation policy to precisely track each bit of information flow, by adding additional logic for each gate used in the design. By analysing how inputs influence outputs, GLIFT accurately measures true information flows and substantially reduces the false positives typically associated with conservative IFT techniques.

Hu et al. established the theoretical foundation for GLIFT~\cite{HOITSMK-11-tcad}. They introduced several algorithms for generating GLIFT logic in large digital circuits. Additionally, the authors identified the primary source of precision discrepancies in GLIFT logic produced by various methods as static logic hazards or variable correlation due to reconvergent fanouts. Many other works have been done on GLIFT to attempt a decrease of the logic complexity.

\paragraph{Off-Core}

\paragraph{Off-Loading}

\paragraph{In-Core}


\subsubsection{Software and Hardware Co-Design-Based DIFT}

%%%%%%%%%%%%%%%%%%%%%%%%%%%%%%%%
\subsection{How DIFT works}
\wip{expliquer DIFT ici ? en détail avec les schémas explicatifs}

%%%%%%%%%%%%%%%%%%%%%%%%%%%%%%%%%%%%%%%%%%%%%%%%%%%%%%%%%%%%%%%%%%%%%%%%%%%%%%%%%%%%%%%%%%%%%%%
\section{Physical Attacks}
\label{section:physicalAttacks}

\subsection{Side-Channel Attacks}
\subsection{Fault Injection Attacks}
% A fault is the cause of an error, that is, an incorrect program or circuit state. If the error caused by the fault does not propagate and the application execution ends normally, the fault is ineffective. On the contrary, the fault is effective if the error affects the application’s execution, causing a failure, an observed behavior different from that expected.
% In the context of electronic circuits, a fault refers to an unintended deviation from the normal operation of the circuit. Faults can occur due to various reasons such as manufacturing defects, environmental factors, ageing, or external interference. These faults can affect the performance, functionality, and reliability of the circuit.

% In fault injection, which is a testing method used to evaluate the robustness and reliability of electronic circuits, a fault is deliberately introduced into the system to observe its behaviour and identify potential vulnerabilities.

%%%%%%%%%%%%%%%%%%%%%%%%%%%%%%%%%%%%%%%%%%%%%%%%%%%%%%%%%%%%%%%%%%%%%%%%%%%%%%%%%%%%%%%%%%%%%%%
\section{Countermeasures against FIA}
\label{section:countermeasuresAgainstFIA}

%%%%%%%%%%%%%%%%%%%%%%%%%%%%%%%%%%%%%%%%%%%%%%%%%%%%%%%%%%%%%%%%%%%%%%%%%%%%%%%%%%%%%%%%%%%%%%%

\clearemptydoublepage
\chapter{D-RI5CY — Vulnerability Assessment}
\chaptermark{D-RI5CY - Vulnerabilities Assessment}
\label{chapter:dift_assessment}
\minitoc

%%%%%%%%%%%%%%%%%%%%%%%%%%%%%%%%%%%%%%%%%%%%%%%%%%%%%%%%%%%%%%%%%%%%%%%%%%%%%%%%%%%%%%%%%%%%%%%
\section{Introduction}
This chapter provides the background of this thesis and the vulnerability assessment. The first section offers a description of the RISC-V Instruction Set Architecture (ISA) and an overview of the specific RISC-V DIFT design under consideration.
The second section details and describes the considered uses cases of this thesis.
Finally, the third section assesses the vulnerabilities of the D-RI5CY, using these three cases.

%%%%%%%%%%%%%%%%%%%%%%%%%%%%%%%%%%%%%%%%%%%%%%%%%%%%%%%%%%%%%%%%%%%%%%%%%%%%%%%%%%%%%%%%%%%%%%%
\section{D-RI5CY}
\label{section:driscy}
In this section, we describe the RISC-V ISA and detail the DIFT design we have chosen to focus on.
We choose to work on a open-source RISC-V core, meaning that we have the ability to access and modify the design according to our needs.

\begin{figure}[t]
    \centering
    \includegraphics[width=\textwidth]{c3_vulnerabilities_assessment/img/RI5CY.pdf}
    \caption{D-RI5CY processor architecture overview. DIFT-related modules are highlighted in red. (inspired by~\cite{PDGLC-18-hpec})}
    \label{fig:driscy}
\end{figure}

%%%%%%%%%%%%%%%%%%%%%%%%%%%%%%%%
\subsection{RISC-V Instruction Set Architecture (ISA)}
RISC-V is an open and free ISA, which was originally developed at University of California, Berkeley, in 2010, and now is managed and supported by the RISC-V Foundation, having more than 70 members including companies such as Google, AMD, or Intel. The architecture was designed with a focus on simplicity and efficiency, embodying the Reduced Instruction Set Computer (RISC) principles. Unlike proprietary ISA, RISC-V is freely available for anyone to use without licensing fees, making it a popular choice for academic research, commercial products, and educational purposes.

Technically, RISC-V features a modular design, allowing developers to incorporate only the necessary components for their specific application, which can significantly reduce the processor's complexity and power consumption. It supports several base integer sets classified by width—mainly RV32I, RV64I, and RV128I for 32-bit, 64-bit, and 128-bit architectures respectively. Each base set can be extended with additional modules for applications requiring floating-point computations (e.g., RV32F, RV64F), atomic operations (e.g., RV32A, RV64A), and more. This modularity and the openness of RISC-V have spurred a wide range of innovations in processor design and applications in areas ranging from embedded systems to high-performance computing.

%%%%%%%%%%%%%%%%%%%%%%%%%%%%%%%%
\subsection{DIFT design}
This thesis focuses on the evaluation of a DIFT against fault injection attacks and the design of dedicated protections. We opted to not develop a Dynamic Information Flow Tracking (DIFT) system from scratch, as this would have required considerable time for implementation and testing, which was not within the scope of our objectives. Consequently, we decided to review the current state of the art and select an open-source DIFT system.
As a result, we have selected the D-RI5CY~\cite{PDGLC-18-hpec, driscy} design, which utilises the RI5CY core supported by PULPino~\cite{pulpino} and developed by PULP platform~\cite{pulp}. This is a 4-stage, in-order, 32-bit RISC-V core optimised for low-power embedded systems and IoT applications. It fully supports the base integer instruction set (RV32I), compressed instructions (RV32C), and the multiplication instruction set extension (RV32M) of the RISC-V ISA. Additionally, it includes a set of custom extensions (RV32XPulp) that support hardware loops, post-incrementing load and store instructions, ALU, and MAC operations.
D-RI5CY has been developed by researchers of Columbia University, USA, in partnership with Politecnico di Torino, Italy. D-RI5CY extends the RI5CY processor to support in-core DIFT.

Figure~\ref{fig:driscy} presents an overview of the D-RI5CY processor's architecture. DIFT modules are represented in red and dark red.
These modules allow tags to be initialised, propagated and checked during the execution of a sensitive application.
The \textit{Tag Update Logic} module is used to initialize or update the tag in the register file according to the tagged data.
Then, when a tag is propagated in the pipeline in parallel to its associated data, the \textit{Tag Propagation Logic} module propagates it according to the propagation policy defined in the \textit{TPR}.
Once a tag has been propagated and its data has been sent out of the pipeline, the \textit{Tag Check Logic} modules check that it conforms to the security policy defined in the TCR. If not, an exception is raised and the application is stopped to avoid accessing or executing corrupted data.

The authors of the D-RI5CY defined a library of routines to initialise the tags of the data coming from potentially malicious channels.
At program startup, D-RI5CY initialises the tags of the registers, program counter and memory blocks to \textit{zero}. The default 1-bit tag is "\textit{0}", this means that the data is trusted, otherwise, the tag would be set to "\textit{1}" which means that the data is untrusted.
They extended the RI5CY ISA with memory and register tagging instructions.
They have added four assembly instructions to initialise tags for user-supplied inputs:
\begin{itemize}
    \item \textbf{p.set rd}: sets to untrusted the security tags of the destination register \textit{rd},% (you can check the register names in the ISA specification\footnote{\url{https://www2.eecs.berkeley.edu/Pubs/TechRpts/2014/EECS-2014-54.pdf}} at page 85),
    \item \textbf{p.spsb x0, offset(rt)}: sets to untrusted the security tags of the memory byte at the address of the value stored in \textit{rt + offset},
    \item \textbf{p.spsh x0, offset(rt)}: sets to untrusted the security tags of the memory half-word at the address of the value stored in \textit{rt + offset},
    \item \textbf{p.spsw x0, offset(rt)}: sets to untrusted the security tags of the memory word at the address of the value stored in \textit{rt + offset}.
\end{itemize}


Moreover, they augmented the program counter with a tag of one bit and the register file with one tag per register's byte (marked as $T$ in Figure~\ref{fig:driscy}). Finally, they added 4-bit tags to the data memory (i.e. 1 tag per byte).
Each data element is physically stored in memory with its associated tag. However, a tag can only have two values as in the Register File Tag, the tag is on one bit.

It is worth noting that the D-RI5CY designers have chosen to rely on the \textit{illegal instruction exception} already implemented in the original RI5CY processor to manage the DIFT exceptions. This choice minimizes the area overhead of the proposed solution.

In the Control and Status Registers (CSR), they added two additional 32-bits registers : Tag Propagation Register (TPR) and Tag Check Register (TCR). These registers are used to store the security policy for both tag propagation and tag check. These registers contain a default policy, and they can be modified during runtime with a simple \textit{csr write} instruction, such as \texttt{csrw \textit{csr, rs1}}.
These policies consist of rules, which have fine-grain control over tag propagation and tag check for different classes of instructions. The rules specify how the tags of the instruction operands are combined and checked.
Table~\ref{tab:insnClasses} shows the different instructions for each category represented in both TPR and TCR.

\begin{table}[t]
    \centering
    \footnotesize
    \caption{Instructions per category}
    \label{tab:insnClasses}
    \begin{tabular}{@{}rl@{}}
        \toprule
        \textbf{Class} & \textbf{Instructions} \\ \midrule
        Load/Store & \textit{LW, LH{[}U{]}, LB{[}U{]}, SW, SH, SB, LUI, AUIPC, XPulp Load/Store} \\
        Logical & \textit{AND, ANDI, OR, ORI, XOR, XORI} \\
        Comparison & \textit{SLTI, SLT} \\
        Shift & \textit{SLL, SLLI, SRL, SRLI, SRA, SRAI} \\
        Jump & \textit{JAL, JALR} \\
        Branch & \textit{BEQ, BNE, BLT{[}U{]}, BGE{[}U{]}} \\
        Integer Arithmetic & \textit{ADD, ADDI, SUB, MUL, MULH{[}U{]}, MULHSU, DIV{[}U{]}, REM{[}U{]}} \\ \bottomrule
    \end{tabular}
\end{table}


Table~\ref{tab:tpr} shows the TPR configurations for the security policies considered in our work.
Each instruction type has a user-configurable 2-bit tag propagation policy field, except for \textit{Load/Store Enable}, which has a 3-bit tag.
The tag propagation policy determines how the instruction result tag is generated according to the instruction operand tags.
For 2-bit fields, value `00' disables the tag propagation and the output tag keeps its previous value, value `01' stands for a logic AND on the 2 operand tags, value `10' stands for a logic OR on the 2 operand tags and value `11' sets the output tag to zero. 
The \textit{Load/Store Enable} field provides a finer-granularity rule to enable/disable the input operands before applying the propagation rule specified in the \textit{Load/Store Mode} field. This extra tag propagation policy is defined through 3 bits. These bits allow enabling the source, source-address, and destination-address tags, respectively.

\begin{table}[t]
    \footnotesize
    \setlength{\tabcolsep}{3pt}
    \centering
    \caption{Tag Propagation Register configuration}
    \label{tab:tpr}
    \begin{tabular}{@{}lcccccccc@{}}
        \toprule
                  & \begin{tabular}{c}Load/Store\\Enable \end{tabular} & \begin{tabular}{c}Load/Store\\ Mode
                \end{tabular}  & \begin{tabular}{c}Logical\\Mode \end{tabular} & \begin{tabular}{c}Comparison\\Mode \end{tabular} & \begin{tabular}{c}Shift\\Mode \end{tabular} & \begin{tabular}{c}Jump\\Mode \end{tabular} & \begin{tabular}{c}Branch\\Mode \end{tabular} & \begin{tabular}{c}Arith\\Mode \end{tabular} \\ 
                  \cmidrule(lr){2-2}\cmidrule(lr){3-3}\cmidrule(lr){4-4}\cmidrule(lr){5-5}\cmidrule(lr){6-6}\cmidrule(lr){7-7}\cmidrule(lr){8-8}\cmidrule(lr){9-9}
        Bit index & 17 16 15          & 13 12           & 11 10        & 9 8             & 7 6        & 5 4       & 3 2         & 1 0        \\ \midrule
        Policy 1  & 0 0 1             & 1  0            & 1  0         & 0 0             & 1 0        & 1 0       & 0 0         & 1 0        \\
        Policy 2  & 1 1 1             & 1  0            & 1  0         & 1 0             & 1 0        & 1 0       & 1 0         & 1 0        \\ 
        \bottomrule
    \end{tabular}
\end{table}

Table~\ref{tab:tcr} shows the TCR configurations considered in our work.
Each instruction type has a user-configurable 3-bits tag control policy field, except for \textit{Execute Check}, \textit{Branch Check} and \textit{Load/Store Check} which have 1, 2 and 4-bits tag control policy fields respectively.
The tag control policy determines whether the integrity of the system is corrupted based on the tags of the instruction's operands. 
The default 3-bits field should be read as follows: the right bit corresponds to input operand 1, the middle bit corresponds to input operand 2 and the left bit corresponds to the output tag of the operation. For each bit set, the corresponding tag is checked to determine whether an exception must be raised.
The \textit{Execute Check} field is used to check the integrity of the PC. 
The \textit{Branch Check} field is used to check both inputs during branch instructions. The right bit is used for input operand 1 and the left bit is used for input operand 2.
Finally, the \textit{Load/Store Check} field is used to enable/disable source or destination tags checking during a \textit{load} or \textit{store} instruction. These bits enable or disable the checking of the source tag, source address tag, destination tag and destination address tag.

\begin{table}[t]
    \footnotesize
    \setlength{\tabcolsep}{3pt}
    \centering
    \caption{Tag Check Register configuration}
    \label{tab:tcr}
    \begin{tabular}{@{}lcccccccc@{}}
        \toprule
                  & \begin{tabular}{c}Execute\\Check \end{tabular} & \begin{tabular}{c}Load/Store\\Check \end{tabular}   & \begin{tabular}{c}Logical\\Check \end{tabular} & \begin{tabular}{c}Comparison\\Check \end{tabular} & \begin{tabular}{c}Shift\\Check \end{tabular} & \begin{tabular}{c}Jump\\Check \end{tabular} & \begin{tabular}{c}Branch\\Check \end{tabular} & \begin{tabular}{c}Arith\\Check \end{tabular} \\\cmidrule(lr){2-2}\cmidrule(lr){3-3}\cmidrule(lr){4-4}\cmidrule(lr){5-5}\cmidrule(lr){6-6}\cmidrule(lr){7-7}\cmidrule(lr){8-8}\cmidrule(lr){9-9}
        Bit index & 21            & 20 19 18 17      & 16 15 14      & 13 12 11         & 10 9 8      & 7 6 5      & 4 3          & 2 1 0       \\ \midrule
        Policy 1   & 1             & 1 0 1 0          & 0 0 0         & 0 0 0            & 0 0 0       & 0 0 0      & 0 0          & 0 0 0       \\
        Policy 2 & 0             & 0 0 0 0          & 0 0 0         & 0 0 0            & 0 0 0       & 0 0 0      & 0 0          & 0 1 1       \\
        \bottomrule
    \end{tabular}
\end{table}

To summarise, at first~\textcircled{\small{1}}, TPR and TCR are configured from the default security policy.
Then at program startup~\textcircled{\small{2}}, the tags are set to \textit{trusted} (i.e, set to 0) or \textit{untrusted} (i.e, set to 1) depending on their source or according to the code of the program as the developer can specify some untrusted part of his code.
The tag propagation~\textcircled{\small{3}} and verification~\textcircled{\small{4}} happen in the D-RI5CY pipeline in parallel with the standard behaviour, without incurring any latency overhead.

%%%%%%%%%%%%%%%%%%%%%%%%%%%%%%%%
\subsection{Pedagogical case study}
\label{section:pedagogical_usecase}
To present the use of the D-RISCY, we will introduce a use case to demonstrate how to use a new security policy and how the DIFT will detect the violation of different security policies.
This use case has been developed for pedagogical purposes but does not involve a real software attack.

In order to specify an untrusted part in the code, the developer has to use an assembly line in C which is constructed from keywords \textit{asm volatile}. The template for this assembly line is: "\textit{asm asm-qualifiers ( AssemblerTemplate : OutputOperands [ : InputOperands [ : Clobbers ] ])}".
So to explain briefly, line 7 in Listing~\ref{code:compcompu} is composed of a custom assembly instruction "\texttt{p.spsw}", that takes the "\texttt{x0}" register as target and specifies an address mode using the placeholder "\texttt{0(\%0)}". Finally, \mbox{"\textit{:: "r" (\&a)}"} part specifies the input operand, with "\texttt{r}" indicating that a general-purpose register should be used to hold the address of the variable "\texttt{a}".

Listing~\ref{code:compcompu} shows the C code used for this use case. Lines 2 to 4 initialize variables, lines 5 and 6 configure a security policy by writing to the TPR and TCR registers thanks to an assembly line. Line 7 tags the variable "\verb|a|" as untrusted (tag is set to "\textit{1}"). In line 8, variables "\verb|a|" and "\verb|b|" are compared to determine which arithmetic operation should be performed.
Lines 9 to 21 detail the assembly code generated from the line 8 C statement. It executes the operations according to the values of "\verb|a|" and "\verb|b|" stored in the registers "\verb|a4|" and "\verb|a5|". The "\verb|(a>b)|" condition and its associated branch is computed in line 9, the "\verb|(a-b)|" subtraction in line 14 and the "\verb|a+b|" addition in line 20.

In terms of security policy, depending on which policy is used in Table~\ref{tab:tpr} and Table~\ref{tab:tcr}, we would have different results of exception.
Security policy 1 propagates the tags with an \textit{OR} logic for five modes (arithmetic, jump, shift, logical, and load/store mode) and enables the propagation of the tag from the source of a load/store.
Security policy 1 checks the tags only for the \textit{Execute Check} (i.e., PC instruction) and for the source address and destination address for a load/store instruction.
In comparison, security policy 2 enables the propagation of all tags and checks tags only for both inputs of arithmetic instructions.
To summarise from our application case, if we use security policy 1, the DIFT will detect the \textit{load} instruction before executing the "\verb|a > b|" comparison and raise an exception; whereas if we use security policy 2, the DIFT protection raises an exception when executing the instruction \verb|add a5,a4,a5| (i.e., the "\verb|a+b|" C statement), since variable \verb|a| is untrusted and \verb|b > a|.

\begin{lstlisting}[style=topPosition, caption=Compare/Compute C Code, language=C, label=code:compcompu]
    int main(){
       int a, b = 5, c;
       register int reg asm("x9");
       a = reg;
       asm volatile("csrw 0x700, tprValue");
       asm volatile("csrw 0x701, tcrValue");
       asm volatile("p.spsw x0, 0(\%0);" :: "r" (&a));
       c = (a > b) ? (a-b) : (a+b);
           //42c:   ble a4,a5,448
           //430:   addi a5,s0,-16
           //434:   lw a4,-12(a5)
           //438:   addi a3,s0,-16
           //43c:   lw a5,-4(a3)
           //440:   sub a5,a4,a5
           //444:   j 45c
           //448:   addi a5,s0,-16
           //44c:   lw a4,-12(a5)
           //450:   addi a3,s0,-16
           //454:   lw a5,-4(a3)
           //458:   add a5,a4,a5
           //45c:   sw a5,-24(s0)
       return EXIT_SUCCESS;
    }\end{lstlisting}

In the continuation of this work, this use case will be referred to as \textit{Compare/Compute}, implementing security policy 2 from Table~\ref{tab:tpr} and Table~\ref{tab:tcr}. The two other use cases will be presented in the following section~\ref{section:uses_cases}.

%%%%%%%%%%%%%%%%%%%%%%%%%%%%%%%%%%%%%%%%%%%%%%%%%%%%%%%%%%%%%%%%%%%%%%%%%%%%%%%%%%%%%%%%%%%%%%%
\section{Use cases}
\label{section:uses_cases}

This section details the considered use cases in our work. The first two use cases come from the original paper~\cite{PDGLC-18-hpec}. The third use case, presented in section~\ref{section:pedagogical_usecase}, is a home-made case which is used to stimulate DIFT elements that are not in others use cases.

%%%%%%%%%%%%%%%%%%%%%%%%%%%%%%%%%%%%%%%%%%
\subsection{First use case: Buffer Overflow}
The first use case involves exploiting a buffer overflow, potentially leading to a Return-Oriented Programming\footnote{\hfill\url{https://en.wikipedia.org/wiki/Return-oriented_programming}} (ROP) attack\footnote{\url{https://github.com/sld-columbia/riscv-dift/blob/master/pulpino\_apps\_dift/wilander\_testbed/}} and the execution of a shellcode.

\begin{figure}[ht]
    \centering
    \begin{subfigure}[b]{0.49\textwidth}
        \includegraphics[width=\textwidth, page=1]{c3_vulnerabilities_assessment/img/buffer_overflow/schemaPedagogique.pdf}
        \caption{Initialisation}
        \label{fig:rop_attack_1}
    \end{subfigure}
    \hfill
    \begin{subfigure}[b]{0.49\textwidth}
        \includegraphics[width=\textwidth, page=2]{c3_vulnerabilities_assessment/img/buffer_overflow/schemaPedagogique.pdf}
        \caption{Copy of the source buffer into the destination buffer}
        \label{fig:rop_attack_2}
    \end{subfigure}
    \hfill
    \begin{subfigure}[b]{0.49\textwidth}
        \includegraphics[width=\textwidth, page=3]{c3_vulnerabilities_assessment/img/buffer_overflow/schemaPedagogique.pdf}
        \caption{An overflow occurs, the $ra$ register is overwritten}
        \label{fig:rop_attack_3}
    \end{subfigure}
    \hfill
    \begin{subfigure}[b]{0.49\textwidth}
        \includegraphics[width=\textwidth, page=4]{c3_vulnerabilities_assessment/img/buffer_overflow/schemaPedagogique.pdf}
        \caption{Corrupted $ra$ register is loaded into the PC}
        \label{fig:rop_attack_4}
    \end{subfigure}
    \hfill
    \begin{subfigure}[b]{0.49\textwidth}
        \includegraphics[width=\textwidth, page=5]{c3_vulnerabilities_assessment/img/buffer_overflow/schemaPedagogique.pdf}
        \caption{PC address instruction is fetched}
        \label{fig:rop_attack_5}
    \end{subfigure}
    \caption{Representation of how the ROP attack works}
    \label{fig:rop_attack}
\end{figure}

The attacker exploits the buffer overflow to access the return address (\textit{RA}) register.
Figure~\ref{fig:rop_attack} represents the five steps from the source buffer initialisation to the first shellcode instruction being fetched. In Figure~\ref{fig:rop_attack_1}, the source buffer, in yellow, is initialised with A's, and as it is manipulated by a user, it is tagged as untrusted (red). The destination buffer is empty, and both \textit{PC} and \textit{RA} register are trusted (green). In Figure~\ref{fig:rop_attack_2}, the source buffer is copied into the destination buffer, the data and its tag are copied. In Figure~\ref{fig:rop_attack_3}, the overflow occurs, and the \textit{RA} register is compromised with the address of the shellcode function from the source buffer. Now, all the memory tags are untrusted.
When the function returns, the corrupted \textit{RA} register is loaded into the \textit{PC} via a \textit{jalr} instruction (Figure~\ref{fig:rop_attack_4}). This hijacks the execution flow, causing the first shellcode instruction to be fetched from address: \textit{0x6fc} (Figure~\ref{fig:rop_attack_5}).  Due to the DIFT mechanism, the tag associated with the buffer data overwrites the \textit{RA} register tag.
As the buffer data is user-manipulated, it is tagged as \textit{untrusted} (tag value = 1).
Consequently, when the first shellcode instruction is fetched, the tag associated with the \textit{PC} propagates through the pipeline. At this moment, the DIFT mechanism detects the untrusted tag and as the security policy do not allow executing an untrusted PC, an exception will be raised and the application will be stopped.
This attack demonstrates the behaviour of DIFT when monitoring the \textit{PC} tag.
This use case employs the first security policy from Table~\ref{tab:tpr} and Table~\ref{tab:tcr}.

To illustrate the use of TCR and TPR registers, we assume that buffer data tags are set to 1 (i.e., \textit{untrusted}) since the user manipulates the buffer.
To detect this kind of attack, it is necessary to ensure the PC integrity by prohibiting the use of untrusted data for this register (i.e., \textit{Execute Check} field of TCR set to 1). Regarding tag propagation configuration, load, and store input operand tags must be propagated to output. Thus, the TPR register \textit{Load/Store Mode} field should be set to value 10 (i.e. destination tag = source tag) and the \textit{Load/Store Enable} field must be set to 001 (i.e., Source tag enabled).

Listing~\ref{code:buffer_overflow} displays the C code for the buffer overflow scenario. The assembly code on line 22 of this listing represents the saving of the register \textit{x8}, which is the \textit{saved register 0} or \textit{frame pointer} register in the RISC-V ISA. Next, the source buffer is filled with A's characters and the shellcode address is appended to the end of this source buffer. Finally, lines 30-33 illustrate the tag initialisation on the source buffer.

\begin{lstlisting}[style=topPosition, language=C, label=code:buffer_overflow, caption=Buffer overflow C code]
#define BUFSIZE 16
#define OVERFLOWSIZE 256

int base_pointer_offset;
long overflow_buffer[OVERFLOWSIZE];

int shellcode() {
    printf("Success !!\n");
    exit(0);
}

void vuln_stack_return_addr(){
    long *stack_pointer;
    long stack_buffer[BUFSIZE];
    char propolice_dummy[10];
    int overflow;
    
    /* Just a dummy pointer setup */
    stack_pointer = &stack_buffer[1];
    
    /* Store in i the address of the stack frame section dedicated to function arguments */
    register int i asm("x8");  
    
    /* First set up overflow_buffer with 'A's and a new return address */
    overflow = (int)((long)i - (long)&stack_buffer);
    memset(overflow_buffer, 'A', overflow-4);
    overflow_buffer[overflow/4-1] = (long)&shellcode;

    /* TAG INITIALISATION */
    for(int j=0; j<overflow/4; j++) {
        asm volatile ("p.spsw x0, 0(%[ovf]);"                
                    ::[ovf] "r" (overflow_buffer+j));
    }

    /* Then overflow stack_buffer with overflow_buffer */
    memcpy(stack_buffer, overflow_buffer, overflow); 
    
    return;
}

int main(){
    vuln_stack_return_addr();
    printf("Attack prevented.\n");
    return EXIT_SUCCESS;
}\end{lstlisting}

%%%%%%%%%%%%%%%%%%%%%%%%%%%%%%%%%%%%%%%%%%
\subsection{Second use case: Format String (WU-FTPd)}
The second use case is a format string attack\footnote{\url{https://github.com/sld-columbia/riscv-dift/tree/master/pulpino_apps_dift/wu-ftpd}} overwriting the return address of a function to jump to a shellcode and starts its execution.  This use case uses the first security policy from Table~\ref{tab:tpr} and Table~\ref{tab:tcr}.
This attack exploits the \verb|printf()| function from the C library. It uses the \verb|%u| and \verb|%n| formats (see Chapter 12, Section 12.14.3 in~\cite{gnu_lib_c} for detailed information) to write the targeted address.

Listing~\ref{code:wu_ftpd} shows the C code of this use case. The \texttt{echo} function assign the \textit{x8} register to a variable '\verb|i|' which is copied into another variable '\verb|a|'.
The lines 13-14 are used to initialise the tag associated to the variable '\verb|a|'. This variable `\verb|a|' is user-defined, so it is tagged as untrusted for DIFT computation.
The vulnerable statement is the \verb|printf| statement in line 16.
The format \verb|%u| is used to print unsigned integer characters.
The format \verb|%n| is used to store in memory the number of characters printed by the \verb|printf()| function, the argument it takes is a pointer to a signed int value. 

The execution of the \verb|printf| at line 16 leads to write in memory 224 (0xe0) at address (a-4), 224+35 so 259 (0x103) at address (a-3), and 512 (0x200) at addresses (a-2) and (a-1). The attacker's objective is to overwrite the return address with `\textit{0x3e0}' which represent the address of the first function, called \textit{secretFunction} in Listing~\ref{code:wu_ftpd}.
Table~\ref{table:ftpdOverwriteMemory} represents the different steps to overwrite the memory with the exact address of the malicious function. We can see that after each write and the right shift of the writing, the address appears. Finally, we have the address '\textit{000002000003E0}' in memory from 'A+2' to 'A-4' but as an address is on 32-bits in our architecture, the address fetched by the pipeline is only '\textit{000003E0}'.
In this use case, security policy prohibits the use of untrusted variables as store addresses. Since variable `\verb|a|' is untrusted, the DIFT protection raises an exception when storing a value at memory address \verb|(a-4)|. This use case has been chosen to activate the load/store modes of the DIFT policy. 

\begin{lstlisting}[style=topPosition, language=C,label=code:wu_ftpd, caption=WU-FTPd C code]
void secretFunction(){
    printf("Congratulations!\n");
    printf("You have entered in the secret function!\n");

    exit(0);
}

void echo(){
    int a;
    register int i asm("x8");
    a = i;

    asm volatile ("p.spsw x0, 0(%[a]);"                
                ::[a] "r" (&a));

    printf("%224u%n%35u%n%253u%n%n", 1, (int*) (a-4), 1, (int*) (a-3), 1, (int*) (a-2), (int*) (a-1));

    return;
}

int main(int argc, char* argv[]){ 
    volatile int a = 1;

    if(a)
        echo();
    else
        secretFunction();

    return 0;
}\end{lstlisting}

\begin{table}[t]
    \centering
    \footnotesize
    \caption{Memory overwrite}
    \label{table:ftpdOverwriteMemory}
    \begin{tabular}{l|ccccccc}
        \toprule
        Address & A-4 & A-3 & A-2 & A-1 & A & A+1 & A+2 \\ \midrule
        A-4     & \textit{0xE0} & \textit{0x00} & \textit{0x00} & \textit{0x00} & \textit{}     & \textit{}     & \textit{}     \\
        A-3     & \textit{}     & \textit{0x03} & \textit{0x01} & \textit{0x00} & \textit{0x00} & \textit{}     & \textit{}     \\
        A-2     & \textit{}     & \textit{}     & \textit{0x00} & \textit{0x02} & \textit{0x00} & \textit{0x00} & \textit{}     \\
        A-1     & \textit{}     & \textit{}     & \textit{}     & \textit{0x00} & \textit{0x02} & \textit{0x00} & \textit{0x00} \\ \midrule
        Memory & {0xE0}    & {0x03}    & 0x00          & 0x00          & 0x02          & 0x00          & 0x00          \\
        \bottomrule
    \end{tabular}
\end{table}

%%%%%%%%%%%%%%%%%%%%%%%%%%%%%%%%%%%%%%%%%%
\subsection{Summary}
To summarise, these three use cases allow stimulating each element of the DIFT mechanism. Consequently, they can be used to study the impact of FIA into this mechanism. The next section studies the behaviour and assesses the DIFT against FIA.

%%%%%%%%%%%%%%%%%%%%%%%%%%%%%%%%%%%%%%%%%%%%%%%%%%%%%%%%%%%%%%%%%%%%%%%%%%%%%%%%%%%%%%%%%%%%%%%
\section{Vulnerability assessment}
\label{section:vuln_assessment}
In order to analyse the behaviour of the processor at application runtime against Fault Injection Attacks, we have simulated some fault injections campaigns in which we inject fault inside the 55 registers associated to the DIFT, which correspond to 127 bits in total. For these campaigns, we use a tool, developed for this purpose. This tool is described in Chapter~\ref{chapter:fissa} and can generate the TCL code to automatise fault injections attacks campaigns at\textit{ Cycle Accurate and Bit Accurate} (CABA) level.
Table~\ref{tab:registersDIFT} shows the repartition of these registers in every pipeline stage of the RI5CY core and the number of associated bits. This work has been published in ACM Sensors S\&P~\cite{PLG-22-SensorsSP}.

\begin{table}[t]
    \centering
    \footnotesize
    \caption{Numbers of registers and quantity of bits represented}
    \label{tab:registersDIFT}
    \begin{tabular}{@{}ccc@{}}
        \toprule
        \textbf{HDL Module} & \textbf{Number of registers} & \textbf{Number of bits in registers} \\ \midrule
        Instruction Fetch Stage & 2 & 2 \\
        Instruction Decode Stage & 14 & 19 \\
        Register File Tag & 1 & 32 \\
        Execution Stage & 1 & 1 \\
        Control and Status Registers & 2 & 64 \\
        Load/Store Unit & 4 & 9 \\
        \midrule
        \midrule
        \textbf{Total} & \textbf{\textbf{24}} & \textbf{\textbf{127}} \\
        \bottomrule
    \end{tabular}
\end{table}

We evaluate the design by conducting fault injection campaigns. By analysing the results of these campaigns, we can determine which specific registers are vulnerable. This evaluation is performed for each individual use case previously presented, allowing for a more detailed analysis. It also helps us to understand how the error tag propagates through the system and is subsequently detected before triggering an exception.

%%%%%%%%%%%%%%%%%%%%%%%%%%%%%%%%%%%%%%%%%%
\subsection{Fault model for vulnerability assessment}
In this vulnerability assessment, we consider an attacker able to inject faults into DIFT-related registers leading to \textit{bit set}, \textit{bit reset}, and \textit{single bit-flip in one register at a given clock cycle}. 
As discussed in section~\ref{section:soa_fm}, these fault models are the main fault models used in FIA for the most precise methods, such as laser fault injection. There is also \textit{skip instruction} fault model which is often used but as we do not target the configuration of the DIFT, we do not attack instructions but only registers.
To bypass the DIFT mechanism, the main attacker's goal is to prevent an exception being raised. To reach this objective, any DIFT-related register maintaining tag value, driving the tag propagation or the tag update process or maintaining the security policy configuration can be targeted.


%%%%%%%%%%%%%%%%%%%%%%%%%%%%%%%%%%%%%%%%%%
\subsection{First use case: Buffer overflow}
Table~\ref{tab:end_sim_from_time_fault_register_bo} shows that 24 fault injections in five different DIFT-related registers can lead to a successful attack despite the DIFT mechanism (i.e., DIFT protection is bypassed). 
For example, it shows that a fault injection targeting the~\textit{pc\_if\_o\_tag} register can defeat the DIFT protection if a fault is injected at cycle 3431 using a bit-flip or a set to 0 fault type.
Furthermore, Table~\ref{tab:end_sim_from_time_fault_register_bo} shows that five different cycles can be targeted for the attack to succeed. In most cases, \textit{bit-flip} leads to a successful injection with 12 successes over 24. Faults in \textit{tpr\_q} and \textit{tcr\_q} are successful, since these registers maintain the propagation rules and the security policy configuration (see Table~\ref{tab:tpr} and Table~\ref{tab:tcr} for more details about each bit position). Both \textit{pc\_if\_o\_tag} and \textit{rf\_reg[1]} are also critical registers for this use case. Indeed, \textit{pc\_if\_o\_tag} allows the propagation of the PC tag while \textit{rf\_reg[1]} stores the tag of the return address register $ra$. It is worth noting that register \textit{memory\_set\_o\_tag} is not in the Figure~\ref{fig:study_buffer_overflow_tag_propagation} of tag propagation but is vulnerable and create a success for bypassing the DIFT in our tests in simulation.

\begin{figure}[ht]
    \centering
    \includegraphics[width=\textwidth]{c3_vulnerabilities_assessment/img/buffer_overflow/bufferOverflowAttack_short.pdf}
    \caption{Tag propagation in a buffer overflow attack}
    \label{fig:study_buffer_overflow_tag_propagation}
\end{figure}

\begin{table}[t]
    \centering
    \footnotesize
    \setlength{\tabcolsep}{2pt}
    \caption{Buffer overflow: success per register, fault type and simulation time}
    \label{tab:end_sim_from_time_fault_register_bo}
    \begin{tabular}{@{}lccccccccccccccc@{}}
        \toprule
         & \multicolumn{3}{c}{Cycle 3428} & \multicolumn{3}{c}{Cycle 3429} & \multicolumn{3}{c}{Cycle 3430} & \multicolumn{3}{c}{Cycle 3431} & \multicolumn{3}{c}{Cycle 3432} \\\cmidrule(lr){2-4}\cmidrule(lr){5-7}\cmidrule(lr){8-10}\cmidrule(lr){11-13}\cmidrule(lr){14-16}
         & set0 & set1 & bit-flip & set0 & set1 & bit-flip & set0 & set1 & bit-flip & set0 & set1 & bit-flip & set0 & set1 & bit-flip \\
        \midrule
        pc\_if\_o\_tag &  &  &  &  &  &  &  &  &  & \checkmark &  & \checkmark &  &  &  \\
        memory\_set\_o\_tag &  & \checkmark & \checkmark &  &  &  &  &  &  &  &  &  &  &  &  \\
        rf\_reg[1] &  &  &  &  &  &  & \checkmark &  & \checkmark &  &  &  &  &  &  \\
        tcr\_q & \checkmark &  &  & \checkmark &  &  & \checkmark &  &  & \checkmark &  &  & \checkmark &  &  \\
        \rowcolor{LightGray} tcr\_q[21] &&& \checkmark &&& \checkmark &&& \checkmark &&& \checkmark &&& \checkmark \\
        tpr\_q & \checkmark & \checkmark &  & \checkmark & \checkmark &  &  &  &  &  &  &  &  &  &  \\
        \rowcolor{LightGray} tpr\_q[12] &&& \checkmark &&& \checkmark &  &  &&&&&&&  \\
        \rowcolor{LightGray} tpr\_q[15] &&& \checkmark &&& \checkmark &  &  &&&&&&&  \\
        \bottomrule
    \end{tabular}
\end{table}

Based on these results, we can present an in-depth analysis of the simulation results leading to successful attacks. The aim is to understand why an attack succeeds. For that purpose, we study the propagation of the fault through both temporal and logical views. Most of the faults targeting both TPR and TCR registers are not detailed in this section. Indeed, these faults mainly target the DIFT configuration and not the tag propagation and tag-checking computations. Faults targeting these registers can be performed in any cycle prior to their use. 

Figure~\ref{fig:study_buffer_overflow_tag_propagation} presents the $ra$ register tag propagation in the context of the first use case for a non-faulty execution. It focuses on three clock cycles from the decoding of a \verb|jalr| instruction (i.e.,  returning from the called function) to the DIFT exception due to a security policy violation. 
In cycle 3430, this tag is extracted from the \textit{register file tag} (i.e., from \textit{rf\_reg[1]}). In cycle 3431, it is propagated to the \textit{pc\_if\_o\_tag} register. Then, in cycle 3432, it is propagated to the \textit{pc\_id\_o\_tag} register and the first shellcode instruction is decoded. Since $ra$ is tagged as untrusted and the security policy prohibits the use of tagged data in PC (\textit{Execute Check} bit = 1 in Table~\ref{tab:tcr}), an exception is raised during the tag check process, which is performed in parallel of the first shellcode instruction decoding.

Figure~\ref{fig:study_buffer_overflow_tag_propagation} illustrates the reason behind the sensitivity of registers \textit{rf\_reg[1]} and \textit{pc\_if\_o\_tag} at cycles 3430, 3431 and 3432 highlighted in  Table~\ref{tab:end_sim_from_time_fault_register_bo}. We can note that \textit{pc\_id\_o\_tag} register does not appear in Table~\ref{tab:end_sim_from_time_fault_register_bo} while Figure~\ref{fig:study_buffer_overflow_tag_propagation} shows its role during tag propagation. Actually, this register gets its value from \textit{pc\_if\_o\_tag}, so a fault injection in this register only delays the exception. 

To further study the propagation of the fault, Figure~\ref{fig:buffer_overflow_tag_propagation} illustrates the logical relations between the DIFT-related registers (yellow boxes) and control signals or processor registers (grey boxes) driving the illegal instruction exception signal (red box). This figure does not describe the actual hardware architecture, but highlights the logic path leading to an exception raise. An attacker performing fault injections would like to drive the exception signal to `0' to defeat the D-RI5CY DIFT solution. Figure~\ref{fig:buffer_overflow_tag_propagation} shows that a single fault could lead to a successful injection, since all logic paths are built with \textit{AND} gates. For instance, if register \textit{rf\_reg[1]} is set to 0, the tag will be propagated from \textit{gate 1} to \textit{gate 4}. Then, \textit{gate 5} inputs are \textit{tcr\_q[21]} (i.e., `1') and \textit{pc\_id\_o\_tag} (i.e., `0',  \textit{gate 4} output). Thus, \textit{gate 5} output is driven to `0', disabling the exception. 
From Figure~\ref{fig:buffer_overflow_tag_propagation}, three fault propagation paths can be identified: from \textit{gate 1} to \textit{gate 5} if the fault is injected into \textit{rf\_reg[1]}, from \textit{gate 4} to \textit{gate 5} if a fault is injected into \textit{pc\_if\_o\_tag} and through \textit{gate 5} if a fault is injected into either the \textit{tcr\_q} or \textit{pc\_id\_o\_tag}.
Analysis of Figure~\ref{fig:buffer_overflow_tag_propagation} strengthens the results presented in Table~\ref{tab:end_sim_from_time_fault_register_bo} where \textit{set to 0} and \textit{bit-flip} fault types lead to successful attacks. The root cause is that the propagation paths consist entirely of \textit{AND} gates. 

\begin{figure}[ht]
    \centering
    \includegraphics[width=\textwidth]{c3_vulnerabilities_assessment/img/buffer_overflow/arborescence_bufferOverflow.pdf}
    \caption{Logic description of the exception driving in a buffer overflow attack}
    \label{fig:buffer_overflow_tag_propagation}
\end{figure}

%%%%%%%%%%%%%%%%%%%%%%%%%%%%%%%%%%%%%%%%%%
\subsection{Second use case: Format string (WU-FTPd)}
Table~\ref{tab:end_sim_from_time_fault_register_mo}, in page~\pageref{tab:end_sim_from_time_fault_register_mo}, shows that 52 fault injections in 10 DIFT-related registers can lead to a successful attack. 
Furthermore, it shows that 8 different cycles can be targeted for the attack to succeed. 29 successes over 52 are obtained with the \textit{bit-flip} fault type. 
\textit{alu\_operand\_a\_ex\_o\_tag}, \textit{alu\_operand\_b\_ex\_o\_tag} and \textit{alu\_operator\_o\_mode} registers are critical during cycles 52477 and 52478 since they are used for tag propagation related to the C statement \verb|(a-4)|. \textit{alu\_operand\_a\_ex\_o\_tag} and \textit{alu\_operand\_b\_ex\_o\_tag} sequentially store the tag associated to `\verb|a|' while \textit{alu\_operator\_o\_mode} stores the propagation rule according to the TPR configuration (see Table~\ref{tab:tpr}). \textit{regfile\_alu\_waddr\_ex\_o\_tag} stores the destination register index in which the tag resulting from tag propagation should be written.
\textit{check\_s1\_o\_tag} maintains the TCR value from the decode stage to the execution stage, it is compared to the value of the operand tag for tag checking.
\textit{rf\_reg[15]} stores the tag associated with the `\verb|a|' variable.
\textit{store\_dest\_addr\_ex\_o\_tag} maintains the tag of the destination address during a store instruction in the execute stage. 
\textit{use\_store\_ops\_ex\_o} drives a multiplexer to propagate the value stored in \textit{store\_dest\_addr\_ex\_o\_tag} register to the tag checking module.
Finally, faults in \textit{tpr\_q} and \textit{tcr\_q} are successful, since these registers maintain the propagation rules and the security policy configuration. 
The last two registers, \textit{tpr\_q} and \textit{tcr\_q} are critical when we fault the bit 12 of TPR because the load/store mode which is set to \textit{10} but if we change it the propagation policy will change and then the tag will not be propagated as a mode set to \textit{11} will clear the tag. A bit-flip at bit 15 will impact the behaviour as it stores the load/store enable source tag. Finally, bit 20 of TCR store the load/store check destination address tag, which is used when the program wants to store at the address (a-4).

\begin{figure}[ht]
    \centering
    \includegraphics[width=\linewidth]{c3_vulnerabilities_assessment/img/wuftpd/full_ftpd_short.pdf}
    \caption{Tag propagation in a format string attack}
    \label{fig:study_mem_overwriting_tag_propagation}
 \end{figure}

Figure~\ref{fig:study_mem_overwriting_tag_propagation} details the tag propagation in the context of a format string attack case for a non-faulty execution and illustrates the reason behind the sensitivity of registers highlighted in Table~\ref{tab:end_sim_from_time_fault_register_mo}.
Figure~\ref{fig:study_mem_overwriting_tag_propagation} focuses on three clock cycles dedicated to the instruction \verb|sw a4,0(a5)| decoding and execution, which should lead to the storage of the value 224 at address (a-4). 
In cycles 52482 and 52483, \verb|sw a4,0(a5)| is decoded and the source operands tag are retrieved from the tag register file. Particularly, the store destination address is retrieved from \textit{rf\_reg[15]} and stored in register \textit{store\_dest\_addr\_ex\_o\_tag}. In cycle 52484, the destination address of the store operation is computed by the processor Arithmetic Logic Unit (ALU).
In parallel, \textit{alu\_operator\_o\_mode}, \textit{alu\_operand\_a\_ex\_o\_tag}, \textit{alu\_operand\_b\_ex\_o\_tag}, \textit{store\_dest\_addr\_ex\_o\_tag} and \textit{check\_s1\_o\_tag} registers drives the tag computation corresponding to the destination address. 
\textit{use\_store\_ops\_ex\_o} drives a multiplexer to propagate the value stored in \textit{alu\_operand\_a\_ex\_o\_tag} register to the tag checking module. 
\textit{alu\_operand\_a\_ex\_o\_tag} and \textit{alu\_operand\_b\_ex\_o\_tag} sequentially store the tag associated to `\verb|a|' while \textit{alu\_operator\_o\_mode} stores the propagation rule according to the TPR configuration (see Table~\ref{tab:tpr}).
\textit{check\_s1\_o\_tag} maintains the TCR value from the decode stage to the execution stage, it is compared to the value of the operand tag for tag checking.
Then, the store should be executed in the Execute stage. However, the tag associated with the store destination address is set to 1 due to tag propagation (since it is computed from variable `\verb|a|'). 
Since the security policy prohibits the use of data tagged as \textit{untrusted} as a store instruction destination address (\textit{Load/Store Check} field of TCR = 1010), an exception is raised.
\textit{use\_store\_ops\_ex\_o}, highlighted in Table~\ref{tab:end_sim_from_time_fault_register_mo} but not shown in Figure~\ref{fig:study_mem_overwriting_tag_propagation}, drives a multiplexer leading to the propagation of register \textit{store\_dest\_addr\_ex\_o\_tag}.

\begin{landscape}
    \begin{table}[t]
        \scriptsize
        \centering
        \caption{Format string attack: success per register, fault type and simulation time}
        \label{tab:end_sim_from_time_fault_register_mo}
        \setlength{\tabcolsep}{1pt}
        \begin{tabular}{@{}lcccccccccccccccccccccccc@{}}
            \toprule
            & \multicolumn{3}{c}{Cycle 52477} & \multicolumn{3}{c}{Cycle 52478} & \multicolumn{3}{c}{Cycle 52479} & \multicolumn{3}{c}{Cycle 52480} & \multicolumn{3}{c}{Cycle 52481} & \multicolumn{3}{c}{Cycle 52482} & \multicolumn{3}{c}{Cycle 52483} & \multicolumn{3}{c}{Cycle 52484} \\\cmidrule(lr){2-4}\cmidrule(lr){5-7}\cmidrule(lr){8-10}\cmidrule(lr){11-13}\cmidrule(lr){14-16}\cmidrule(lr){17-19}\cmidrule(lr){20-22}\cmidrule(lr){23-25}
            & set0 & set1 & bit-flip & set0 & set1 & bit-flip & set0 & set1 & bit-flip & set0 & set1 & bit-flip & set0 & set1 & bit-flip & set0 & set1 & bit-flip & set0 & set1 & bit-flip & set0 & set1 & bit-flip \\
            \midrule
            alu\_operand\_a\_ex\_o\_tag & \checkmark &  & \checkmark &  &  &  &  &  &  &  &  &  &  &  &  &  &  &  &  &  &  \\
            alu\_operand\_b\_ex\_o\_tag &&&& \checkmark &  & \checkmark &  &  &  &  &  &  &  &  &  &  &  &  &  &  &  &  &  &  \\
            alu\_operator\_o\_mode &\checkmark & \checkmark && \checkmark & \checkmark &  &  &  &  &  &  &  &  &  &  &  &  &  &  &  &  &  &  &  \\
            \rowcolor{LightGray} alu\_operator\_o\_mode[0] &&& \checkmark &&& \checkmark &  &  &  &  &  &&&&&&&&&&&&&  \\
            \rowcolor{LightGray} alu\_operator\_o\_mode[1] &&& \checkmark &&& \checkmark &  &  &  &  &  &&&&&&&&&&&&&  \\
            check\_s1\_o\_tag &&&&  &  &  &  &  &  &  &  &  &  &  &  &  &  &  &  &  &  & \checkmark &  & \checkmark \\
            regfile\_alu\_waddr\_ex\_o\_tag[1] &&&&  &  &  &  &  &  &  &  &  &  &  & \checkmark &  &  &  &  &  &  &  &  &  \\
            rf\_reg[15] &&&&  &  &  &  &  &  &  &  &  &  &  &  & \checkmark &  & \checkmark & \checkmark &  & \checkmark &  &  &  \\
            store\_dest\_addr\_ex\_o\_tag &&&&  &  &  &  &  &  &  &  &  &  &  &  &  &  &  &  &  &  & \checkmark &  & \checkmark \\
            tcr\_q & \checkmark &&& \checkmark &  &  & \checkmark &  &  & \checkmark &  &  & \checkmark &  &  & \checkmark &  &  & \checkmark &  &  &  &  &  \\
            \rowcolor{LightGray} tcr\_q[20] &&& \checkmark &&& \checkmark &&& \checkmark &&& \checkmark &&& \checkmark &&& \checkmark &&& \checkmark &&&  \\
            tpr\_q && \checkmark &&  & \checkmark &  &  & \checkmark &  &  & \checkmark &  &  & \checkmark &  &  &  &  &  &  &  &  &  &  \\
            \rowcolor{LightGray} tpr\_q[12] &&& \checkmark &&& \checkmark &&& \checkmark &&& \checkmark &&& \checkmark &  &  &&&&&&&  \\
            \rowcolor{LightGray} tpr\_q[15] &&& \checkmark &&& \checkmark &&& \checkmark &&& \checkmark &&& \checkmark &  &  &&&&&&&  \\
            use\_store\_ops\_ex\_o &&&&  &  &  &  &  &  &  &  &  &  &  &  &  &  &  &  &  &  & \checkmark &  & \checkmark \\
            \bottomrule
        \end{tabular}
    \end{table}
\end{landscape}

\begin{figure}[ht]
    \centering
    \includegraphics[width=\linewidth]{c3_vulnerabilities_assessment/img/wuftpd/arborescence_v3_wuftpd.pdf}
    \caption{Logic description of the exception driving in a format string attack}
    \label{fig:mem_overwriting_tag_propagation}
\end{figure}

To further study the propagation of the fault, Figure~\ref{fig:mem_overwriting_tag_propagation} illustrates the logical relations between the DIFT-related registers (yellow boxes) and control signals or processor registers (gray boxes) driving the illegal instruction exception signal (red box) for the second use case. 
Figure~\ref{fig:mem_overwriting_tag_propagation} shows that a single fault could lead to a successful injection, since all logic paths are built with \textit{AND} gates. For instance, if register \textit{rf\_reg[15]} is set to 0, this tag value will be propagated from \textit{gate 8} to \textit{gate 11} and to \textit{mux 12}. Then, since \textit{mux 12} output drives one \textit{gate 3} input, \textit{gate 3} output is driven to `0', the exception is disabled. 
From Figure~\ref{fig:mem_overwriting_tag_propagation}, seven fault propagation paths can be identified: 
from \textit{gate 1} to \textit{gate 3} if the fault is injected into \textit{tcr\_q[20]},
through \textit{gate 3} if a fault is injected into \textit{check\_s1\_o\_tag},
from \textit{gate 4} or \textit{gate 5} to \textit{gate 3} if a fault is injected into \textit{alu\_operand\_b\_ex\_o\_tag} or \textit{alu\_operand\_a\_ex\_o\_tag},
from \textit{mux 6} to \textit{gate 3} if a fault is injected into \textit{alu\_operator\_o\_mode},
from \textit{mux 7} to \textit{gate 3} if a fault is injected into \textit{regfile\_alu\_waddr\_ex\_o\_tag},
from \textit{gate 8} to \textit{gate 3} if a fault is injected in the tag register file (i.e., register \textit{rf\_reg[15]}) and
from \textit{mux 11} to \textit{gate 3} if a fault is injected in either \textit{store\_dest\_addr\_ex\_o\_tag} or \textit{use\_store\_ops\_ex\_o}.
Analysis of Figure~\ref{fig:mem_overwriting_tag_propagation} reinforces the results presented in Table~\ref{tab:end_sim_from_time_fault_register_mo} where \textit{set to 0} and \textit{bit-flip} fault types lead to successful attacks. As with the first use case, the main cause is that the propagation paths are fully made of \textit{AND} gates. As shown in Table~\ref{tab:end_sim_from_time_fault_register_mo} \textit{alu\_operator\_o\_mode} register is sensitive to \textit{set to 0} and \textit{set to 1} fault types. Indeed, this register determines the tag propagation according to TPR. The tag propagation is disabled when a TPR field is set to `00' and the output tag is set to 0 (i.e., trusted) when a TPR field is set to `11'.

%%%%%%%%%%%%%%%%%%%%%%%%%%%%%%%%%%%%%%%%%%
\subsection{Third use case: Compare/Compute}
Table~\ref{tab:end_sim_from_time_fault_register_secpoV3} shows that 19 fault injections in 6 DIFT-related registers can lead to a successful attack. Furthermore, it shows that 4 different cycles can be targeted for the attack to succeed. The highest success rate is obtained with the \textit{bit-flip} fault type, with 10 successes over 19. 
Faults in \textit{rf\_reg[14]} and \textit{alu\_operand\_a\_ex\_o\_tag} are successful, since these registers store the tag associated to variable \verb|a| during tag propagation. \textit{check\_s1\_o\_tag} maintains one configuration bit from \textit{tcr\_q} during tag checking.
\textit{use\_store\_ops\_ex\_o} drives a multiplexer to propagate the value stored in \textit{alu\_operand\_a\_ex\_o\_tag} register to the tag checking module. 
For this case, the critical registers can be found in previous case, \textit{alu\_operand\_a\_ex\_o\_tag} propagate the tag of the tagged variable in the code (variable \verb|a|). 
Finally, observations for both \textit{tpr\_q} and \textit{tcr\_q} are similar than for previous case studies. 
Finally, faults in \textit{tpr\_q} and \textit{tcr\_q} are successful, since these registers maintain the propagation rules and the security policy configuration.

\begin{figure}[ht]
    \centering
    \includegraphics[width=\linewidth]{c3_vulnerabilities_assessment/img/comp_compu/attaquePropag_v3_short.pdf}
    \caption{Tag propagation in a computation case with the compare/compute use case}
    \label{fig:study_attack_propag_v3_tag_propagation}
\end{figure}

\begin{figure}[ht]
    \centering
    \includegraphics[width=\textwidth]{c3_vulnerabilities_assessment/img/comp_compu/arborescence_propagation.pdf}
    \caption{Logic representation of tag propagation in a computation case}
    \label{fig:attack_propag_v3_tag_propagation}
\end{figure}

\begin{table}[t]
   \small
   \centering
   \caption{Compare/compute: number of faults per register, per fault type and per cycle}
   \label{tab:end_sim_from_time_fault_register_secpoV3}
   \setlength{\tabcolsep}{3pt}
   \begin{tabular}{@{}lcccccccccccc@{}}
        \toprule
        & \multicolumn{3}{c}{Cycle 832} & \multicolumn{3}{c}{Cycle 833} & \multicolumn{3}{c}{Cycle 834} & \multicolumn{3}{c}{Cycle 835} \\\cmidrule(lr){2-4}\cmidrule(lr){5-7}\cmidrule(lr){8-10}\cmidrule(lr){11-13}
        & set0 & set1 & bit-flip & set0 & set1 & bit-flip & set0 & set1 & bit-flip & set0 & set1 & bit-flip \\
        \midrule
        alu\_operand\_a\_ex\_o\_tag &  &  &  &  &  &  &  &  &  & \checkmark &  & \checkmark \\
        check\_s1\_o\_tag &  &  &  &  &  &  &  &  &  & \checkmark &  & \checkmark \\
        rf\_reg[14] &  &  &  & \checkmark &  & \checkmark & \checkmark &  & \checkmark &  &  &  \\
        tcr\_q & \checkmark &  &  & \checkmark &  &  & \checkmark &  &  &  &  &  \\
        \rowcolor{LightGray} tcr\_q[0] &&& \checkmark &&& \checkmark &&& \checkmark &&&  \\
        tpr\_q &  & \checkmark &  &  &  &  &  &  &  &  &  &  \\
        \rowcolor{LightGray} tpr\_q[12] &&& \checkmark &  &  &&&&&&&  \\
        \rowcolor{LightGray} tpr\_q[15] &&& \checkmark &  &  &&&&&&&  \\
        use\_store\_ops\_ex\_o &  &  &  &  &  &  &  &  &  &  & \checkmark & \checkmark \\
        \bottomrule
    \end{tabular}
\end{table}

Figure~\ref{fig:study_attack_propag_v3_tag_propagation} focuses on the three cycles, represented in red, corresponding to \verb|add a5,a4,a5| instruction (C statement \verb|(a+b)|) decoding and execution in the context of the third use case. 
The instruction \verb|add a5,a4,a5| is in decode stage during cycles 833 and 834 and the tag associated to the untrusted variable \verb|a| is retrieved from \textit{rf\_reg[14]}. In cycle 835, this addition is executed. In parallel, variable \verb|a| tag is propagated to the tag check logic unit, which behaviour is driven by \textit{check\_s1\_o\_tag} through \textit{alu\_operand\_a\_ex\_o\_tag}. Since the V2 security policy prohibits the use of untrusted data as a source operand of an arithmetic operation, an exception is raised.

Figure~\ref{fig:study_attack_propag_v3_tag_propagation} illustrates the reason behind the sensitivity of registers \textit{rf\_reg[14]}, \textit{alu\_operand\_a\_ex\_o\_tag} and \textit{check\_s1\_o\_tag} highlighted in Table~\ref{tab:end_sim_from_time_fault_register_secpoV3}.
Note that \textit{use\_store\_ops\_ex\_o} does not appear in Figure~\ref{fig:study_attack_propag_v3_tag_propagation}. This register drives a multiplexer leading to tag propagation presented in Figure~\ref{fig:study_attack_propag_v3_tag_propagation}.

To further study the faults' propagation, Figure~\ref{fig:attack_propag_v3_tag_propagation} illustrates the logical relations between the DIFT-related registers (yellow boxes) and control signals or processor registers (gray boxes) driving the illegal instruction exception signal (red box).
Figure~\ref{fig:attack_propag_v3_tag_propagation} shows that a single fault could lead to a successful injection, since all logic paths are built with \textit{AND} gates. For instance, if register \textit{rf\_reg[14]} is set to 0, the tag will be propagated from \textit{gate 8} to \textit{gate 10} and to \textit{mux 12}. Then, since \textit{mux 12} output drives one \textit{gate 3} output, the exception is disabled.
From Figure~\ref{fig:attack_propag_v3_tag_propagation}, seven fault propagation paths can be identified. We won't go into detail here about the seven different paths, as they were mentioned in case 2, bearing in mind that colour differentiation must be taken into account (for example: \textit{alu\_operand\_a\_ex\_o\_tag} instead of \textit{store\_dest\_addr\_ex\_o\_tag}
from \textit{gate 1} to \textit{gate 3} if the fault is injected into \textit{tcr\_q[0]},
through \textit{gate 3} if a fault is injected into \textit{check\_s1\_o\_tag},
from \textit{gate 4} or \textit{gate 5} to \textit{gate 3} if a fault is injected into \textit{alu\_operand\_b\_ex\_o\_tag} or \textit{alu\_operand\_a\_ex\_o\_tag},
from \textit{mux 6} to \textit{gate 3} if a fault is injected into \textit{alu\_operator\_o\_mode},
from \textit{mux 7} to \textit{gate 3} if a fault is injected into \textit{regfile\_alu\_waddr\_ex\_o\_tag}, from \textit{gate 8} to \textit{gate 3} if a fault is injected into \textit{rf\_reg[14]}, and
from \textit{mux 11} to \textit{gate 3} if a fault is injected into either \textit{alu\_operand\_a\_ex\_o\_tag} or \textit{use\_store\_ops\_ex\_o}.
Analysis of Figure~\ref{fig:attack_propag_v3_tag_propagation} supports the results presented in Table~\ref{tab:end_sim_from_time_fault_register_secpoV3} where \textit{set to 0} and \textit{bit-flip} fault types lead to successful attacks. As with first and second use cases, the main reason is that the propagation paths are built entirely from \textit{AND} gates.

%%%%%%%%%%%%%%%%%%%%%%%%%%%%%%%%%%%%%%%%%%%%%%%%%%%%%%%%%%%%%%%%%%%%%%%%%%%%%%%%%%%%%%%%%%%%%%%
\begin{table}[t]
    \centering
    \footnotesize
    \caption{Results for \textit{bit reset} for the baseline version}
    \label{table:end_sim_by_status_wop_1_set0}
    \begin{tabular}{@{}lcccccc@{}}
        \toprule
                        & Crash & Silent & Delay & Success     & Total & Execution time \\
        \midrule
        Buffer Overflow & 0     & 320    & 1     & 9 (2.73\%)  & 330   & 0:04           \\
        WU-FTPd         & 0     & 424    & 0     & 16 (3.64\%) & 440   & 0:47           \\
        Compare/Compute & 0     & 213    & 0     & 7 (3.18\%)  & 220   & 0:01           \\
        \bottomrule
    \end{tabular}
\end{table}

\begin{table}[t]
    \centering
    \footnotesize
    \caption{Results for \textit{bit set} for the baseline version}
    \label{table:end_sim_by_status_wop_1_set1}
    \begin{tabular}{@{}lcccccc@{}}
        \toprule
                        & Crash & Silent & Delay & Success    & Total & Execution time \\
        \midrule
        Buffer Overflow & 0     & 320    & 7     & 3 (0.91\%) & 330   & 0:04           \\
        WU-FTPd         & 0     & 397    & 36    & 7 (1.59\%) & 440   & 0:48           \\
        Compare/Compute & 0     & 213    & 5     & 2 (0.91\%) & 220   & 0:01           \\
        \bottomrule
    \end{tabular}
\end{table}

\begin{table}[t]
    \centering
    \footnotesize
    \caption{Results for a \textit{single bit-flip} for the baseline version}
    \label{table:end_sim_by_status_wop_1_bitflip}
    \begin{tabular}{@{}lcccccc@{}}
        \toprule
                        & Crash & Silent & Delay & Success     & Total & Execution time \\
        \midrule
        Buffer Overflow & 0     & 738    & 12    & 12 (1.57\%) & 762   & 0:11           \\
        WU-FTPd         & 0     & 946    & 41    & 29 (2.85\%) & 1016  & 01:52          \\
        Compare/Compute & 0     & 491    & 7     & 10 (1.97\%) & 508   & 0:02           \\
        \bottomrule
    \end{tabular}
\end{table}

%%%%%%%%%%%%%%%%%%%%%%%%%%%%%%%%%%%%%%%%%%%%%%%%%%%%%%%%%%%%%%%%%%%%%%%%%%%%%%%%%%%%%%%%%%%%%%%
\section{Summary}
In this chapter, we described the processor we focus on, with its implementation of a hardware in-core DIFT. We described how it works and how to use the DIFT mechanism with the default configuration. Then, we described the different use cases we choose to work with, in order to analyse the DIFT behaviour and assess it against fault injection attacks. Finally, we presented the vulnerability assessment on these use cases using the D-RI5CY security mechanism. We have shown that this DIFT implementation is vulnerable to FIA within different registers depending on the fault model and depending on the application, as different paths are used and so different registers are going to be critical.

Tables~\ref{table:end_sim_by_status_wop_1_set0}, \ref{table:end_sim_by_status_wop_1_set1}, \ref{table:end_sim_by_status_wop_1_bitflip} present the results obtained from the campaign with their respective fault model.
This vulnerability analysis revealed that the majority of weaknesses in this mechanism are caused by single bit-flips, with 51 successful faults out of 95. Furthermore, the registers involved in this mechanism are predominantly 1-bit registers, as they are used for the tag data path. This indicates that, to effectively safeguard the mechanism, the primary focus should be on protecting it against single bit-flip errors.

%%%%%%%%%%%%%%%%%%%%%%%%%%%%%%%%%%%%%%%%%%%%%%%%%%%%%%%%%%%%%%%%%%%%%%%%%%%%%%%%%%%%%%%%%%%%%%%

\clearemptydoublepage
\chapter{FISSA -- Fault Injection Simulation for Security Assessment}
\chaptermark{FISSA - Fault Injection Simulation for Security Assessment}
\label{chapter:fissa}
\minitoc

This section introduces and presents a tool, called FISSA -- Fault Injection Simulation for Security Assessment --, created to automate fault injection attacks campaigns in simulation. The first section presents the state of the art of existing tools for FIA campaigns in emulation, formal methods or even perform real world attacks. The second section presents the architecture and details how FISSA works and presents how to extend it depending on other needs. Finally, we will discuss and draw some perspectives for the tool's development and usability.

%%%%%%%%%%%%%%%%%%%%%%%%%%%%%%%%%%%%%%%%%%%%%%%%%%%%%%%%%%%%%%%%%%%%%%%%%%%%%%%%%%%%%%%%%%%%%%%
\section{Simulation tools for Fault Injection}
This section presents recent works related to methods and tools for vulnerability assessment when considering fault injection attacks. For such vulnerability assessment, main strategies include actual fault injections, emulations, formal methods and simulations.

\begin{table}[t]
    \centering
    \caption{Fault Injection based methods for vulnerability assessment comparison}
    \label{table:FI_type_comparison}
    \normalsize
    \setlength{\tabcolsep}{1pt}
    \begin{tabular}{@{}lccccccc@{}}
        \toprule
                          & References  & Cost                                & \begin{tabular}[c]{@{}c@{}}Control over\\fault scenarios\end{tabular}       & Scalability                         & Speed of execution                                      & Realism                             & Expertise \\ \midrule
        Formal Methods    & \cite{BSSMG-21-tchess, ANR-18-ices, BBCFGS-19-esorics, SVPMRDKMS-24-eprint}     & \textcolor{ForestGreen}{Very low}  & \textcolor{ForestGreen}{Very high}  & \textcolor{red}{Very low}           & \textcolor{red}{Low}                                    & \textcolor{red}{Low}                & \textcolor{red}{Very high} \\
        Simulations  & \cite{AB-23-acns, fisim, AWMN-20-host}     & \textcolor{ForestGreen}{Very low}       & \textcolor{ForestGreen}{Very high}  & \textcolor{red}{Low}                & \textcolor{red}{Low}/\textcolor{ForestGreen}{Moderate}  & \textcolor{ForestGreen}{Moderate}   & \textcolor{ForestGreen}{Low} \\
        Emulations        & \cite{CMLCVR-11-crypto, HGASO-21-fdtc,BLK-23-access, NNHRS-14-dsd}     & \textcolor{Red}{High}           & \textcolor{ForestGreen}{Moderate}   & \textcolor{ForestGreen}{High}  & \textcolor{ForestGreen}{Very high}                      & \textcolor{ForestGreen}{High}       & \textcolor{red}{Moderate} \\
        Actual FIA        & \cite{BCNTW-06-procieee, BFP-19-tches, GBD-23-paine, CGVCBLC-22-cardis}     & \textcolor{Red}{Very high}           & \textcolor{Red}{Very low}           & \textcolor{ForestGreen}{Very high}  & \textcolor{ForestGreen}{Very high}                      & \textcolor{ForestGreen}{Very high}  & \textcolor{red}{Very high} \\
        \bottomrule
    \end{tabular}
\end{table}

Actual FIAs involve physically injecting faults into the target hardware using techniques such as variations in supply voltage or clock signal~\cite{BCNTW-06-procieee, BFP-19-tches}, laser pulses~\cite{BCNTW-06-procieee, CGVCBLC-22-cardis}, electromagnetic emanations~\cite{BCNTW-06-procieee} or X-Rays~\cite{GBD-23-paine}.
This approach offers valuable insights into the real impact of faults on hardware components.
However, a significant drawback of actual fault injections is that they demand considerable expertise to prepare the target, involving intricate setup procedures.
Additionally, this approach can only be executed once the physical circuit is available, potentially delaying the vulnerability assessment process until later stages of development.


Fault emulation can, for instance, rely on FPGA~\cite{CMLCVR-11-crypto}, or on an emulator such as QEMU~\cite{HGASO-21-fdtc,BLK-23-access} to perform fault injection campaigns. This approach is four times faster than simulation-based techniques~\cite{NNHRS-14-dsd}, and unlike simulation-based or formal method-based fault injections techniques, the size of the evaluated circuit has no major impact on the fault injection campaign timing performances.
However, configuring an emulation environment can be complex and time-consuming. Achieving an accurate representation of the target system may require detailed configuration and parameter tuning. The accuracy of emulation is contingent on the quality of the models used to replicate the target hardware. If the models are inaccurate or incomplete, the results of fault injections may not precisely reflect actual behaviour.

Formal methods provide an advantage with mathematical proofs, ensuring a rigorous verification of the system's behaviour during fault injection experiments. Formal methods approaches such as~\cite{BSSMG-21-tchess} allow the analysis of a circuit design in order to detect sensitive logic or sequential hardware elements. \cite{ANR-18-ices}, \cite{BBCFGS-19-esorics} and~\cite{SVPMRDKMS-24-eprint} present formal verification methods to analyse the behaviour of HDL implementation.
However, this type of tool usually suffers from restrictions limiting its actual usage on a complete processor.
Conventional formal approaches encounter scalability challenges due to limitations in verification techniques.
In particular, the circuit structure it can analyse is usually limited.

Fault Injections simulations can be performed at processor instructions level. Authors of~\cite{AB-23-acns} explore the impact of fault injection attacks on software security. They evaluate four open-source fault simulators, comparing their techniques and suggest enhancing them with AI methods inspired by advances in cryptographic fault simulation. \cite{fisim} is an open-source deterministic fault attack simulator prototype utilising the Unicorn Framework and Capstone disassembler.
\cite{AWMN-20-host} introduces VerFI, a gate-level granularity fault simulator for hardware implementations. For instance, it has been used to spot an implementation mistake in ParTI~\cite{SMG-16-crypto}.
However, this tool has been developed to check if implemented countermeasures can really protect against fault injection on cryptographic implementations, but it cannot evaluate components such as registers or memories.
In this paper, we focus on Cycle Accurate Bit Accurate (CABA) Simulation, which provides a controlled virtual environment for injecting faults. There are several solutions of simulations in an HDL simulator like Questasim, Vivado, etc. \textit{Behavioural} simulation is used to detect functional issues and ensuring that the design behaves as expected. \textit{Post-synthesis} simulation verifies that the synthesised netlist matches the expected functionality. \textit{Timed} simulation is used to ensure that the design meets timing requirements and can operate at the specified clock frequency. And finally, \textit{post-implementation} simulations are used to verify that the implemented design meets all requirements and constraints, including those related to the physical layout on the target.
Simulation-based fault injection offers the advantage of enabling designers to test their system throughout the design cycle, providing valuable insights and uncovering potential vulnerabilities early in the development process. However, a limitation lies in the potential lack of absolute fidelity to actual conditions, as simulations might not perfectly replicate all hardware intricacies, introducing a slight risk of overlooking certain faults that could manifest in the actual hardware.

Table~\ref{table:FI_type_comparison} shows a comparison between these four methods for vulnerability assessment when considering FIA regarding six metrics. These metrics are the financial cost of setting up the fault injection campaign, the control over fault scenarios (how configurable are the scenarios), scalability which refers to the method capacity to be applied to systems of different sizes or complexities, speed of execution of the campaign, realism of the fault injection campaign and the level of required expertise.
Table~\ref{table:FI_type_comparison} shows that no method is completely optimal. Each method has its own advantages and disadvantages and must be chosen by the designer according to the requirements and the available financial and human resources. Indeed, setting up an actual fault injection campaign requires much more expertise in this domain and also requires costly equipment, whereas setting up a simulation campaign can be easier for a circuit designer familiar with HDL simulation tools such as Questasim.
Table~\ref{table:FI_type_comparison} shows that CABA simulation offers a good compromise to assess the security level of a circuit design. In particular, it provides an efficient solution for investigating security throughout the design cycle, enabling the concept of “Security by Design”.

%%%%%%%%%%%%%%%%%%%%%%%%%%%%%%%%%%%%%%%%%%%%%%%%%%%%%%%%%%%%%%%%%%%%%%%%%%%%%%%%%%%%%%%%%%%%%%%
\section{FISSA}
This section presents our open-source tool, FISSA~\cite{fissa}. It is under the CeCILL-B licence and available on GitHub.

\subsection{Main software architecture}
FISSA is designed to help circuit designers to analyse, throughout the design cycle, the sensitivity to FIA of the developed circuit.
Figure~\ref{fig:archi_fissa} presents the software architecture of FISSA.
It consists of 3 different modules: \textit{TCL generator}, \textit{Fault Injection Simulator} and \textit{Analyser}. The first and third modules correspond to a set of Python classes.
\textit{The TCL generator}, detailed in Section~\ref{subsec:tcl_gen}, relies on a configuration file and a target file to create a set of parameterised TCL scripts. These scripts are tailored based on the provided configuration file and are used to drive the fault injection simulation campaign.

\textit{Fault Injection Simulator}, detailed in Section~\ref{subsec:FIS}, performs the fault injection simulation campaign based on inputs files from \textit{TCL generator} for a circuit design described through HDL files and memory initialisation files. For that purpose it relies on an existing HDL simulator such as Questasim~\cite{questasim}, Verilator~\cite{verilator}, or Vivado~\cite{vivado}.

\textit{The Analyser}, detailed in Section~\ref{subsec:analyser}, evaluates the outcomes of the simulations and generates a set of files that allows the designers to examine fault injection effects on their designs through various information.


\begin{figure}[ht]
    \centering
    \includegraphics[width=\textwidth]{c4_fissa/img/fissa/archi_gen_tcl.pdf}
    \caption{Software architecture of FISSA}
    \label{fig:archi_fissa}
\end{figure}

Algorithm~\ref{algo:pseudoCodeSimuStages} shows a representation of a fault injection campaign. The algorithm requires a set of targets  (i.e. hardware elements in which a fault should be
injected), the fault model and the considered injection window(s) which identifies the period(s), in number of clock cycles, in which fault injections are performed.
Then, it runs a first simulation with no fault injected, which is used as a reference for comparison with the following simulations to determine end-of-simulation statuses. 
Then, for each target, each fault model and for each clock cycle within the injection window, the corresponding simulation is executed, and the corresponding logs are stored in a dedicated file.

Customising end-of-simulation statuses allows for adaptation to the specific requirements of each design assessment. To configure these statuses, adjustments need to be made either directly in FISSA's code or the HDL code. This process may involve evaluating factors such as:
\begin{itemize}
    \item hardware element content (signal, registers, \ldots),
    \item simulation time (e.g. the simulation exceeds a reference number of clock cycles),
    \item simulation's end (e.g. an assert statement introduced in the HDL code is reached)
\end{itemize}

\begin{algorithm}
    \caption{Simulated FIA campaign pseudo-code}
    \label{algo:pseudoCodeSimuStages}
    \normalsize
    \begin{algorithmic}[1]
        \Require $targets \leftarrow list(targets)$
        \Require $faults \leftarrow list(fault_model)$
        \Require $windows \leftarrow list(injection_windows)$
            \State $ref_sims = simulate()$
            \For{$target \in targets$}
                \For{$fault \in faults$}
                    \For{$cycle \in windows$}
                        \State $logs = simulate(target, fault, cycle)$
                    \EndFor
                \EndFor
            \EndFor
    \end{algorithmic}
\end{algorithm}

\subsection{Supported fault models}
\label{subsec:supported_fault_models}

A set of fault models has already been integrated into FISSA. For a given fault injection campaign, the relevant fault model is defined in the input configuration file and is applied to targets during the simulation phase.
Currently, supported fault models are:
\begin{itemize}
    \justifying
    \item target set to 0/1,
    \item single bit-flip in one target at a given clock cycle,
    \item single bit-flip in two targets at a given clock cycle,
    \item single bit-flip in two targets at two different clock cycles,
    \item exhaustive multi-bits faults in one target at a given clock cycle,
    \item exhaustive multi-bits faults in two targets at a given clock cycle.
\end{itemize}

\subsection{TCL Generator}
\label{subsec:tcl_gen}

The \textit{TCL Generator} is used to generate the set of TCL script files which drive the \textit{fault injection simulator}. This module requires two input files.
Figure~\ref{fig:archi-scriptGen} details the \textit{TCL Generator}. Each blue box represents a python class used to generate the set of output TCL scripts.
The initialisation class gets inputs from a configuration file. This JSON-formatted file includes various parameters such as the targeted HDL simulator, the considered fault model and the injection window(s). Furthermore, it encompasses parameters such as the clock period (in ns) of the HDL design and the maximum number of simulated clock cycles used to stop the simulation in case of divergence due to the injected fault. Moreover, one extra parameter defines the quantity of simulations per TCL file, allowing a simulation parallelism degree.
The \textit{Targets} file contains, in YAML format, the list of the circuit elements (e.g. registers or logic gates) that need to be targeted during the fault injection campaign. For each target, its HDL path and bit-width are specified.
\textit{TCL Script Generator} class gets the configuration parameters from \textit{Initialisation} class, reads the \textit{Targets'} file and calls three others classes.
The first one, \textit{Basic Code Generator}, undertakes the fundamental generation of TCL code for initialising a simulation, running a simulation, and ending a simulation.
The second one, \textit{Fault Generator}, produces the TCL code related to fault injection. The \textit{TCL Script Generator} provides specific parameters to the \textit{Fault Generator} to produce code for a designated set of targets and a specified set of clock cycles for fault injection.
The third one, \textit{Log Generator}, produces the TCL code to produce logs after each simulation.
Logs comprise the simulation's ID, fault model, faulted targets, injection clock cycle(s), end-of-simulation status, values for all targets, and the end-of-simulation clock cycle. This data constitutes the automated aspect of logging.
Finally, the \textit{TCL Script Generator} outputs a set of TCL files, each one correspond to a batch of simulations. It is worth noting that each batch starts with a reference simulation (i.e. without fault injection). This allows the user to perform a per batch results analysis.
Furthermore, it produces a target file used by TCL scripts to get the target list (see Subsection~\ref{subsec:FIS}).

\begin{figure}[ht]
    \centering
    \includegraphics[width=\textwidth]{c4_fissa/img/fissa/detail_tcl_gen.pdf}
    \caption{Software architecture of the TCL Generator module}
    \label{fig:archi_tcl_gen}
\end{figure}

Algorithm~\ref{algo:pseudoCodeSimus} depicts a fault injection simulation pseudo-code, showcasing requirements, each state with essential parameters, and the corresponding Python class from Figure~\ref{fig:archi_tcl_gen}.
Line 5 in Algorithm~\ref{algo:pseudoCodeSimuStages} corresponds to Algorithm~\ref{algo:pseudoCodeSimus}. This algorithm is executed multiple times with different inputs to build a TCL script.


\begin{algorithm}
    \caption{FIA simulation pseudo-code}
    \label{algo:pseudoCodeSimus}
    \normalsize
    \begin{algorithmic}[1]
        \Require $target$
        \Require $cycle$
        \Require $fault_model$
        \State $tcl_script = init_sim(fault_model, cycle, target)$ \textcolor{blue}{\scriptsize // generated by Basic Code Generator}
        \State $tcl_script += inject_fault(fault_model)$  \textcolor{red}{\scriptsize // generated by Fault Generator}
        \State $tcl_script += run_sim()$ \textcolor{blue}{\scriptsize // generated by Basic Code Generator}
        \State $tcl_script += log_sim(fault_model)$ \textcolor{ForestGreen}{\scriptsize // generated by Log Generator}
        \State $tcl_script += end_sim()$ \textcolor{blue}{\scriptsize // generated by Basic Code Generator}
        \State $tcl_file.write(tcl_script)$
    \end{algorithmic}
\end{algorithm}

\subsection{Fault Injection Simulator}
\label{subsec:FIS}

\subsection{Analyser}
\label{subsec:analyser}

\subsection{Extending FISSA}

%%%%%%%%%%%%%%%%%%%%%%%%%%%%%%%%%%%%%%%%%%%%%%%%%%%%%%%%%%%%%%%%%%%%%%%%%%%%%%%%%%%%%%%%%%%%%%%
\section{Discussion and Perspectives}

%%%%%%%%%%%%%%%%%%%%%%%%%%%%%%%%%%%%%%%%%%%%%%%%%%%%%%%%%%%%%%%%%%%%%%%%%%%%%%%%%%%%%%%%%%%%%%%
\section{Summary}

%%%%%%%%%%%%%%%%%%%%%%%%%%%%%%%%%%%%%%%%%%%%%%%%%%%%%%%%%%%%%%%%%%%%%%%%%%%%%%%%%%%%%%%%%%%%%%%

\clearemptydoublepage
\chapter{Countermeasures Implementations}
\chaptermark{Countermeasures Implementations}
\label{chapter:countermeasures}
\minitoc

%%%%%%%%%%%%%%%%%%%%%%%%%%%%%%%%%%%%%%%%%%%%%%%%%%%%%%%%%%%%%%%%%%%%%%%%%%%%%%%%%%%%%%%%%%%%%%%
\section{Introduction}
Previous chapters have shown that the D-RI5CY's DIFT security mechanism is vulnerable to FIAs, mainly due to single-bit flips. This D-RI5CY essentially uses single-bit registers, as its data path is a single bit.

In this chapter, we present two countermeasures in order to protect the DIFT against fault injection attacks, and bit-flip.
The first countermeasure implemented to detect and prevent the use of corrupted data is simple parity. We selected the simple parity code as the error detection countermeasure because of its suitability and limited overhead.
The second countermeasure is implemented to detect any single-bit errors that may occur, but also to correct them without time overhead. With this countermeasure, we want to correct to the nearest cycle so that the fault cannot propagate and give a potential attacker the impression that the fault he injected had no effect on the system.
This work has been published in ISVLSI 2024~\cite{PRLG-24-isvlsi}.

The first section of this chapter presents the different fault models considered. Then, the second section details the implementation of simple parity and briefly presents how it works. Afterwards, the third section presents the working of Hamming code, with a simple example, and details our implementation. Finally, we discuss these countermeasures and compare them.

%%%%%%%%%%%%%%%%%%%%%%%%%%%%%%%%%%%%%%%%%%%%%%%%%%%%%%%%%%%%%%%%%%%%%%%%%%%%%%%%%%%%%%%%%%%%%%%
\section{Fault models used in this chapter}
In Chapter~\ref{chapter:dift_assessment}, we assessed the design by considering \textit{single bit-flip in one register at a given clock cycle}, \textit{set to 0}, and \textit{set to 1} fault models. The conclusion of this chapter was that the D-RI5CY is vulnerable to single bit-flip.

In this chapter, we consider an attacker able to inject faults into DIFT-related registers, leading to single bit-flips at any position of the targeted register. To reach this objective, any DIFT-related register maintaining 1-bit tag value, driving the tag propagation or the tag update process or maintaining the security policy configuration can be targeted. Studies presented in~\cite{ZDCRT-12-dcis,CLFT-14-cosade} have shown that such precise single bit-flip attacks targeting registers can be performed using, for example, laser shots. We also consider an attacker able to inject a single bit-flip in two registers at two distinct clock cycles, with a minimum delay of one clock cycle.

%%%%%%%%%%%%%%%%%%%%%%%%%%%%%%%%%%%%%%%%%%%%%%%%%%%%%%%%%%%%%%%%%%%%%%%%%%%%%%%%%%%%%%%%%%%%%%%
\section{Countermeasure 1: Simple Parity}
\label{chapter:simpleparity}

Error detection is often achieved through the use of parity codes, which involve adding an extra bit to the data bits for redundancy. Simple parity codes can detect single-bit errors. Simple parity can take into account odd parity or even parity, which means in case of an even parity, the number of bit set to '\texttt{1}' will be even with the message bits and parity bit.

%%%%%%%%%%%%%%%%%%%%%%%%%%%%%%
\subsection{Presentation of the simple parity}
Simple parity involves adding one bit to the data. This one bit store the parity of the initial message. Figures~\ref{fig:simpleparity_functionning} shows how to data and the parity bit are combined. The data, in blue, and the parity bit, in red, are associated to form an encoded data.

\begin{figure}[ht]
    \centering
    \includegraphics[page=1]{c5_countermeasures_dift/img/simple_parity.pdf}
    \caption{Simple Parity - functioning}
    \label{fig:simpleparity_functionning}
\end{figure}

Equation~\ref{equat:simpleparity} shows how the parity bit is computed. Each bit of the initial message is xor'd to show parity. Table~\ref{tab:xor_truthtable} present the truth table of XOR operations.

\begin{equation} \label{equat:simpleparity}
    \begin{split}
        p_{0} &= d_{0} \oplus d_{1} \oplus d_{2} \oplus d_{3} \oplus d_{4} \oplus d_{5} \oplus d_{6}
    \end{split}
\end{equation}

\begin{table}[t]
    \centering
    \caption{XOR truth table}
    \label{tab:xor_truthtable}
    \begin{tabular}{@{}c|c|c@{}}
        \toprule
        A & B & $A \oplus B$ \\\midrule
        0 & 0 & 0            \\
        0 & 1 & 1            \\
        1 & 0 & 1            \\
        1 & 1 & 0            \\
        \bottomrule
    \end{tabular}
\end{table}

Figures~\ref{fig:simpleparity_example_1} and \ref{fig:simpleparity_example_2} show an example of a message with its parity bit associated. The message is \texttt{0b1001101} in binary, then, as there is an even number of '\texttt{1}', the parity bit is set to '\texttt{0}' (cf Table~\ref{tab:xor_truthtable}).
\begin{figure}[ht]
    \centering
    \begin{subfigure}[b]{0.49\textwidth}
        \includegraphics[width=\textwidth, page=2]{c5_countermeasures_dift/img/simple_parity.pdf}
        \caption{Initial message}
        \label{fig:simpleparity_example_1}
    \end{subfigure}
    \hfill
    \begin{subfigure}[b]{0.49\textwidth}
        \includegraphics[width=\textwidth, page=3]{c5_countermeasures_dift/img/simple_parity.pdf}
        \caption{Message with its parity bit}
        \label{fig:simpleparity_example_2}
    \end{subfigure}
    \hfill
    \begin{subfigure}[b]{0.49\textwidth}
        \includegraphics[width=\textwidth, page=4]{c5_countermeasures_dift/img/simple_parity.pdf}
        \caption{Single-bit fault inside the message}
        \label{fig:simpleparity_faulted_example_3}
    \end{subfigure}
    \hfill
    \begin{subfigure}[b]{0.49\textwidth}
        \includegraphics[width=\textwidth, page=5]{c5_countermeasures_dift/img/simple_parity.pdf}
        \caption{Two single-bit faults inside the message}
        \label{fig:simpleparity_faulted_example_4}
    \end{subfigure}
    \caption{Exemple of a simple parity computation}
    \label{fig:simpleparity_example}
\end{figure}

Figures~\ref{fig:simpleparity_faulted_example_3} and \ref{fig:simpleparity_faulted_example_4} present, respectively, two examples of when a fault occur or when two faults happen. In the first example, Figure~\ref{fig:simpleparity_faulted_example_3}, the bit $d_2$ is faulted.
As the faulted message is \texttt{0b1001001}, it means that the parity bit will change from \texttt{0} to \texttt{1}. Hence, the fault will be detected as the parity bit differs from original computed message (Figure~\ref{fig:simpleparity_example_2}).

In the second case, two faults happen in the message at bit $d_2$ and $d_5$. Hence, the faulted message will be \texttt{0b1101001}, so when the parity is computed, the parity bit will not change as there is still an even number of \texttt{1} compared to the initial message.

%%%%%%%%%%%%%%%%%%%%%%%%%%%%%%
\subsection{Implementation: Optimisation of redundancy bits}

\begin{table}[t]
    \centering
    \caption{DIFT-related protected registers - simple parity}
    \label{tab:sp_group}
    \begin{tabular}{@{}cccc@{}}
        \toprule
                & Protected register                                                                                & Number of protected bits & \begin{tabular}[c]{@{}c@{}}Number of parity\\ bits for Simple Parity\end{tabular} \\ \midrule
        Group 1 & TCR                                                                                               & 22                       & 1                                                                                 \\
        Group 2 & TPR                                                                                               & 22                       & 1                                                                                 \\
        Group 3 & Register File (Tag)                                                                               & 32                       & 1                                                                                 \\
        Group 4 & Tag destination address                                                                           & 5                        & 1                                                                                 \\
        Group 5 & \begin{tabular}[c]{@{}c@{}}16×1-bit registers\\ 3×2-bit registers\\ 1×4-bit register\end{tabular} & 26                       & 1                                                                                 \\ \midrule
        Total   &                                                                                                   & 107                      & 5                                                                                 \\
        \bottomrule
    \end{tabular}
\end{table}

\begin{figure}[ht]
    \centering
    \includegraphics[page=1]{c5_countermeasures_dift/img/archi_contremesures.pdf}
    \caption{Implementation of simple parity}
    \label{fig:implementation_sp}
\end{figure}


%%%%%%%%%%%%%%%%%%%%%%%%%%%%%%%%%%%%%%%%%%%%%%%%%%%%%%%%%%%%%%%%%%%%%%%%%%%%%%%%%%%%%%%%%%%%%%%
\section{Countermeasure 2: Hamming Code}
\label{chapter:hammingcode}

%%%%%%%%%%%%%%%%%%%%%%%%%%%%%%
\subsection{Presentation of Hamming Code}

\begin{figure}[ht]
    \centering
    \includegraphics[page=1]{c5_countermeasures_dift/img/hamming_bit.pdf}
    \caption{Hamming code - functioning}
    \label{fig:hamming_functionning}
\end{figure}

\begin{equation} \label{equat:hamming_encoder}
    \begin{split}
        r_{0} &= d_{0} \oplus d_{1} \oplus d_{3} \oplus d_{4} \oplus d_{6} \\
        r_{1} &= d_{0} \oplus d_{2} \oplus d_{3} \oplus d_{5} \oplus d_{6} \\
        r_{2} &= d_{1} \oplus d_{2} \oplus d_{3} \\
        r_{3} &= d_{4} \oplus d_{5} \oplus d_{6}
    \end{split}
\end{equation}

%%%%%%%%%%%%%%%%%%%%%%%%%%%%%%
\subsection{Implementation 1: Optimisation of redundancy bits}

\begin{table}[t]
    \centering
    \caption{DIFT-related protected registers - Hamming code}
    \label{tab:hammingcode_group}
    \begin{tabular}{@{}cccc@{}}
        \toprule
                & Protected register                                                                                & Number of protected bits & \begin{tabular}[c]{@{}c@{}}Number of redundancy\\ bits for Hamming Code\end{tabular} \\ \midrule
        Group 1 & TCR                                                                                               & 22                       & 5                                                                                    \\
        Group 2 & TPR                                                                                               & 22                       & 5                                                                                    \\
        Group 3 & Register File (Tag)                                                                               & 32                       & 6                                                                                    \\
        Group 4 & Tag destination address                                                                           & 5                        & 4                                                                                    \\
        Group 5 & \begin{tabular}[c]{@{}c@{}}16×1-bit registers\\ 3×2-bit registers\\ 1×4-bit register\end{tabular} & 26                       & 5                                                                                    \\ \midrule
        Total   &                                                                                                   & 107                      & 25                                                                                   \\
        \bottomrule
    \end{tabular}
\end{table}

\begin{figure}[ht]
    \centering
    \includegraphics[page=2, width=\textwidth]{c5_countermeasures_dift/img/archi_contremesures.pdf}
    \caption{Implementation of Hamming Code}
    \label{fig:implementation_hc_1}
\end{figure}

\begin{figure}[ht]
    \centering
    \includegraphics[page=3, width=\textwidth]{c5_countermeasures_dift/img/archi_contremesures.pdf}
    \caption{Implementation of Hamming Code - Register Tile Tag}
    \label{fig:implementation_hc_2}
\end{figure}

%%%%%%%%%%%%%%%%%%%%%%%%%%%%%%%%%%%%%%%%%%%%%%%%%%%%%%%%%%%%%%%%%%%%%%%%%%%%%%%%%%%%%%%%%%%%%%%
\section{Discussion}

%%%%%%%%%%%%%%%%%%%%%%%%%%%%%%%%%%%%%%%%%%%%%%%%%%%%%%%%%%%%%%%%%%%%%%%%%%%%%%%%%%%%%%%%%%%%%%%
\section{Summary}

%%%%%%%%%%%%%%%%%%%%%%%%%%%%%%%%%%%%%%%%%%%%%%%%%%%%%%%%%%%%%%%%%%%%%%%%%%%%%%%%%%%%%%%%%%%%%%%

\clearemptydoublepage
\chapter{Implementation strategies: evaluation and results}
\chaptermark{Implementation strategies: evaluation and results}
\label{chapter:exp_setup_results}
\minitoc

%%%%%%%%%%%%%%%%%%%%%%%%%%%%%%%%%%%%%%%%%%%%%%%%%%%%%%%%%%%%%%%%%%%%%%%%%%%%%%%%%%%%%%%%%%%%%%%
\section{Introduction}
The previous chapter presented two countermeasures against fault injection attacks taking into account simple fault models, such as \textit{single-bit flip inside one register at a given clock cycle}. These countermeasures have been implemented by grouping the different DIFT-related registers in order to minimise the number of parity and redundancy bits. However, some studies~\cite{CGVCBLC-22-cardis,VDSPB-24-jce} have shown that is it possible to fault multiple bits precisely.

In this chapter, we present four different implementation's strategies of countermeasures to better protect the D-RI5CY mechanism against more complex fault models. Then, we evaluate each of these strategies in terms of security against more complex fault models. Finally, we compare them in terms of performance and area overhead. We implemented the minimisation of redundancy bits strategy in the last chapter. As shown in Chapter~\ref{chapter:countermeasures}, Hamming Code or even SECDED is better to use than just the simple parity for the correction and detection capacity. Hence, in this chapter, we do not implement others strategies for the simple parity protection. However, we present the results obtained from our simulations campaigns on the fault models considered in this chapter.

Section~\ref{section:chap6_faultmodels} introduces the different fault models considered.
Section~\ref{section:chap6_implem_strategies} introduces four different strategies developed and assessed in this chapter. Some tables are presented in Appendix~\ref{app:strat_details} due to their size.
Section~\ref{section:chap6_evaluation} presents the security assessment of these four strategies by giving the results associated to each fault model and use cases, and evaluate them in terms of security, performance, and area overhead.
Finally, in Section~\ref{section:chap6_discussion}, we discuss the results obtained from these four strategies and with the implementation strategy of Chapter~\ref{chapter:countermeasures} according to their performance and area overhead and give the limitations for each strategy.

%%%%%%%%%%%%%%%%%%%%%%%%%%%%%%%%%%%%%%%%%%%%%%%%%%%%%%%%%%%%%%%%%%%%%%%%%%%%%%%%%%%%%%%%%%%%%%%
\section{Fault models considered in this chapter}
\label{section:chap6_faultmodels}
In Chapter~\ref{chapter:countermeasures}, we presented the results of fault injection campaigns targeting \textit{a single bit-flip in one register at a given clock cycle}, and \textit{a single bit-flip in two registers at two distinct clock cycles}. We demonstrated that lightweight countermeasures, such as simple parity, Hamming Code, or SECDED version of Hamming Code, are effective in protecting our DIFT mechanism against single bit-flips occurring in one register at one clock cycle or in two registers at two distinct clock cycles.

In this chapter, we extend our analysis to consider an attacker capable of injecting faults into DIFT-related registers, leading to a \textit{single bit-flip in two registers at a given clock cycle}. Furthermore, we account for an attacker able to induce \textit{multi-bit faults in one register at a given clock cycle}, as well as, \textit{multi-bit faults in two registers at a given clock cycle}. These fault models, introduced in Chapter~\ref{chapter:fissa}, are exhaustively tested across registers ranging from 1-bit to 10-bit. Registers larger than 10 bits, such as the configuration registers TPR and TCR, are out of the question due to their size. Indeed, simulating an exhaustive attack on a single 32-bit register for one cycle would require $2^{32}$ simulations (i.e: \powerTwo{pow(32,2)}{0} simulations), and for the combination of two 32-bit registers, the number of simulations would reach $2^{32} \times 2^{32}$ which is too large to be simulated in a reasonable time. However, it is worth noting that the biggest register after these two 32-bit registers is a 6-bit register (cf. Table~\ref{tab:strategies_register_info}), so we fault every 1-bit to 6-bit registers.

The three fault models are exhaustively simulated across all possible values of these registers. To meet this objective, any DIFT-related register that maintains a 1-bit tag value, drives tag propagation or tag update processes, or holds security policy configurations, can be targeted. Additionally, registers storing redundancy bits for protection mechanisms are also considered.

% %%%%%%%%%%%%%%%%%%%%%%%%%%%%%%%%%%%%%%%%%%%%%%%%%%%%%%%%%%%%%%%%%%%%%%%%%%%%%%%%%%%%%%%%%%%%%%%
\section{Implementation strategies}
\label{section:chap6_implem_strategies}

Assessing the robustness of DIFT against more complex fault models requires comprehensive strategies that can identify vulnerabilities to enhance the system integrity. This section introduces four distinct strategies aimed at evaluating and enhancing the security of DIFT mechanisms against complex fault models. Each strategy offers a unique perspective on detecting, mitigating, or preventing the effects of multi bit-flip faults, contributing to a holistic approach in fortifying DIFT systems. By exploring these methodologies, we aim to provide actionable insights for developing more resilient DIFT solutions thanks to lightweight countermeasures.

\subsection{Strategy 2: Pipeline Stage Register Coupling for Robust Error Mitigation}

\begin{table}[t]
    \centering
    \scriptsize
    \caption{D-RI5CY Registers Details List for Strategy 2}
    \label{tab:strategy_2_register_info}
    % \setlength{\tabcolsep}{5pt}
    \begin{tabular}{@{}rccc@{}}
        \toprule
        Register Name                   & Module                                & Size   & \tableTwoLines{Strategy}{2} \\\midrule
        pc\_id\_o\_tag                  & \textcolor{red}{Instruction}          & 1      & Gr1                         \\
        pc\_if\_o\_tag                  & \textcolor{red}{Fetch Stage}          & 1      & Gr1                         \\\hdashline
        alu\_operand\_a\_ex\_o\_tag     &                                       & 1      & Gr2                         \\
        alu\_operand\_b\_ex\_o\_tag     &                                       & 1      & Gr2                         \\
        alu\_operand\_c\_ex\_o\_tag     &                                       & 1      & Gr2                         \\
        alu\_operator\_o\_mode          &                                       & 2      & Gr2                         \\
        check\_d\_o\_tag                &                                       & 1      & Gr2                         \\
        check\_s1\_o\_tag               &                                       & 1      & Gr2                         \\
        check\_s2\_o\_tag               & \textcolor{blue}{Instruction}         & 1      & Gr2                         \\
        is\_store\_post\_o\_tag         & \textcolor{blue}{Decode Stage}        & 1      & Gr2                         \\
        memory\_set\_o\_tag             &                                       & 1      & Gr2                         \\
        regfile\_alu\_waddr\_ex\_o\_tag &                                       & 5      & Gr2                         \\
        register\_set\_o\_tag           &                                       & 1      & Gr2                         \\
        store\_dest\_addr\_ex\_o\_tag   &                                       & 1      & Gr2                         \\
        store\_source\_ex\_o\_tag       &                                       & 1      & Gr2                         \\
        use\_store\_ops\_ex\_o          &                                       & 1      & Gr2                         \\\hdashline
        rf\_reg[0]                      &                                       & 1      & Gr3                         \\
        rf\_reg[1]                      &                                       & 1      & Gr3                         \\
        rf\_reg[2]                      & \textcolor{LimeGreen}{Register File}  & 1      & Gr3                         \\
        \ldots                          & \textcolor{LimeGreen}{Tag}            & \ldots & Gr3                         \\
        rf\_reg[30]                     &                                       & 1      & Gr3                         \\
        rf\_reg[31]                     &                                       & 1      & Gr3                         \\\hdashline
        rs1\_o\_tag                     & \textcolor{DarkOrange}{Execute Stage} & 1      & Gr4                         \\\hdashline
        tcr\_q                          & \textcolor{DarkRed}{Control and}      & 32     & Gr5                         \\
        tpr\_q                          & \textcolor{DarkRed}{Status Registers} & 32     & Gr6                         \\\hdashline
        data\_type\_q\_tag              &                                       & 2      & Gr7                         \\
        data\_we\_q\_tag                & \textcolor{magenta}{Load/Store}       & 1      & Gr7                         \\
        rdata\_offset\_q\_tag           & \textcolor{magenta}{Unit}             & 2      & Gr7                         \\
        rdata\_q\_tag                   &                                       & 4      & Gr7                         \\
        \bottomrule
    \end{tabular}
\end{table}

In the second implemented strategy, we rely on protecting each pipeline stage of our processor individually to minimise the impact on performances. To achieve this implementation, we decided to form seven groups: Instruction Fetch (IF) Stage, Instruction Decode (ID) Stage, Register File Tag, Execute (EX) Stage, two groups for the two registers TPR and TCR containing the security policy, and a last group with the Load/Store Unit. 

Table~\ref{tab:strategy_2_register_info} represents the different DIFT-related registers with their associated group.
Table~\ref{tab:strategy_2_groups} represents the number of protected bits inside each pipeline stage and their associated number of redundancy, and parity bits, when SECDED is used. As depicted in this table, the number of protected bits differs a lot depending on the pipeline stage, ranging from one bit to thirty-two bits.
Otherwise, the HDL implementations are the same as Chapter~\ref{chapter:countermeasures} with two proposed implementations (see Figure~\ref{fig:implementation_sd_1} and Figure~\ref{fig:implementation_sd_2}). This strategy protects 107 bits by adding 30 redundancy bits and 7 parity bits, which led to approximatively a 30\% increase in number of bits stored into registers.

\subsection{Strategy 3: Individual Register Encapsulation for Robust Error Mitigation}

\begin{table}[t]
    \centering
    \scriptsize
    \caption{D-RI5CY Registers Details List for Strategy 3}
    \label{tab:strategy_3_register_info}
    % \setlength{\tabcolsep}{5pt}
    \begin{tabular}{@{}rccc@{}}
        \toprule
        Register Name                   & Module                                & Size   & \tableTwoLines{Strategy}{3} \\\midrule
        pc\_if\_o\_tag                  & \textcolor{red}{Fetch Stage}          & 1      & Gr1                         \\
        pc\_id\_o\_tag                  & \textcolor{red}{Instruction}          & 1      & Gr2                         \\\hdashline
        alu\_operand\_a\_ex\_o\_tag     &                                       & 1      & Gr3                         \\
        alu\_operand\_b\_ex\_o\_tag     &                                       & 1      & Gr4                         \\
        alu\_operand\_c\_ex\_o\_tag     &                                       & 1      & Gr5                         \\
        alu\_operator\_o\_mode          &                                       & 2      & Gr6                         \\
        check\_d\_o\_tag                &                                       & 1      & Gr7                         \\
        check\_s1\_o\_tag               &                                       & 1      & Gr8                         \\
        check\_s2\_o\_tag               & \textcolor{blue}{Instruction}         & 1      & Gr9                         \\
        is\_store\_post\_o\_tag         & \textcolor{blue}{Decode Stage}        & 1      & Gr10                        \\
        memory\_set\_o\_tag             &                                       & 1      & Gr11                        \\
        regfile\_alu\_waddr\_ex\_o\_tag &                                       & 5      & Gr12                        \\
        register\_set\_o\_tag           &                                       & 1      & Gr13                        \\
        store\_dest\_addr\_ex\_o\_tag   &                                       & 1      & Gr14                        \\
        store\_source\_ex\_o\_tag       &                                       & 1      & Gr15                        \\
        use\_store\_ops\_ex\_o          &                                       & 1      & Gr16                        \\\hdashline
        rf\_reg[0]                      &                                       & 1      & Gr17                        \\
        rf\_reg[1]                      &                                       & 1      & Gr17                        \\
        rf\_reg[2]                      & \textcolor{LimeGreen}{Register File}  & 1      & Gr17                        \\
        \ldots                          & \textcolor{LimeGreen}{Tag}            & \ldots & Gr17                        \\
        rf\_reg[30]                     &                                       & 1      & Gr17                        \\
        rf\_reg[31]                     &                                       & 1      & Gr17                        \\\hdashline
        rs1\_o\_tag                     & \textcolor{DarkOrange}{Execute Stage} & 1      & Gr18                        \\\hdashline
        tcr\_q                          & \textcolor{DarkRed}{Control and}      & 32     & Gr19                        \\
        tpr\_q                          & \textcolor{DarkRed}{Status Registers} & 32     & Gr20                        \\\hdashline
        data\_type\_q\_tag              &                                       & 2      & Gr21                        \\
        data\_we\_q\_tag                & \textcolor{magenta}{Load/Store}       & 1      & Gr22                        \\
        rdata\_offset\_q\_tag           & \textcolor{magenta}{Unit}             & 2      & Gr23                        \\
        rdata\_q\_tag                   &                                       & 4      & Gr24                        \\
        \bottomrule
    \end{tabular}
\end{table}

In the third implementation strategy, we aim to enhance protection for every register associated to the DIFT within our processor, except the registers inside the Register File Tag. To achieve this, we created 24 groups for all the registers, ensuring a more targeted and effective protection mechanism. Specifically, two groups were formed in the IF stage, addressing the initial handling of PC addresses. A significant portion, fourteen groups, was allocated to the ID stage, as this stage contains processing and handling of tags information. Additionally, one group was dedicated to the Register File Tag, as we consider this Register File as one register even if it is 32 registers to avoid an increase overhead for the Register File. For the EX stage, we formed a single group. Furthermore, two separated groups were created for the TPR and TCR registers, recognising their distinct control functions. Finally, four groups were designated for the Load/Store Unit, as it can be considered as the fourth stage of our processor. This structure allows for a granular protection approach, ensuring that each aspect of the processor's DIFT-related registers is securely managed. The issue with this strategy is the use of two redundancy bits and one parity bit to protect one-bit registers.

Table~\ref{tab:strategy_3_register_info} represents the group composition with the different DIFT-related registers.
Table~\ref{tab:strategy_3_groups} represents the number of protected bits inside each protected group and their associated number of redundancy and parity bits, when SECDED is used. As depicted in this table, there is mainly only one bit protected in the majority of groups (16 groups over 24). This strategy protects 107 bits by adding 64 redundancy bits and 24 parity bits, which led to approximatively a 70\% increase in number of bits stored into registers.

\subsection{Strategy 4: DIFT-Enhanced CSR Register Splitting for Strengthened Security}

\begin{table}[t]
    \centering
    \scriptsize
    \caption{D-RI5CY Registers Details List for Strategy 4}
    \label{tab:strategy_4_register_info}
    % \setlength{\tabcolsep}{5pt}
    \begin{tabular}{@{}rccc@{}}
        \toprule
        Register Name                   & Module                                & Size   & \tableTwoLines{Strategy}{3} \\\midrule
        pc\_if\_o\_tag                  & \textcolor{red}{Fetch Stage}          & 1      & Gr1                         \\
        pc\_id\_o\_tag                  & \textcolor{red}{Instruction}          & 1      & Gr2                         \\\hdashline
        alu\_operand\_a\_ex\_o\_tag     &                                       & 1      & Gr3                         \\
        alu\_operand\_b\_ex\_o\_tag     &                                       & 1      & Gr4                         \\
        alu\_operand\_c\_ex\_o\_tag     &                                       & 1      & Gr5                         \\
        alu\_operator\_o\_mode          &                                       & 2      & Gr6                         \\
        check\_d\_o\_tag                &                                       & 1      & Gr7                         \\
        check\_s1\_o\_tag               &                                       & 1      & Gr8                         \\
        check\_s2\_o\_tag               & \textcolor{blue}{Instruction}         & 1      & Gr9                         \\
        is\_store\_post\_o\_tag         & \textcolor{blue}{Decode Stage}        & 1      & Gr10                        \\
        memory\_set\_o\_tag             &                                       & 1      & Gr11                        \\
        regfile\_alu\_waddr\_ex\_o\_tag &                                       & 5      & Gr12                        \\
        register\_set\_o\_tag           &                                       & 1      & Gr13                        \\
        store\_dest\_addr\_ex\_o\_tag   &                                       & 1      & Gr14                        \\
        store\_source\_ex\_o\_tag       &                                       & 1      & Gr15                        \\
        use\_store\_ops\_ex\_o          &                                       & 1      & Gr16                        \\\hdashline
        rf\_reg[0]                      &                                       & 1      & Gr17                        \\
        rf\_reg[1]                      &                                       & 1      & Gr17                        \\
        rf\_reg[2]                      & \textcolor{LimeGreen}{Register File}  & 1      & Gr17                        \\
        \ldots                          & \textcolor{LimeGreen}{Tag}            & \ldots & Gr17                        \\
        rf\_reg[30]                     &                                       & 1      & Gr17                        \\
        rf\_reg[31]                     &                                       & 1      & Gr17                        \\\hdashline
        rs1\_o\_tag                     & \textcolor{DarkOrange}{Execute Stage} & 1      & Gr18                        \\\hdashline
        tpr\_q                          & \textcolor{DarkRed}{Control and}      & 32     & Gr19 -- Gr26                 \\
        tcr\_q                          & \textcolor{DarkRed}{Status Registers} & 32     & Gr27 -- Gr34                 \\\hdashline
        data\_type\_q\_tag              &                                       & 2      & Gr35                        \\
        data\_we\_q\_tag                & \textcolor{magenta}{Load/Store}       & 1      & Gr36                        \\
        rdata\_offset\_q\_tag           & \textcolor{magenta}{Unit}             & 2      & Gr37                        \\
        rdata\_q\_tag                   &                                       & 4      & Gr38                        \\
        \bottomrule
    \end{tabular}
\end{table}

In the fourth implementation strategy, we keep the protection on each register individually as implementation strategy 3. However, we improve the protection on the two CSRs registers. Our idea is to split these two registers by group of operations (arithmetic, branching, etc. -- see Table~\ref{tab:tpr} and table~\ref{tab:tcr} for more details). In this way, we aim to enhance the detection of errors occurring in the security policy related registers.

Table~\ref{tab:strategy_4_register_info} shows the group affectation for each register. As TPR and TCR are split, they take eight groups each.
Table~\ref{tab:strategy_4_groups} depicts the number of redundancy and parity bits for each group. As the different operations of TPR and TCR are on one to four bits, the number of redundancy bits vary from two to three. This strategy protects 103 bits by adding 101 redundancy bits and 38 parity bits, which led to approximatively a 109\% increase in number of bits stored into registers.

\subsection{Strategy 5: Sliced Register Bit Coupling for Improved Data Integrity}

\begin{figure}[ht]
    \centering
    \includegraphics[page=1]{c6_group_composition/img/implem5_spaghetti.pdf}
    \caption{Strategy 5 -- Mixing Registers Implementation}
    \label{fig:strategy_5_functionning}
\end{figure}

\begin{table}[t]
    \centering
    \scriptsize
    \caption{D-RI5CY Registers Details List for Strategy 5}
    \label{tab:strategy_5_register_info}
    \begin{tabular}{@{}rccc@{}}
        \toprule
        Register Name                   & Module                                & Size   & \tableTwoLines{Strategy}{3} \\\midrule
        pc\_if\_o\_tag                  & \textcolor{red}{Fetch Stage}          & 1      & Gr1                         \\
        pc\_id\_o\_tag                  & \textcolor{red}{Instruction}          & 1      & Gr1                         \\\hdashline
        alu\_operand\_a\_ex\_o\_tag     &                                       & 1      & Gr4                         \\
        alu\_operand\_b\_ex\_o\_tag     &                                       & 1      & Gr5                         \\
        alu\_operand\_c\_ex\_o\_tag     &                                       & 1      & Gr6                         \\
        alu\_operator\_o\_mode          &                                       & 2      & Gr2 -- Gr3                   \\
        check\_d\_o\_tag                &                                       & 1      & Gr9                         \\
        check\_s1\_o\_tag               &                                       & 1      & Gr7                         \\
        check\_s2\_o\_tag               & \textcolor{blue}{Instruction}         & 1      & Gr8                         \\
        is\_store\_post\_o\_tag         & \textcolor{blue}{Decode Stage}        & 1      & Gr10                        \\
        memory\_set\_o\_tag             &                                       & 1      & Gr11                        \\
        regfile\_alu\_waddr\_ex\_o\_tag &                                       & 5      & Gr5 -- Gr9                   \\
        register\_set\_o\_tag           &                                       & 1      & Gr10                        \\
        store\_dest\_addr\_ex\_o\_tag   &                                       & 1      & Gr2                         \\
        store\_source\_ex\_o\_tag       &                                       & 1      & Gr3                         \\
        use\_store\_ops\_ex\_o          &                                       & 1      & Gr4                         \\\hdashline
        rf\_reg[0]                      &                                       & 1      & Gr12                        \\
        rf\_reg[1]                      &                                       & 1      & Gr12                        \\
        rf\_reg[2]                      & \textcolor{LimeGreen}{Register File}  & 1      & Gr12                        \\
        \ldots                          & \textcolor{LimeGreen}{Tag}            & \ldots & Gr12                        \\
        rf\_reg[30]                     &                                       & 1      & Gr12                        \\
        rf\_reg[31]                     &                                       & 1      & Gr12                        \\\hdashline
        rs1\_o\_tag                     & \textcolor{DarkOrange}{Execute Stage} & 1      & Gr35                        \\\hdashline
        tpr\_q                          & \textcolor{DarkRed}{Control and}      & 32     & Gr13 -- Gr26 / Gr28 -- Gr30   \\
        tcr\_q                          & \textcolor{DarkRed}{Status Registers} & 32     & Gr13 -- Gr34                 \\\hdashline
        data\_type\_q\_tag              &                                       & 2      & Gr36 -- Gr37                 \\
        data\_we\_q\_tag                & \textcolor{magenta}{Load/Store}       & 1      & Gr39                        \\
        rdata\_offset\_q\_tag           & \textcolor{magenta}{Unit}             & 2      & Gr37 -- Gr38                 \\
        rdata\_q\_tag                   &                                       & 4      & Gr35 -- Gr36 / Gr38 -- Gr39   \\
        \bottomrule
    \end{tabular}
\end{table}

In the fifth strategy, we propose a less straightforward idea. Instead of protecting registers individually or by pipeline stage, we could protect them by mixing them. By mixing the registers, an attacker would require to attack precisely one bit of two different registers or more. Figure~\ref{fig:strategy_5_functionning} presents this strategy with four registers: one 4-bit register (i.e $R_0$), one 2-bit register (i.e $R_1$) and two 1-bit registers (i.e. $R_2$ and $R_3$). Then we take the larger register, and we decompose each bit into one Hamming Code or SECDED encoder, and we add one bit of another register to this encoder. Each of these encoder take maximum as inputs two bits. If possible, we try to never mix the same registers together. In the following, the encoder and decoder computes in the same manner as other strategies. This strategy is more complex to implement, as it requires separating each registers into different encoders. In our strategy, we have 39 encoders.

Table~\ref{tab:strategy_5_register_info} shows the group affectation for each register. For example, register \texttt{pc\_if\_o\_tag} is in group 1 only because it is a 1-bit register, while \texttt{regfile\_alu\_waddr\_ex\_o\_tag} is present in group 5 to group 9 because it is a 5-bit register. For the TPR, we encode only the used bits: 0 to 13 and 15 to 17.
Table~\ref{tab:strategy_5_groups} presents the number of redundancy and parity bits for each group. This strategy protects 102 bits by adding 114 redundancy bits and 39 parity bits, which led to a 120\% increase in number of bits stored into registers.

%%%%%%%%%%%%%%%%%%%%%%%%%%%%%%%%%%%%%%%%%%%%%%%%%%%%%%%%%%%%%%%%%%%%%%%%%%%%%%%%%%%%%%%%%%%%%%%
\section{Experimental results}
\label{section:chap6_evaluation}

\begin{table}[t]
    \centering
    \scriptsize
    \caption{D-RI5CY Registers Details List}
    \label{tab:strategies_register_info}
    \setlength{\tabcolsep}{4pt}
    \begin{tabular}{@{}rccccccc@{}}
        \toprule
        Register Name                   & Module                                & Size   & \tableTwoLines{Strategy}{1} & \tableTwoLines{Strategy}{2} & \tableTwoLines{Strategy}{3} & \tableTwoLines{Strategy}{4} & \tableTwoLines{Strategy}{5}                \\
        \midrule
        pc\_id\_o\_tag                  & \textcolor{red}{Instruction}          & 1      & Gr5                         & Gr1                         & Gr1                         & Gr1                         & Gr1                                        \\
        pc\_if\_o\_tag                  & \textcolor{red}{Fetch Stage}          & 1      & Gr5                         & Gr1                         & Gr2                         & Gr2                         & Gr1                                        \\\hdashline
        alu\_operand\_a\_ex\_o\_tag     &                                       & 1      & Gr5                         & Gr2                         & Gr3                         & Gr3                         & Gr4                                        \\
        alu\_operand\_b\_ex\_o\_tag     &                                       & 1      & Gr5                         & Gr2                         & Gr4                         & Gr4                         & Gr5                                        \\
        alu\_operand\_c\_ex\_o\_tag     &                                       & 1      & Gr5                         & Gr2                         & Gr5                         & Gr5                         & Gr6                                        \\
        alu\_operator\_o\_mode          &                                       & 2      & Gr5                         & Gr2                         & Gr6                         & Gr6                         & Gr2 - Gr3                                  \\
        check\_d\_o\_tag                &                                       & 1      & Gr5                         & Gr2                         & Gr7                         & Gr7                         & Gr9                                        \\
        check\_s1\_o\_tag               &                                       & 1      & Gr5                         & Gr2                         & Gr8                         & Gr8                         & Gr7                                        \\
        check\_s2\_o\_tag               & \textcolor{blue}{Instruction}         & 1      & Gr5                         & Gr2                         & Gr9                         & Gr9                         & Gr8                                        \\
        is\_store\_post\_o\_tag         & \textcolor{blue}{Decode Stage}        & 1      & Gr5                         & Gr2                         & Gr10                        & Gr10                        & Gr10                                       \\
        memory\_set\_o\_tag             &                                       & 1      & Gr5                         & Gr2                         & Gr11                        & Gr11                        & Gr11                                       \\
        regfile\_alu\_waddr\_ex\_o\_tag &                                       & 5      & Gr4                         & Gr2                         & Gr12                        & Gr12                        & Gr5 - Gr9                                  \\
        register\_set\_o\_tag           &                                       & 1      & Gr5                         & Gr2                         & Gr13                        & Gr13                        & Gr10                                       \\
        store\_dest\_addr\_ex\_o\_tag   &                                       & 1      & Gr5                         & Gr2                         & Gr14                        & Gr14                        & Gr2                                        \\
        store\_source\_ex\_o\_tag       &                                       & 1      & Gr5                         & Gr2                         & Gr15                        & Gr15                        & Gr3                                        \\
        use\_store\_ops\_ex\_o          &                                       & 1      & Gr5                         & Gr2                         & Gr16                        & Gr16                        & Gr4                                        \\\hdashline
        rf\_reg[0]                      &                                       & 1      & Gr3                         & Gr3                         & Gr17                        & Gr17                        & Gr12                                       \\
        rf\_reg[1]                      &                                       & 1      & Gr3                         & Gr3                         & Gr17                        & Gr17                        & Gr12                                       \\
        rf\_reg[2]                      & \textcolor{LimeGreen}{Register File}  & 1      & Gr3                         & Gr3                         & Gr17                        & Gr17                        & Gr12                                       \\
        \ldots                          & \textcolor{LimeGreen}{Tag}            & \ldots & Gr3                         & Gr3                         & Gr17                        & Gr17                        & Gr12                                       \\
        rf\_reg[30]                     &                                       & 1      & Gr3                         & Gr3                         & Gr17                        & Gr17                        & Gr12                                       \\
        rf\_reg[31]                     &                                       & 1      & Gr3                         & Gr3                         & Gr17                        & Gr17                        & Gr12                                       \\\hdashline
        rs1\_o\_tag                     & \textcolor{DarkOrange}{Execute Stage} & 1      & Gr5                         & Gr4                         & Gr18                        & Gr18                        & Gr35                                       \\\hdashline
        tcr\_q                          & \textcolor{DarkRed}{Control and}      & 32     & Gr1                         & Gr5                         & Gr19                        & Gr19 - Gr26                 & \tableTwoLines{Gr13 - Gr26 /}{Gr28 - Gr30} \\
        tpr\_q                          & \textcolor{DarkRed}{Status Registers} & 32     & Gr2                         & Gr6                         & Gr20                        & Gr27 - Gr34                 & Gr13 - Gr34                                \\\hdashline
        data\_type\_q\_tag              &                                       & 2      & Gr5                         & Gr7                         & Gr21                        & Gr35                        & Gr36 - Gr37                                \\
        data\_we\_q\_tag                & \textcolor{magenta}{Load/Store}       & 1      & Gr5                         & Gr7                         & Gr22                        & Gr36                        & Gr39                                       \\
        rdata\_offset\_q\_tag           & \textcolor{magenta}{Unit}             & 2      & Gr5                         & Gr7                         & Gr23                        & Gr37                        & Gr37 - Gr38                                \\
        rdata\_q\_tag                   &                                       & 4      & Gr5                         & Gr7                         & Gr24                        & Gr38                        & \tableTwoLines{Gr35 - Gr36 /}{Gr38 - Gr39} \\
        \bottomrule
    \end{tabular}
\end{table}

\begin{table}[t]
    \centering
    \footnotesize
    \caption{Registers by Strategy (SECDED count): Summary of Number and Size}
    \label{tab:strategies_summary}
    \begin{tabular}{rcc}
        \toprule
        Strategy            & Number of Registers & Number of Bits                                            \\
        \midrule
        Baseline -- D-RI5CY & 55                  & 127 {\scriptsize {\tiny (0\%)}                                 } \\
        Strategy 1          & 65                  & 157 {\scriptsize {\tiny (\compute{(((157/127) - 1)*100)}{0}\%)}} \\
        Strategy 2          & 69                  & 164 {\scriptsize {\tiny (\compute{(((164/127) - 1)*100)}{0}\%)}} \\
        Strategy 3          & 103                 & 215 {\scriptsize {\tiny (\compute{(((215/127) - 1)*100)}{0}\%)}} \\
        Strategy 4          & 131                 & 266 {\scriptsize {\tiny (\compute{(((266/127) - 1)*100)}{0}\%)}} \\
        Strategy 5          & 133                 & 280 {\scriptsize {\tiny (\compute{(((280/127) - 1)*100)}{0}\%)}} \\
        \bottomrule
    \end{tabular}
\end{table}

In this section, we present our experimental results for our five implemented strategies against the fault models described in Section~\ref{section:chap6_faultmodels}, also, we give the FPGA implementation results for each strategy to compare them taking into account constraints such as area and performance overhead before evaluating the security induced by these strategies. For each fault model, we give the associated result with only the D-RI5CY without any protection, with our simple parity protection, and with the five strategies for Hamming Code and SECDED.
Table~\ref{tab:strategies_register_info} summarises the tables in the previous subsections, presenting the different strategies. Table~\ref{tab:strategies_summary} summarises the number of registers and the associated number of bits for each strategy. Each strategy values are taken from SECDED protection, where there is the maximum number of extra bits. Percentage values are presented in regards with the baseline -- D-RI5CY. Between Strategy 2 and Strategy 3 and Strategy 3 and Strategy 4, there is a 40 \% difference due to increase number of encoder, as we protect each register individually. The discrepancy between Strategy 4 and Strategy 5 is relatively minor, amounting to only 11\%. This is due to the fact that the number of groups is increased from 38 to 39, while the total number of bits rises from 266 to 280. This discrepancy can be attributed to the fact that, whereas Strategy 4 safeguards 16 one-bit registers, Strategy 5 only protects six. The remaining groups are allocated to two-bit or larger registers. 

\subsection{FPGA Implementation Results}

This subsection presents the implementation results targeting the Xilinx Zynq 7000 of the Zedboard development board. Synthesis and implementation is performed using, using Vivado 2023.2.
Table~\ref{tab:chap6_implementation} shows the FPGA implementation results for D-RISCY, compared with various Hamming and SECDED code strategies. The results for the first implementations of Hamming Code, SECDED, and for the D-RI5CY are from Chapter~\ref{chapter:countermeasures} (Table~\ref{tab:chap5_implementation}). The metrics assessed include the number of Look-Up Tables (LUTs), the number of Flip-Flops (FFs), and the maximum operating frequency. The D-RISCY design, without any protection mechanism, utilises 6,911 LUTs and 2,335 FFs, operating at a frequency of \SI{47.4}{\mega\hertz}. In contrast, the application of Hamming code protection strategies increases resource utilisation. Hamming Code Strategy 2 exhibits the highest LUT overhead, increasing by 6.63\% (7,369 LUTs), while its impact on FFs remains relatively modest at a 1.21\% increase. However, this strategy also results in the most significant frequency reduction, dropping by 1.43\% to 46.90 MHz. Among the Hamming strategies, Strategy 5 offers the least resource overhead (4.27\% for LUTs and 3.29\% for FFs) but also experiences a slight frequency reduction of 0.84\%.
SECDED strategies show a similar trend, with Strategy 2 consuming the most resources (7.55\% more LUTs and 1.33\% more FFs than the D-RISCY). Notably, SECDED Strategy 4 offers an improvement in frequency, increasing the maximum operating frequency by 1.43\% to \SI{48.3}{\mega\hertz}, while maintaining a resource overhead of 4.98\% and 1.29\% for LUTs and FFs, respectively. Overall, SECDED strategies generally offer a better frequency compared to Hamming strategies, particularly Strategy 4, which demonstrates an optimal balance between resource overhead and performance improvement. These results highlight the trade-offs between error protection mechanisms and FPGA resource consumption, with Hamming codes leading to greater resource usage and frequency reduction, while SECDED codes, particularly Strategy 4, offer better frequency with moderate resource impact. In conclusion, given that the discrepancy remains within the 1-2\% range, it can be stated that the implementation of the aforementioned protections does not result in any discernible impact on performance, as this range represents the margin of error associated with the Vivado synthesis and implementation process.

Comparing the results from Table~\ref{tab:strategies_summary} and Table~\ref{tab:chap6_implementation}, the data may seem inconsistent. Although strategies 4 and 5 lead to the greatest increase in the number of bits stored in the registers, they result in the smallest increase in surface area. This is because, in strategies 4 and 5, there are numerous groups, though most of these groups only protect 1 or 2 bits. As a result, the majority of encoders are lightweight, and the single parity registers are only one bit long. During synthesis and implementation, optimisation likely occurs, to minimise the area overhead.

\begin{table}[t]
    \footnotesize
    \centering
    \caption{FPGA implementation results — Vivado 2023.2}
    \label{tab:chap6_implementation}
    \begin{tabular}{@{}rccc@{}}
        \toprule
        \tableCentered{Protection} & Number of LUTs & Number of FFs  & Maximum frequency                \\ \midrule
        D-RI5CY                    & \num{6911} {\tiny (0\%)   } & \num{2335} {\tiny (0\%)   } & \SI{47.6}{\mega\hertz} {\tiny (0\%)    } \\
        Hamming Code Strategy 1    & \num{7283} {\tiny (5.38\%)} & \num{2361} {\tiny (1.11\%)} & \SI{47.4}{\mega\hertz} {\tiny (-0.36\%)} \\
        Hamming Code Strategy 2    & \num{7369} {\tiny (6.63\%)} & \num{2363} {\tiny (1.2\%) } & \SI{46.9}{\mega\hertz} {\tiny (-1.43\%)} \\
        Hamming Code Strategy 3    & \num{7251} {\tiny (4.92\%)} & \num{2361} {\tiny (1.11\%)} & \SI{46.8}{\mega\hertz} {\tiny (-1.67\%)} \\
        Hamming Code Strategy 4    & \num{7203} {\tiny (4.23\%)} & \num{2371} {\tiny (1.54\%)} & \SI{47.6}{\mega\hertz} {\tiny (0\%)    } \\
        Hamming Code Strategy 5    & \num{7182} {\tiny (3.92\%)} & \num{2411} {\tiny (3.25\%)} & \SI{47.3}{\mega\hertz} {\tiny (-0.57\%)} \\
        SECDED Strategy 1          & \num{7428} {\tiny (7.48\%)} & \num{2366} {\tiny (1.33\%)} & \SI{47.2}{\mega\hertz} {\tiny (-0.95\%)} \\
        SECDED Strategy 2          & \num{7433} {\tiny (7.55\%)} & \num{2366} {\tiny (1.41\%)} & \SI{47.2}{\mega\hertz} {\tiny (-0.95\%)} \\
        SECDED Strategy 3          & \num{7324} {\tiny (5.98\%)} & \num{2368} {\tiny (1.28\%)} & \SI{47.5}{\mega\hertz} {\tiny (-0.24\%)} \\
        SECDED Strategy 4          & \num{7255} {\tiny (4.98\%)} & \num{2365} {\tiny (1.93\%)} & \SI{48.3}{\mega\hertz} {\tiny (1.43\%) } \\
        SECDED Strategy 5          & \num{7228} {\tiny (4.59\%)} & \num{2428} {\tiny (3.98\%)} & \SI{48.3}{\mega\hertz} {\tiny (1.43\%) } \\
        \bottomrule
    \end{tabular}
\end{table}

\subsection{Fault Models Evaluation}

\begin{figure}[ht]
    \centering
    \includegraphics[width=\textwidth]{c6_group_composition/img/heatmap_buffer_overflow_wop_1_single_bitflip_spatial_2.pdf}
    \caption{Distribution of successes in the case of buffer overflow, unprotected, with a \textit{single bit-flip in two registers at a given clock cycle} fault model (1406 successes).}
    \label{fig:heatmap_bo_spatial_wop}
\end{figure}

\begin{figure}[ht]
    \centering
    \includegraphics[width=\textwidth]{c6_group_composition/img/heatmap_buffer_overflow_hamming_5_single_bitflip_spatial_2.pdf}
    \caption{Distribution of successes in the case of buffer overflow, with the strategy 5 of Hamming Code, with a \textit{single bit-flip in two registers at a given clock cycle} fault model (98 successes).}
    \label{fig:heatmap_bo_spatial_hc5}
\end{figure}

\begin{figure}[ht]
    \centering
    \includegraphics[width=.97\textwidth]{c6_group_composition/img/heatmap_buffer_overflow_hamming_2_multi_bitflip_reg_multi_2.pdf}
    \caption{Distribution of successes in the case of buffer overflow, with the 2nd strategy of Hamming Code, with \textit{multi-bit faults in two registers at a given clock cycle} fault model (4356 successes).}
    \label{fig:heatmap_bo_multireg_hc2}
\end{figure}

\begin{figure}[ht]
    \centering
    \includegraphics[width=\textwidth]{c6_group_composition/img/heatmap_buffer_overflow_secded_5_multi_bitflip_reg_multi_2.pdf}
    \caption{Distribution of successes in the case of buffer overflow, with the strategy 5 of SECDED, with \textit{multi-bit faults in two registers at a given clock cycle} fault model (66 successes).}
    \label{fig:heatmap_bo_multireg_sd5}
\end{figure}

In this subsection, we present our fault injection campaigns results targeting the DIFT-related registers of the D-RI5CY and its associated protection registers. We present one table for each fault model considered in this chapter containing the results for all the three use cases and all strategies (no protection, simple parity, Hamming Code 1 -- 5, and SECDED 1 -- 5).

Table~\ref{tab:chap6_results_single_bitflip_spatial} shows the results obtained from the \textit{single bit-flip in two registers at a given clock cycle} fault model according to each use case. This table shows that without any protection in the case of the buffer overflow, the D-RI5CY lead to 1406 successes with this fault model, while with the simple parity, we decrease from 1406 to 239 successes. However, due to the increase number of registers and the fact that Hamming Code can detect only one error, as we inject two faults, Hamming Code try to correct a fault but in many cases, it can cause a third fault increasing the number of successes. Nevertheless, the different proposed strategies decrease this number of successes by a factor of approximatively 50, going from 2.93\% to 0.06\%. The SECDED protection detect all injected faults, thus no successes happen thanks to this countermeasure.
Table~\ref{tab:chap6_results_multi_bitflip_reg} shows the results obtained from the \textit{multi-bit faults in one register at a given clock cycle} fault model. This table depicts that Hamming Code induces more or less the same amount of fault than without any protection, while simple parity shows better performances in terms of security. However, we can note a slight decrease for the second and third case. The best protection with Hamming Code is the fifth strategy, as it shows the lower amount of successes. On the other hand, SECDED offers the best results in terms of protection, although it is not always successful against attacks depending on the strategy considered. For example, for the second use case, all five strategies led to some successes, whereas for the third use case, only the first strategy led to successes. As we are injecting multiple faults, up to 6 bit-flips, with this fault model, it is normal that there will still be some successes.
Table~\ref{tab:chap6_results_multi_bitflip_reg_multi} shows the results obtained from the \textit{multi-bit faults in two registers at a given clock cycle} fault model. This table represents the more complex fault model considered, for each use case and for each strategy there will be some successes. It is due to that fact that we can inject up to 12 faults in two registers, even though SECDED can detect up to two faults. Hamming Code increase the number of successes depending on the strategy and the use case. However, if we just take into account the percentage, the different strategies allow decreasing with a significant factor the number of successes. For example, for the buffer overflow, the highest ratio of successes is due to the second strategy of Hamming Code at 1.33\% and the lowest ratio is thanks to the fifth strategy of SECDED at 0.0067\% (round to 0.01 in the table), which is approximatively 200 times less than 1.33\%.

\begin{table}[ht]
    \scriptsize
    \centering
    \caption{Logical fault injection simulation campaigns results for single bit-flip in two registers at a given clock cycle}
    \label{tab:chap6_results_single_bitflip_spatial}
    \setlength{\tabcolsep}{2pt}
    \begin{tabular}{@{}ccccccccccc@{}}
        \toprule
                                                           &                & Crash & Silent      & Delay      & Detection   & \tableTwoLines{Detection \&}{Correction} & \tableTwoLines{Double Error}{Detection} & Success                                              & Total         & \tableTwoLines{Execution}{time} \\\midrule
        \multirow{12}{*}{\tableTwoLines{Buffer}{Overflow}} & No protection  & 0     & \num{45097} & \num{1503} & --          & --                                       & --                                      & \num{1406} {\tiny (2.93\%)}                          & \num{48006}   & 13:43                           \\
                                                           & Simple parity  & 0     & \num{10551} & 134        & \num{40952} & --                                       & --                                      & 239 {\tiny (0.46\%)}                                 & \num{51876 }  & 14:07                           \\
                                                           & Hamming 1 & 0     & 0           & 575        & --          & \num{67829 }                             & --                                      & 452 {\tiny (0.66\%)}                                 & \num{68856 }  & 19:48                           \\
                                                           & Hamming 2 & 0     & 0           & 297        & --          & \num{72867 }                             & --                                      & 312 {\tiny (0.42\%)}                                 & \num{73476 }  & 97:16                           \\
                                                           & Hamming 3 & 0     & 0           & 263        & --          & \num{108326}                             & --                                      & 281 {\tiny (0.26\%)}                                 & \num{108870}  & 30:00                           \\
                                                           & Hamming 4 & 0     & 0           & 57         & --          & \num{155112}                             & --                                      & 99  {\tiny (0.06\%)}                                 & \num{155268}  & 46:30                           \\
                                                           & Hamming 5 & 0     & 0           & 55         & --          & \num{173367}                             & --                                      & 98  {\tiny (0.06\%)}                                 & \num{173520}  & 53:00                           \\
                                                           & SECDED 1       & 0     & 2436        & 0          & --          & \num{59424 }                             & \num{11616}                             & 0                                                    & \num{73476 }  & 20:56                           \\
                                                           & SECDED 2       & 0     & 0           & 0          & --          & \num{69354 }                             & \num{10842}                             & 0                                                    & \num{80196 }  & 21:49                           \\
                                                           & SECDED 3       & 0     & 0           & 0          & --          & \num{128376}                             & \num{9654 }                             & 0                                                    & \num{138030}  & 40:14                           \\
                                                           & SECDED 4       & 0     & 0           & 0          & --          & \num{204060}                             & \num{7410 }                             & 0                                                    & \num{211470}  & 64:02                           \\
                                                           & SECDED 5       & 0     & \num{12096} & 0          & --          & \num{214722}                             & \num{7542 }                             & 0                                                    & \num{234360}  & 69:44                           \\\midrule
        \multirow{12}{*}{\tableTwoLines{Format}{String}}   & No protection  & 0     & \num{55589} & \num{5035} & --          & --                                       & --                                      & \num{3384} {\tiny (5.29\%)}                          & \num{64008}   & 163:09                          \\
                                                           & Simple parity  & 0     & \num{13361} & 450        & \num{54590} & --                                       & --                                      & 767  {\tiny (1.11\%)}                                & \num{69168 }  & 114:06                          \\
                                                           & Hamming 1 & 0     & 0           & 1709       & --          & \num{89010 }                             & --                                      & 1089 {\tiny (1.19\%)}                                & \num{91808 }  & 179:38                          \\
                                                           & Hamming 2 & 0     & 0           & 982        & --          & \num{96182 }                             & --                                      & 804  {\tiny (0.82\%)}                                & \num{97968 }  & 136:40                          \\
                                                           & Hamming 3 & 0     & 0           & 659        & --          & \num{143883}                             & --                                      & 618  {\tiny (0.43\%)}                                & \num{145160}  & 261:40                          \\
                                                           & Hamming 4 & 0     & 0           & 379        & --          & \num{206423}                             & --                                      & 222  {\tiny (0.11\%)}                                & \num{207024}  & 368:10                          \\
                                                           & Hamming 5 & 0     & 0           & 391        & --          & \num{230758}                             & --                                      & 211  {\tiny (0.09\%)}                                & \num{231360}  & 445:58                          \\
                                                           & SECDED 1       & 0     & 0           & 0          & --          & \num{82480 }                             & \num{15488}                             & 0                                                    & \num{97968 }  & 233:28                          \\
                                                           & SECDED 2       & 0     & 0           & 0          & --          & \num{92472 }                             & \num{14456}                             & 0                                                    & \num{106928}  & 185:35                          \\
                                                           & SECDED 3       & 0     & 0           & 0          & --          & \num{171168}                             & \num{12872}                             & 0                                                    & \num{184040}  & 317:20                          \\
                                                           & SECDED 4       & 0     & 0           & 0          & --          & \num{272080}                             & \num{9880 }                             & 0                                                    & \num{281960}  & 462:58                          \\
                                                           & SECDED 5       & 0     & \num{16128} & 0          & --          & \num{286296}                             & \num{10056}                             & 0                                                    & \num{312480}  & 558:16                          \\\midrule
        \multirow{12}{*}{\tableTwoLines{Compare}{Compute}} & No protection  & 0     & \num{29906} & 919        & --          & --                                       & --                                      & \num{1179} {\tiny (3.68\%)}                          & \num{32004}   & 05:24                           \\
                                                           & Simple parity  & 0     & \num{6697}  & 202        & \num{27678} & --                                       & --                                      & 7   {\tiny (0.02\%)}                                 & \num{34584 }  & 04:48                           \\
                                                           & Hamming 1 & 0     & 0           & 450        & --          & \num{45192  }                            & --                                      & 262 {\tiny (0.57\%)}                                 & \num{45904 }  & 09:21                           \\
                                                           & Hamming 2 & 0     & 0           & 440        & --          & \num{48419  }                            & --                                      & 125 {\tiny (0.26\%)}                                 & \num{48984 }  & 08:47                           \\
                                                           & Hamming 3 & 0     & 0           & 315        & --          & \num{72140  }                            & --                                      & 125 {\tiny (0.17\%)}                                 & \num{72580 }  & 13:53                           \\
                                                           & Hamming 4 & 0     & 0           & 97         & --          & \num{103345 }                            & --                                      & 70  {\tiny (0.07\%)}                                 & \num{103512}  & 22:23                           \\
                                                           & Hamming 5 & 0     & 0           & 96         & --          & \num{115511 }                            & --                                      & 73  {\tiny (0.06\%)}                                 & \num{115680}  & 23:48                           \\
                                                           & SECDED 1       & 0     & 0           & 0          & --          & \num{37740  }                            & \num{11244}                             & 0                                                    & \num{48984 }  & 17:00                           \\
                                                           & SECDED 2       & 0     & 0           & 0          & --          & \num{46236  }                            & \num{7228}                              & 0                                                    & \num{53464 }  & 10:12                           \\
                                                           & SECDED 3       & 0     & 0           & 0          & --          & \num{85584  }                            & \num{6436}                              & 0                                                    & \num{92020 }  & 18:25                           \\
                                                           & SECDED 4       & 0     & 0           & 0          & --          & \num{136040 }                            & \num{4940}                              & 0                                                    & \num{140980}  & 28:37                           \\
                                                           & SECDED 5       & 0     & 0           & 0          & --          & \num{151212 }                            & \num{5028}                              & 0                                                    & \num{156240}  & 32:52                           \\\midrule
        Total                                              &                &       &             &            &             &                                          &                                         & \num{11823} {\tiny (\compute{11823*100/4252212}{2})} & \num{4252212} &                                 \\
        \bottomrule
    \end{tabular}
\end{table}

\begin{table}[ht]
    \scriptsize
    \centering
    \caption{Logical fault injection simulation campaigns results for exhaustive multi-bits faults in one register at a given clock cycle}
    \label{tab:chap6_results_multi_bitflip_reg}
    \setlength{\tabcolsep}{3pt}
    \begin{tabular}{@{}ccccccccccc@{}}
        \toprule
                                                           &                & Crash & Silent & Delay & Detection & \tableTwoLines{Detection \&}{Correction} & \tableTwoLines{Double Error}{Detection} & Success                                       & Total       & \tableTwoLines{Execution}{time} \\\midrule
        \multirow{12}{*}{\tableTwoLines{Buffer}{Overflow}} & No protection  & 0     & 927    & 6     & --        & --                                       & --                                      & 3 {\tiny (0.32\%)}                            & 936         & 00:08                           \\
                                                           & Simple parity  & 0     & 498    & 0     & 498       & --                                       & --                                      & 0                                             & 996         & 00:14                           \\
                                                           & Hamming 1 & 0     & 0      & 20    & --        & 1962                                     & --                                      & 10 {\tiny (0.50\%)}                           & 1992        & 00:28                           \\
                                                           & Hamming 2 & 0     & 0      & 12    & --        & 2038                                     & --                                      & 14 {\tiny (0.68\%)}                           & 2064        & 00:32                           \\
                                                           & Hamming 3 & 0     & 0      & 12    & --        & 2352                                     & --                                      & 12 {\tiny (0.51\%)}                           & 2376        & 00:28                           \\
                                                           & Hamming 4 & 0     & 0      & 12    & --        & 2712                                     & --                                      & 12 {\tiny (0.44\%)}                           & 2736        & 00:35                           \\
                                                           & Hamming 5 & 0     & 0      & 12    & --        & 2976                                     & --                                      & 12 {\tiny (0.40\%)}                           & 3000        & 00:45                           \\
                                                           & SECDED 1       & 0     & 0      & 8     & --        & 1393                                     & 648                                     & 3 {\tiny (0.15\%)}                            & 2052        & 00:30                           \\
                                                           & SECDED 2       & 0     & 0      & 5     & --        & 1475                                     & 666                                     & 2 {\tiny (0.09\%)}                            & 2148        & 00:30                           \\
                                                           & SECDED 3       & 0     & 0      & 4     & --        & 1932                                     & 726                                     & 2 {\tiny (0.08\%)}                            & 2664        & 00:40                           \\
                                                           & SECDED 4       & 0     & 0      & 0     & --        & 2370                                     & 822                                     & 0                                             & 3192        & 00:45                           \\
                                                           & SECDED 5       & 0     & 0      & 0     & --        & 2670                                     & 798                                     & 0                                             & 3468        & 00:55                           \\\midrule
        \multirow{12}{*}{\tableTwoLines{Format}{String}}   & No protection  & 0     & 1202   & 32    & --        & --                                       & --                                      & 14 {\tiny (1.12\%)}                           & 1248        & 01:24                           \\
                                                           & Simple parity  & 0     & 661    & 0     & 665       & --                                       & --                                      & 2  {\tiny (0.15\%)}                           & 1328        & 02:12                           \\
                                                           & Hamming 1 & 0     & 0      & 62    & --        & 2565                                     & --                                      & 29 {\tiny (1.09\%)}                           & 2656        & 04:24                           \\
                                                           & Hamming 2 & 0     & 0      & 53    & --        & 2666                                     & --                                      & 33 {\tiny (1.20\%)}                           & 2752        & 03:36                           \\
                                                           & Hamming 3 & 0     & 0      & 47    & --        & 3090                                     & --                                      & 31 {\tiny (0.98\%)}                           & 3168        & 03:55                           \\
                                                           & Hamming 4 & 0     & 0      & 47    & --        & 3570                                     & --                                      & 31 {\tiny (0.85\%)}                           & 3648        & 04:25                           \\
                                                           & Hamming 5 & 0     & 0      & 41    & --        & 3930                                     & --                                      & 29 {\tiny (0.73\%)}                           & 4000        & 05:18                           \\
                                                           & SECDED 1       & 0     & 0      & 22    & --        & 1832                                     & 864                                     & 18 {\tiny (0.66\%)}                           & 2736        & 03:30                           \\
                                                           & SECDED 2       & 0     & 0      & 14    & --        & 1938                                     & 894                                     & 18 {\tiny (0.63\%)}                           & 2864        & 03:48                           \\
                                                           & SECDED 3       & 0     & 0      & 10    & --        & 2560                                     & 968                                     & 14 {\tiny (0.39\%)}                           & 3552        & 04:42                           \\
                                                           & SECDED 4       & 0     & 0      & 5     & --        & 3146                                     & 1096                                    & 9  {\tiny (0.21\%)}                           & 4256        & 05:42                           \\
                                                           & SECDED 5       & 0     & 0      & 4     & --        & 3554                                     & 1064                                    & 2  {\tiny (0.04\%)}                           & 4624        & 06:30                           \\\midrule
        \multirow{12}{*}{\tableTwoLines{Compare}{Compute}} & No protection  & 0     & 616    & 2     & --        & --                                       & --                                      & 6  {\tiny (0.96\%)}                           & 624         & 00:04                           \\
                                                           & Simple parity  & 0     & 330    & 0     & 334       & --                                       & --                                      & 0                                             & 664         & 00:04                           \\
                                                           & Hamming 1 & 0     & 0      & 9     & --        & 1311                                     & --                                      & 8 {\tiny (0.60\%)}                            & 1328        & 00:09                           \\
                                                           & Hamming 2 & 0     & 0      & 15    & --        & 1356                                     & --                                      & 5 {\tiny (0.36\%)}                            & 1376        & 00:09                           \\
                                                           & Hamming 3 & 0     & 0      & 12    & --        & 1567                                     & --                                      & 5 {\tiny (0.32\%)}                            & 1584        & 00:11                           \\
                                                           & Hamming 4 & 0     & 0      & 12    & --        & 1807                                     & --                                      & 5 {\tiny (0.27\%)}                            & 1824        & 00:13                           \\
                                                           & Hamming 5 & 0     & 0      & 12    & --        & 1983                                     & --                                      & 5 {\tiny (0.25\%)}                            & 2000        & 00:14                           \\
                                                           & SECDED 1       & 0     & 0      & 2     & --        & 888                                      & 476                                     & 2 {\tiny (0.15\%)}                            & 1368        & 00:09                           \\
                                                           & SECDED 2       & 0     & 0      & 6     & --        & 977                                      & 449                                     & 0                                             & 1432        & 00:10                           \\
                                                           & SECDED 3       & 0     & 0      & 2     & --        & 1290                                     & 484                                     & 0                                             & 1776        & 00:12                           \\
                                                           & SECDED 4       & 0     & 0      & 0     & --        & 1580                                     & 548                                     & 0                                             & 2128        & 00:15                           \\
                                                           & SECDED 5       & 0     & 0      & 0     & --        & 1780                                     & 532                                     & 0                                             & 2312        & 00:16                           \\\midrule
        Total                                              &                &       &        &       &           &                                          &                                         & \num{336} {\tiny(\compute{336*100/82872}{2})} & \num{82872} &                                 \\
        \bottomrule
    \end{tabular}
\end{table}

\begin{table}[ht]
    \scriptsize
    \centering
    \caption{Logical fault injection simulation campaigns results for exhaustive multi-bits faults in two registers at a given clock cycle}
    \label{tab:chap6_results_multi_bitflip_reg_multi}
    \setlength{\tabcolsep}{1pt}
    \begin{tabular}{@{}ccccccccccc@{}}
        \toprule
                                                           &               & Crash & Silent       & Delay       & Detection   & \tableTwoLines{Detection \&}{Correction} & \tableTwoLines{Double Error}{Detection} & Success                      & Total          & \tableTwoLines{Execution}{time} \\\midrule
        \multirow{12}{*}{\tableTwoLines{Buffer}{Overflow}} & No protection & 0     & \num{67072 } & 926         & --          & --                                       & --                                      & 450 {\tiny (0.66\%)}         & \num{68448 }   & 11:11                           \\
                                                           & Simple parity & 0     & \num{24622 } & 8           & \num{53359} & --                                       & --                                      & 59 {\tiny (0.08\%)}          & \num{78048 }   & 25:00                           \\
                                                           & Hamming 1     & 0     & \num{294464} & 6273        & --          & --                                       & --                                      & 3103 {\tiny (1.02\%)}        & \num{303840}   & 99:36                           \\
                                                           & Hamming 2     & 0     & 0            & 3992        & --          & \num{319588}                             & --                                      & 4356 {\tiny (1.33\%)}        & \num{327936}   & 131:12                          \\
                                                           & Hamming 3     & 0     & 0            & 4557        & --          & \num{436187}                             & --                                      & 4408 {\tiny (0.99\%)}        & \num{445152}   & 121:20                          \\
                                                           & Hamming 4     & 0     & 0            & 5446        & --          & \num{590953}                             & --                                      & 5329 {\tiny (0.89\%)}        & \num{601728}   & 167:00                          \\
                                                           & Hamming 5     & 0     & 0            & 5987        & --          & \num{714873}                             & --                                      & 5860 {\tiny (0.81\%)}        & \num{726720}   & 210:31                          \\
                                                           & SECDED 1      & 0     & 0            & 1911        & --          & \num{150791}                             & \num{170575}                            & 723 {\tiny (0.22\%)}         & \num{324000}   & 86:59                           \\
                                                           & SECDED 2      & 0     & 0            & 1186        & --          & \num{170805}                             & \num{184761}                            & 584 {\tiny (0.16\%)}         & \num{357336}   & 94:04                           \\
                                                           & SECDED 3      & 0     & 0            & 1230        & --          & \num{300260}                             & \num{263665}                            & 669 {\tiny (0.12\%)}         & \num{565824}   & 161:30                          \\
                                                           & SECDED 4      & 0     & 0            & 18          & --          & \num{457498}                             & \num{368959}                            & 61 {\tiny (0.01\%)}          & \num{826536}   & 244:48                          \\
                                                           & SECDED 5      & 0     & 0            & 39          & --          & \num{576992}                             & \num{401407}                            & 66 {\tiny (0.01\%)}          & \num{978504}   & 284:45                          \\\midrule
        \multirow{12}{*}{\tableTwoLines{Format}{String}}   & No protection & 0     & \num{84419}  & 4836        & --          & --                                       & --                                      & 2009 {\tiny (2.20\%)}        & \num{91264 }   & 104:15                          \\
                                                           & Simple parity & 0     & \num{32275}  & 147         & \num{71198} & --                                       & --                                      & 444 {\tiny (0.43\%)}         & \num{104064}   & 138:40                          \\
                                                           & Hamming 1     & 0     & 0            & \num{20050} & --          & \num{375836}                             & --                                      & 9234 {\tiny (2.28\%)}        & \num{405120}   & 902:08                          \\
                                                           & Hamming 2     & 0     & 0            & \num{17597} & --          & \num{408894}                             & --                                      & \num{10757} {\tiny (2.46\%)} & \num{437248}   & 774:40                          \\
                                                           & Hamming 3     & 0     & 0            & \num{17926} & --          & \num{564154}                             & --                                      & \num{11456} {\tiny (1.93\%)} & \num{593536}   & 1021:50                         \\
                                                           & Hamming 4     & 0     & 0            & \num{20986} & --          & \num{767604}                             & --                                      & \num{13714} {\tiny (1.71\%)} & \num{802304}   & 1418:24                         \\
                                                           & Hamming 5     & 0     & 0            & \num{20547} & --          & \num{934077}                             & --                                      & \num{14336} {\tiny (1.48\%)} & \num{968960}   & 1690:05                         \\
                                                           & SECDED 1      & 0     & 0            & 5408        & --          & \num{194766}                             & \num{227655}                            & 4171 {\tiny (0.97\%)}        & \num{432000}   & 740:21                          \\
                                                           & SECDED 2      & 0     & 0            & 3611        & --          & \num{220568}                             & \num{247704}                            & 4565 {\tiny (0.96\%)}        & \num{476448}   & 836:41                          \\
                                                           & SECDED 3      & 0     & 0            & 3088        & --          & \num{395487}                             & \num{351553}                            & 4304 {\tiny (0.57\%)}        & \num{754432}   & 1305:36                         \\
                                                           & SECDED 4      & 0     & 0            & 1939        & --          & \num{604649}                             & \num{491945}                            & 3515 {\tiny (0.32\%)}        & \num{1102048}  & 1915:20                         \\
                                                           & SECDED 5      & 0     & 0            & 1938        & --          & \num{766527}                             & \num{535209}                            & 998 {\tiny (0.08\%)}         & \num{1304672}  & 2287:38                         \\\midrule
        \multirow{12}{*}{\tableTwoLines{Compare}{Compute}} & No protection & 0     & \num{44444}  & 323         & --          & --                                       & --                                      & 865 {\tiny (1.90\%)}         & \num{45632 }   & 05:36                           \\
                                                           & Simple parity & 0     & \num{16033}  & 53          & \num{35943} & --                                       & --                                      & 3 {\tiny (0.01\%)}           & \num{52032 }   & 08:05                           \\
                                                           & Hamming 1     & 0     & 0            & 2912        & --          & \num{196958}                             & --                                      & 2690 {\tiny (1.33\%)}        & \num{202560}   & 34:17                           \\
                                                           & Hamming 2     & 0     & 0            & 4677        & --          & \num{211969}                             & --                                      & 1978 {\tiny (0.90\%)}        & \num{218624}   & 37:24                           \\
                                                           & Hamming 3     & 0     & 0            & 4377        & --          & \num{290302}                             & --                                      & 2089 {\tiny (0.70\%)}        & \num{296768}   & 53:50                           \\
                                                           & Hamming 4     & 0     & 0            & 5282        & --          & \num{393423}                             & --                                      & 2447 {\tiny (0.61\%)}        & \num{401152}   & 74:31                           \\
                                                           & Hamming 5     & 0     & 0            & 5829        & --          & \num{475987}                             & --                                      & 2664 {\tiny (0.55\%)}        & \num{484480}   & 94:21                           \\
                                                           & SECDED 1      & 0     & 0            & 656         & --          & \num{92123 }                             & \num{122731}                            & 490 {\tiny (0.23\%)}         & \num{216000}   & 35:42                           \\
                                                           & SECDED 2      & 0     & 0            & 1452        & --          & \num{112110}                             & \num{124659}                            & 3 {\tiny (0.00\%)}           & \num{238224}   & 43:38                           \\
                                                           & SECDED 3      & 0     & 0            & 640         & --          & \num{200702}                             & \num{175871}                            & 3 {\tiny (0.00\%)}           & \num{377216}   & 72:32                           \\
                                                           & SECDED 4      & 0     & 0            & 68          & --          & \num{304920}                             & \num{246033}                            & 3 {\tiny (0.00\%)}           & \num{551024}   & 109:22                          \\
                                                           & SECDED 5      & 0     & 0            & 96          & --          & \num{384572}                             & \num{267665}                            & 3 {\tiny (0.00\%)}           & \num{652336}   & 128:21                          \\\midrule
        Total                                              &               &       &              &             &             &                                          &                                         & \num{118409} {\tiny (0.7\%)} & \num{16812216} &                                 \\
        \bottomrule
    \end{tabular}
\end{table}

Figure~\ref{fig:heatmap_bo_spatial_wop} and Figure~\ref{fig:heatmap_bo_spatial_hc5} present the distribution of successes (coloured boxes) according to the buffer overflow use case and taking into account the \textit{single bit-flip in two registers at a given clock cycle} fault model. Figure~\ref{fig:heatmap_bo_spatial_wop} depicts the distribution of the 1406 successes, it shows 2 lines and 3 columns with many coloured boxes. These boxes show where are the most critical registers to be protected for this fault model. By comparing the two figures, we can see a major decrease of coloured boxes, showing that the protection is effective. The highest number without protection is 272 at the intersection of \texttt{tcr\_q} and \texttt{tpr\_q}, while when applying the protection this number decrease to 10 at the intersection of \texttt{hc\_csr\_group21/hc\_o} and \texttt{tcr\_q}. The \texttt{hc\_csr\_group21/hc\_o} register protect the 21st bit of the \texttt{tcr\_q} which store the \textit{Execute Check} bit of the security policy (see Chapter~\ref{chapter:dift_assessment} for more details).
Figure~\ref{fig:heatmap_bo_multireg_hc2} and Figure~\ref{fig:heatmap_bo_multireg_sd5} present the distribution of successes (coloured boxes) according to the buffer overflow use case and taking into account the \textit{multi-bit faults in two registers at a given clock cycle} fault model. Figure~\ref{fig:heatmap_bo_multireg_hc2} shows 5 lines and 2 columns representing different critical registers. The highest number is set at 339 successes. This line represent the redundancy bits to protect the \texttt{tcr\_q} which explain the high number of successes due to this register. Once the best protection has been applied, the number decrease to 66 successes with only 1 line, the redundancy bits of the register file which is the only register not protected in the same way as the other due to the constraints we consider on the number of register file write ports.

%%%%%%%%%%%%%%%%%%%%%%%%%%%%%%%%%%%%%%%%%%%%%%%%%%%%%%%%%%%%%%%%%%%%%%%%%%%%%%%%%%%%%%%%%%%%%%%
\section{Discussion}
\label{section:chap6_discussion}

In this section, we discuss the results obtained considered the fault model of this chapter and the use cases.

Against our three fault models and taking into account the three use cases, \textit{single bit-flip in two registers at a given clock cycle}, \textit{multi-bit faults in one register at a given clock cycle}, \textit{multi-bit faults in two registers at a given clock cycle}, the D-RI5CY show a lot of vulnerabilities alone against fault injections. Our first protection, the simple parity, helps to reduce this number of successes by only detecting the fault. On the other hand, Hamming Code has mixed results depending on the fault model and the strategy. In fact, given that it can correct one fault, and that at least two are inserted, it will attempt to correct, but will often introduce a third, which leads to an increase in the number of successes. The implemented strategies reduce the probability of correcting a faulty bit when multiple faults are introduced into a single register. This is achieved by splitting the register across multiple encoders, which enables the detection and correction of faults as if they were single-bit faults. We can assume that the finer the grain, the greater the protection. However, this raises the question of the area overhead of this protection. It has been demonstrated that the proposed protections are effective in protecting the two CSR registers, TPR and TCR. However, it is noteworthy that some successes still occur. It would be prudent to consider implementing a more robust protection mechanism for these registers, such as an ECC, to detect and correct multi-bit errors.

Now we can discuss and compare these implementations in terms of area and performance overhead. The D-RI5CY, only, use 6911 LUTs, 2335 FFs at a frequency of \SI{47.6}{\mega\hertz}. However, if we consider the fifth strategy with SECDED which gives the best security results on all fault models only adds 4.59\% overhead on LUTs (7228) and 3.98\% of FFs (2428). The frequency measure indicates an increase to \SI{48.3}{\mega\hertz}, which needs to be taken with precaution.
Nevertheless, if we consider an embedded system with constraints such as performance and area and make the best security compromise, it turns out that strategy four or five are the best. Although a 5\% increase in surface area may seem high, it's important to remember that we are working on a very small processor that contains only 6597 LUTs, and 2211 FFs.

%%%%%%%%%%%%%%%%%%%%%%%%%%%%%%%%%%%%%%%%%%%%%%%%%%%%%%%%%%%%%%%%%%%%%%%%%%%%%%%%%%%%%%%%%%%%%%%
\section{Summary}

This chapter has presented four different implementation's strategies of countermeasures to better protect the D-RI5CY mechanism against more complex fault models. We evaluated each of them in terms of security against more complex fault models considering multi bit-flips faults in one or two register(s) in one clock cycle and single bit-flip in two registers at one clock cycle. The obtained results show good performance in terms of security, area, and performance overhead. Thus, our strategies allow protecting efficiently our DIFT against fault injection attacks using lightweight countermeasures. However, as we test exhaustively all possible cases, there are still some successes due to some combination when targeting specific registers. For these cases, another protection, such as a more robust ECC like BCH or LDPC code, would be interesting to evaluate.

%%%%%%%%%%%%%%%%%%%%%%%%%%%%%%%%%%%%%%%%%%%%%%%%%%%%%%%%%%%%%%%%%%%%%%%%%%%%%%%%%%%%%%%%%%%%%%%

\clearemptydoublepage
\backmatter
\chapter{Conclusion}
\chaptermark{Conclusion}
\label{chapter:conclusion}

\epigraph{\textit{The only truly secure system is one that is powered off, cast in a block of concrete and sealed in a lead-lined room with armed guards - and even then I have my doubts.}}{Gene Spafford}

\minitoc

%%%%%%%%%%%%%%%%%%%%%%%%%%%%%%%%%%%%%%%%%%%%%%%%%%%%%%%%%%%%%%%%%%%%%%%%%%%%%%%%%%%%%%%%%%%%%%%
\section{Synthesis}

With the rapid expansion of IoT and the growing ubiquity of embedded systems, ensuring robust security has become a critical priority for both hardware designers and software developers. Protecting these systems from potential threats, especially physical attacks, remains a key challenge. Among these threats, Fault Injection Attacks (FIA) stand out as a significant risk due to their capacity to disrupt device operation and compromise data integrity.

FIAs are particularly dangerous because they allow attackers to inject faults into a system during runtime, potentially bypassing even the most robust software security mechanisms. By manipulating voltage, clock signals, or using techniques like laser-based injections, adversaries can induce unexpected behaviour, leading to data leakage, corruption, or system hijacking. These attacks are becoming more accessible due to the decreasing cost of fault injection tools, making it imperative to design systems with built-in resilience. Existing security mechanisms, like Dynamic Information Flow Tracking (DIFT), which is use as a security against software threats, are not immune to these attacks, necessitating deeper investigation and the development of tailored countermeasures. Without effective defences, FIAs remain a potent threat, capable of undermining the reliability and trustworthiness of critical IoT systems.

This thesis has aimed to address these challenges by assessing vulnerabilities and proposing lightweight countermeasures to strengthen digital systems against FIAs. By evaluating and improving the security of Dynamic Information Flow Tracking (DIFT) mechanisms, we have proposed a solution on how to protect systems against sophisticated physical and software-based threats. In this concluding chapter, we summarise the contributions made, reflect on the findings, and discuss the potential for further advancements in securing embedded systems against physical attacks.

In Chapter~\ref{chapter:dift_assessment}, we assessed the DIFT mechanisms against different simple fault model such as \textit{bit set}, \textit{bit reset}, and \textit{single bit-flip} theoretically. We have shown that the DIFT mechanism considered, the D-RI5CY, is mostly vulnerable to single bit-flip fault as its tag datapath is 1-bit. The fault injection campaign in simulation have confirmed our previous results and let us with different critical registers depending on the use case considered.

In Chapter~\ref{chapter:fissa}, we presented our developped tool to facilitate the \textit{Security by Design}, allowing the designer to assess its design during the conception phase.

%%%%%%%%%%%%%%%%%%%%%%%%%%%%%%%%%%%%%%%%%%%%%%%%%%%%%%%%%%%%%%%%%%%%%%%%%%%%%%%%%%%%%%%%%%%%%%%
\section{Perspectives}

%%%%%%%%%%%%%%%%%%%%%%%%%%%%%%%%%%%%%%%%%%%%%%%%%%%%%%%%%%%%%%%%%%%%%%%%%%%%%%%%%%%%%%%%%%%%%%%

% Chapitre pour la bibliographie
% Bibliography chapter
\clearemptydoublepage
\phantomsection % To have a correct link in the table of contents
\addcontentsline{toc}{chapter}{Bibliography}

% nocite: Pour citer la totalit\'{e} des r\'{e}f\'{e}rences contenues dans le fichier biblio
% nocite: In order to cite all the references included biblio
% \nocite{*}
\printbibliography
% \newpage
% \nocite{*}
% \printbibliography[heading=secondary,keyword=secondary]

% \clearemptydoublepage
% Pour avoir la quatrième de couverture sur une page paire
% To have the back cover on an even page
\cleartoevenpage[\thispagestyle{empty}]
\markboth{}{}
% Plus petite marge du bas pour la quatrième de couverture
% Shorter bottom margin for the back cover
\newgeometry{inner=30mm,outer=20mm,top=40mm,bottom=20mm}

%insertion de l'image de fond du dos (resume)
%background image for resume (back)
\backcoverheader

% Switch font style to back cover style
\selectfontbackcover{ % Font style change is limited to this page using braces, just in case

\titleFR{Extension de la Protection des Processeurs Contre les Menaces Physiques et Logicielles par la Sécurisation du Mécanisme DIFT Contre les Attaques par Injections de Fautes}

\keywordsFR{\small DIFT, Fault Injection Attacks, Contre-mesures, Hamming Code, Code de Correction d'Erreur}

\abstractFR{\small L'augmentation de l'IoT, dans des domaines tels que la santé ou l'industrie, favorise la hausse de la surface d'attaque, ce qui soulève d'importantes préoccupations en termes de sécurité. Ces systèmes, traitant des données sensibles, sont vulnérables aux attaques logicielles et physiques en raison de leur connectivité réseau et de leur proximité avec les attaquants.
Le suivi dynamique des flux d'informations (DIFT) détecte les attaques logicielles, comme les maliciels, en étiquetant et en suivant les données au moment de l'exécution. Les attaques par injection de fautes (FIA) induisent des erreurs (par exemple, via la tension ou des lasers) perturbant le comportement et contournant les mécanismes de sécurité. Les FIA sont critiques dans les systèmes embarqués et cryptographiques, où les vulnérabilités peuvent compromettre les données. Bien que de nombreuses études aient exploré les vulnérabilités des FIA, aucune n'a ciblé les mécanismes DIFT.
Nous travaillons sur le processeur D-RI5CY, implémentant un DIFT matériel in-core. Nous évaluons l'impact des FIA sur l'efficacité du DIFT. Grâce à des simulations d'injection de fautes, en utilisant FISSA, un outil conçu pour l'évaluation des fautes, nous identifions les registres vulnérables et implémentons trois protections : la parité simple pour la détection, le code de Hamming pour la correction d'erreurs sur un bit, et SECDED pour détecter les erreurs sur deux bits. Ces protections ont été optimisées en regroupant les registres afin de minimiser le coût. Nous avons ensuite évalué d'autres compositions de groupes pour améliorer la protection contre des modèles plus complexes, en développant quatre stratégies pour améliorer la détection et la correction des erreurs.}



\titleEN{Enhanced Processor Defence Against Physical and Software Threats by Securing DIFT Against Fault Injection Attacks}

\keywordsEN{\small DIFT, Fault Injection Attacks, Countermeasures, Hamming Code, Error Correction Code}

\abstractEN{\small Embedded security is more and more crucial with the huge increase of IoT devices, enhancing efficiency and addressing challenges like industrial change and health. However, their widespread use also increases the attack surface, raising significant security concerns. These systems, handling sensitive data, are vulnerable to both software and physical attacks due to their network connectivity and proximity to attackers.
Dynamic Information Flow Tracking (DIFT) detects software attacks, such as buffer overflows, by tagging and tracking data at runtime. Fault Injection Attacks (FIA) deliberately introduce hardware errors to disrupt normal operation and bypass security mechanisms. These faults can be introduced physically (e.g., via voltage or lasers). FIAs are concerning in embedded and cryptographic systems, where low-level faults can compromise sensitive data. Although many studies have explored FIA vulnerabilities, none have targeted DIFT mechanisms.
Our research focuses on the D-RI5CY processor, which implements a hardware in-core DIFT. We assess the impact of FIAs on DIFT's effectiveness in this processor. Through fault injection simulations, using FISSA, a tool developed to facilitate fault evaluation, we identify vulnerable hardware registers and implement three countermeasures: simple parity for error detection, Hamming Code for single-bit error correction, and SECDED to detect double-bit errors. These were optimised by grouping registers to minimise redundancy overhead. We further evaluated multiple register group compositions to enhance countermeasures against complex fault models, developing four strategies to improve error detection and correction efficiency.}

}

% Rétablit les marges d'origines
% Restore original margin settings
% \restoregeometry


\end{document}
