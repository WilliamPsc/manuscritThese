\chapter{State of the Art}
\chaptermark{State of the Art}
\label{chapter:soa}
\minitoc


%%%%%%%%%%%%%%%%%%%%%%%%%%%%%%%%%%%%%%%%%%%%%%%%%%%%%%%%%%%%%%%%%%%%%%%%%%%%%%%%%%%%%%%%%%%%%%%
\section{Introduction}
This chapter provides an overview of related work to contextualize the primary objectives of this thesis. Firstly, in Section~\ref{section:ift}, Information Flow Tracking (IFT) is introduced, detailing the different types and their respective purposes. We will discuss the various levels of monitoring, from program behaviour to the detection of hardware trojans.
Then in Section~\ref{section:physicalAttacks}, Physical Attacks are examined, focusing on two main types: Side-Channel Attacks (SCA) and Fault Injection Attacks (FIA).
Finally in Section~\ref{section:countermeasuresAgainstFIA}, as this work will concentrate on FIA, we will exclusively present countermeasures against Fault Injection Attacks.

%%%%%%%%%%%%%%%%%%%%%%%%%%%%%%%%%%%%%%%%%%%%%%%%%%%%%%%%%%%%%%%%%%%%%%%%%%%%%%%%%%%%%%%%%%%%%%%
\section{Information Flow Tracking}
\label{section:ift}
This section introduces Information Flow Tracking mechanisms, explains how they work, and presents the various types of IFT with their different functional levels.
% We mainly focus on presenting hardware IFT architectures.
    
%%%%%%%%%%%%%%%%%%%%%%%%%%%%%%%%
\subsection{Different types of IFT}
There are two distinct types of IFT approaches: static and dynamic, each with its own specific objectives.

\subsubsection{Static IFT}
Static Information Flow Tracking (SIFT) is a security technique used to analyse and control the flow of information within a program or system without executing it, by examining the source code or compiled binary. This method is particularly useful for identifying theoretical vulnerabilities, ensuring compliance with design principles, and preventing unauthorised information leaks before deployment. SIFT is comprehensive, covering all possible execution paths and detecting both explicit information flows (direct data assignments) and implicit flows (leaks through control flow structures). By performing checks at compile-time, SIFT helps developers address potential security issues early, enforcing principles like non-interference and data confidentiality through security policies. However, static analysis may generate false positives by flagging theoretical flows that might not occur in practice and may struggle with certain dynamic language features or runtime-dependent behaviours. SIFT is employed in various contexts, such as verifying secure information flow in operating systems, programming languages with built-in information flow controls, and hardware design for secure systems.



\subsubsection{Dynamic IFT}
Dynamic Information Flow Tracking (DIFT) is a powerful security technique that monitors and analyses, in real-time, the flow of information within a program during its execution~\cite{CGDJ-21-micromac}. DIFT operates by tagging or labelling input data from potentially untrusted sources and tracking how this data propagates through the system~\cite{SLD-04-sigplan}. As the program executes, DIFT maintains metadata about the tagged information, updating it as operations are performed on the data. This allows the system to detect when tainted data is used in security-critical operations, such as modifying control flow or accessing sensitive resources. DIFT can be implemented at various levels, including hardware, software, or a combination of both. Hardware-based implementations often offer better performance but require specialized processor modifications, while software-based approaches provide more flexibility but may incur higher overhead~\cite{CGDJ-21-micromac}. DIFT has proven effective in detecting and preventing a wide range of security vulnerabilities, including buffer overflows, format string attacks, and code injection attacks~\cite{SLD-04-sigplan}. However, DIFT also faces challenges, such as handling implicit information flows, managing performance overhead, and addressing over-tainting issues.
This approach might not cover all potential data paths, as it is dependent on the specific conditions and inputs provided during the monitoring period.
Despite these challenges, DIFT remains a valuable tool in the cybersecurity arsenal, particularly for runtime attack detection and prevention in modern computing systems.

%%%%%%%%%%%%%%%%%%%%%%%%%%%%%%%%    
\subsection{Different levels of IFT}
IFT can be implemented at various levels of abstraction in computing systems. Each level presents unique trade-offs between precision, performance overhead, and ease of implementation, allowing designers to choose the most appropriate approach for their security requirements.
At the lowest level, Gate-Level IFT (GLIFT) tracks information flow through individual logic gates, providing precise security guarantees.
Moving up, Register Transfer Level IFT (RTLIFT) operates on hardware descriptions, enabling early-stage verification of security properties.
At the microarchitecture level, IFT can be implemented within processor cores or as off-core designs, balancing performance and security.
Software-based IFT operates at either the system level, monitoring OS-wide information flows, or the program level, focusing on specific applications.
Finally, hardware-software co-designed IFT solutions aim to leverage the strengths of both hardware and software implementations.

Figure~\ref{fig:TODO} shows the four different levels of an embedded system: application layer, system service layer, OS layer, and hardware layer.

\subsubsection{Application level}
\subsubsection{OS level}
\subsubsection{Architecture level}
\subsubsection{Gate level}

%%%%%%%%%%%%%%%%%%%%%%%%%%%%%%%%
\subsection{DIFT Architectures}
\subsubsection{Off-Core}

\subsubsection{Off-Loading}

\subsubsection{In-Core}

%%%%%%%%%%%%%%%%%%%%%%%%%%%%%%%%%%%%%%%%%%%%%%%%%%%%%%%%%%%%%%%%%%%%%%%%%%%%%%%%%%%%%%%%%%%%%%%
\section{Physical Attacks}
\label{section:physicalAttacks}

\subsection{Side-Channel Attacks}
\subsection{Fault Injection Attacks}
% A fault is the cause of an error, that is, an incorrect program or circuit state. If the error caused by the fault does not propagate and the application execution ends normally, the fault is ineffective. On the contrary, the fault is effective if the error affects the application’s execution, causing a failure, an observed behavior different from that expected.
% In the context of electronic circuits, a fault refers to an unintended deviation from the normal operation of the circuit. Faults can occur due to various reasons such as manufacturing defects, environmental factors, ageing, or external interference. These faults can affect the performance, functionality, and reliability of the circuit.

% In fault injection, which is a testing method used to evaluate the robustness and reliability of electronic circuits, a fault is deliberately introduced into the system to observe its behaviour and identify potential vulnerabilities.

%%%%%%%%%%%%%%%%%%%%%%%%%%%%%%%%%%%%%%%%%%%%%%%%%%%%%%%%%%%%%%%%%%%%%%%%%%%%%%%%%%%%%%%%%%%%%%%
\section{Countermeasures against FIA}
\label{section:countermeasuresAgainstFIA}

%%%%%%%%%%%%%%%%%%%%%%%%%%%%%%%%%%%%%%%%%%%%%%%%%%%%%%%%%%%%%%%%%%%%%%%%%%%%%%%%%%%%%%%%%%%%%%%