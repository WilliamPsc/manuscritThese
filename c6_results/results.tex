\chapter{Experimental setup and results}
\chaptermark{Experimental setup and results}
\label{chapter:exp_setup_results}
\minitoc

%%%%%%%%%%%%%%%%%%%%%%%%%%%%%%%%%%%%%%%%%%%%%%%%%%%%%%%%%%%%%%%%%%%%%%%%%%%%%%%%%%%%%%%%%%%%%%%
\section{Introduction}
The previous chapter presented two countermeasures against fault injection attacks and taking into account simple fault models, such as single-bit flip error inside one register at a given clock cycle. These countermeasures have been implemented by grouping the different registers in order to reduce the number of parity and redundancy bits. However nowadays, studies~\cite{CGVCBLC-22-cardis,VDSPB-24-jce} have shown that is it possible to fault multiple bits precisely.

In this chapter, we present four different implementations of countermeasures to better protect the D-RI5CY mechanism against more complex fault models. Each implementation will be presented in their respective section. Next, we implement another version of Hamming code to detect double-bit errors and correct single-bit errors, this method is called SECDED for \textit{Single Error Correction, Double Error Detection}.

The first section of this chapter presents the different fault models considered. Then, the second section explains the 4 different implementations of Hamming code and gives the results associated to the fault models. The third section introduces SECDED countermeasure and gives the results associated to the five re-implementations done for this new protection. Finally,  we compare the results obtained from these two countermeasures and evaluate them in terms of efficiency, performance and area overhead.

%%%%%%%%%%%%%%%%%%%%%%%%%%%%%%%%%%%%%%%%%%%%%%%%%%%%%%%%%%%%%%%%%%%%%%%%%%%%%%%%%%%%%%%%%%%%%%%
\section{Fault models used in this chapter}


%%%%%%%%%%%%%%%%%%%%%%%%%%%%%%%%%%%%%%%%%%%%%%%%%%%%%%%%%%%%%%%%%%%%%%%%%%%%%%%%%%%%%%%%%%%%%%%
\section{Countermeasure 1: Simple Parity}

\subsection{Implementation 1: Optimisation of redundancy bits}

%%%%%%%%%%%%%%%%%%%%%%%%%%%%%%%%%%%%%%%%%%%%%%%%%%%%%%%%%%%%%%%%%%%%%%%%%%%%%%%%%%%%%%%%%%%%%%%
\section{Countermeasure 2: Hamming Code}

\subsection{Implementation 2: Protection by pipeline stage}

\subsection{Implementation 3: Protection of all registers individually}

\subsection{Implementation 4: Protection of all registers individually with CSRs slicing}

\subsection{Implementation 5: Cooking spaghetti is not forbidden}

%%%%%%%%%%%%%%%%%%%%%%%%%%%%%%%%%%%%%%%%%%%%%%%%%%%%%%%%%%%%%%%%%%%%%%%%%%%%%%%%%%%%%%%%%%%%%%%
\section{Countermeasure 3: Hamming Code - SECDED}

\subsection{Implementation 1: Optimisation of redundancy bits}

\subsection{Implementation 2: Protection by pipeline stage}

\subsection{Implementation 3: Protection of all registers individually}

\subsection{Implementation 4: Protection of all registers individually with CSRs slicing}

\subsection{Implementation 5: Smart protection by pipeline stage}


%%%%%%%%%%%%%%%%%%%%%%%%%%%%%%%%%%%%%%%%%%%%%%%%%%%%%%%%%%%%%%%%%%%%%%%%%%%%%%%%%%%%%%%%%%%%%%%
\section{Discussion}


%%%%%%%%%%%%%%%%%%%%%%%%%%%%%%%%%%%%%%%%%%%%%%%%%%%%%%%%%%%%%%%%%%%%%%%%%%%%%%%%%%%%%%%%%%%%%%%
\section{Summary}


%%%%%%%%%%%%%%%%%%%%%%%%%%%%%%%%%%%%%%%%%%%%%%%%%%%%%%%%%%%%%%%%%%%%%%%%%%%%%%%%%%%%%%%%%%%%%%%