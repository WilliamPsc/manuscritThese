\chapter{Conclusion}
\chaptermark{Conclusion}
\label{chapter:conclusion}

\epigraph{\textit{The only truly secure system is one that is powered off, cast in a block of concrete and sealed in a lead-lined room with armed guards - and even then I have my doubts.}}{Gene Spafford}

\minitoc

%%%%%%%%%%%%%%%%%%%%%%%%%%%%%%%%%%%%%%%%%%%%%%%%%%%%%%%%%%%%%%%%%%%%%%%%%%%%%%%%%%%%%%%%%%%%%%%
\section{Synthesis}

With the rapid expansion of IoT and the growing ubiquity of embedded systems, ensuring robust security has become a critical priority for both hardware designers and software developers. Protecting these systems from potential threats, especially physical attacks, remains a key challenge. Among these threats, Fault Injection Attacks (FIA) stand out as a significant risk due to their capacity to disrupt device operation and compromise data integrity.

FIAs are particularly dangerous because they allow attackers to inject faults into a system during runtime, potentially bypassing even the most robust software security mechanisms. By manipulating voltage, clock signals, or using techniques like laser-based injections, adversaries can induce unexpected behaviour, leading to data leakage, corruption, or system hijacking. These attacks are becoming more accessible due to the decreasing cost of fault injection tools, making it imperative to design systems with built-in resilience. Existing security mechanisms, like Dynamic Information Flow Tracking (DIFT), which is use as a security against software threats, are not immune to these attacks, necessitating deeper investigation and the development of tailored countermeasures. Without effective defences, FIAs remain a potent threat, capable of undermining the reliability and trustworthiness of critical IoT systems.

This thesis has aimed to address these challenges by assessing vulnerabilities and proposing lightweight countermeasures to strengthen digital systems against FIAs. By evaluating and improving the security of Dynamic Information Flow Tracking (DIFT) mechanisms, we have proposed a solution on how to protect systems against sophisticated physical and software-based threats. In this concluding chapter, we summarise the contributions made, reflect on the findings, and discuss the potential for further advancements in securing embedded systems against physical attacks.

The second chapter, we systematically introduced the three main parts of this research. First, we provided a comprehensive explanation of hardware-based DIFT and conducted a detailed review of the state-of-the-art of Information Flow Tracking methodologies, spanning software implementations, hardware solutions, and co-design approaches that integrate both. Second, we categorised various forms of physical attacks, with a particular emphasis on an in-depth analysis of FIA and their diverse mechanisms for compromising system security. Finally, we presented a critical overview of the existing countermeasures designed to effectively protect systems against FIAs, laying the foundations for the subsequent development of enhanced lightweight protection strategies.

In the third chapter, we presented the processor utilised in this work, detailing its implementation of in-core hardware-based DIFT and demonstrating its use in its default configuration. In the second part, we described three specific use cases developed to analyse the behaviour of the DIFT mechanism, and we conducted a theoretical assessment of its resilience against Fault Injection Attacks, considering classical single fault models such as \textit{bit set}, \textit{bit reset}, and \textit{single bit-flip}. Finally, we evaluated the DIFT's vulnerabilities through simulation campaigns to validate our theoretical results. Our findings revealed that the DIFT mechanism is predominantly vulnerable to single bit-flip faults due to its 1-bit data path. The fault injection simulations corroborated these results, highlighting critical registers that varied depending on the specific use case under consideration.

In the fourth chapter, we introduced FISSA (Fault Injection Simulation for Security Assessment), a novel open-source tool developed to support \textit{Security by Design}. FISSA enables designers to assess the security of their systems during the conceptual phase of development. Seamlessly integrated with well-known HDL tools and simulators, such as Questasim, FISSA accepts a set of parameters and generates corresponding TCL scripts, which are executed within the HDL simulator. Each simulation produces detailed JSON log files, providing a comprehensive basis for security analysis. The tool is highly configurable, allowing designers to tailor it to meet specific design requirements, offering flexibility in the evaluation process.

In the fifth chapter, we proposed and implemented three lightweight countermeasures to enhance the security of the D-RI5CY mechanism. The first countermeasure involves the use of simple parity as a fault detector. Upon detecting a fault, the parity bit triggers a signal to alert the system. The second countermeasure employs Hamming Code as a single fault corrector, capable of detecting and correcting single-bit errors with a 100\% accuracy at cycle accurate. This technique effectively corrected all single bit-flips induced by the fault models evaluated in Chapter~\ref{chapter:dift_assessment}. However, with the advent of more sophisticated fault injection platforms capable of inducing multiple faults, single bit-flips are no longer the predominant threat. This led to the introduction of more complex fault models, such as \textit{single bit-flip in two registers at two distinct clock cycles}. To address this, we implemented the third countermeasure, SECDED (Single Error Correction, Double Error Detection), which extends the Hamming Code by adding an additional bit for parity to enable the detection of double-bit errors. These three countermeasures demonstrated strong effectiveness against the fault models considered, while maintaining minimal impact on system performance and area overhead.

In the sixth chapter, we took into account even more complex fault models, such as \textit{single bit-flip in two registers at one clock cycles}, \textit{multi-bit faults in one register at a given clock cycle}, and \textit{multi-bit faults in two registers at a given clock cycle}. These fault models access the limit of our three countermeasures. As we can inject two to twelve faults at the same time, the possibilities of detection and correction are not enough. To achieve a better protection by staying with our three lightweight countermeasures, we decided to evaluate different group composition on our encoders. This evaluation allowed to assess the security performances of each strategy and take into account the performance and area overhead induced by each strategies to better compare them for a small constraint embedded system. Thanks to these strategies, we have shown better security performances by doing some compromises on the size. However, with an increase of 5\% of our processor size, we are able to detect and correct the vast majority of successes. For the remaining successes, a better protection would need to be evaluated such as a better ECC (BCH code for example).

%%%%%%%%%%%%%%%%%%%%%%%%%%%%%%%%%%%%%%%%%%%%%%%%%%%%%%%%%%%%%%%%%%%%%%%%%%%%%%%%%%%%%%%%%%%%%%%
\section{Perspectives}

In terms of perspectives, this work has reach its primary objective: to propose a protected DIFT mechanism against Fault Injection Attacks. However, many possibilities still exist to pursue this research. A non-exhaustive list of perspectives is thus provided hereunder.

We worked on a specific DIFT which consider tags on 1-bit. However, they are others implementations~\cite{DKK-07-sigarch} with multiple bits tags, as well as, they are also more complex CPUs with features, such as deeper pipeline, prefetching, speculation, out-of-order execution, etc. The vulnerabilities can be different depending on the implementation. An evaluation of different DIFT should be done to get a better overview on DIFT vulnerabilities and propose a protection for these mechanisms.

Another way of improving this research is to continue the development of FISSA to take into account more HDL tools such as Vivado, Verilator. But also, it needs more fault models considered in the literature such as fault models against laser-based injection, X-Ray, or other novels techniques that would come later. The integration into the design workflow needs to be ameliorated to allow a better and simpler usage by the designers. And finally, a graphical user interface can be implemented to enhance the use of the tool and allow a direct overview of the results analysis.

Moreover, a third possibility is to conduct real-world fault injection attacks on a FPGA board to assess the vulnerabilities of the D-RI5CY against fault injection attacks and check if our proposed countermeasures are enough to protect our system effectively such as in simulations. It would give a confirmation on the well working of these protections against faults in real conditions. With this possibility, we would also exhaustively assess the two CSRs registers against multi-bit faults as we weren't able to do it in simulation.

Additionally, as shown in Chapter~\ref{chapter:exp_setup_results}, even with our proposed strategies, they are still some successes than can happen. To have a full protection we would need to implement a better protection against multi-bit faults. For that, we can consider using redundancy on the registers or staying with the Error Correction Code and implement a better linear or cyclic code, such as LDPC, Bose–Chaudhuri–Hocquenghem codes (BCH), Reed-Solomon. However, these codes are a lot bigger and need more calculations to compute the redundancy of the data, but it can correct multiples bit errors. An evaluation of these protections can be made to have a compromise between performance and security. It is worth noting that for example BCH code take in general multiples cycles to execute, and even if it possible to implement them to check in one cycle, it would require a gigantic amount of area.  

For the long foreseeable future, it could be interesting to study if a DIFT mechanism can detect fault injection attacks happening in the processor itself. A FIA can enable an modification of the instruction path, change a value or even a tag value and hence allows the detection of this error. The behaviour of the DIFT against FIA needs to be assessed.

%%%%%%%%%%%%%%%%%%%%%%%%%%%%%%%%%%%%%%%%%%%%%%%%%%%%%%%%%%%%%%%%%%%%%%%%%%%%%%%%%%%%%%%%%%%%%%%