\chapter{Introduction}
\chaptermark{Introduction}
\label{chapter:introduction}

\epigraph{\textit{IoT without security means Internet of Threats}}{Stéphane Nappo}

\minitoc

%%%%%%%%%%%%%%%%%%%%%%%%%%%%%%%%%%%%%%%%%%%%%%%%%%%%%%%%%%%%%%%%%%%%%%%%%%%%%%%%%%%%%%%%%%%%%%%
\section{Context}

%%%%%%%%%%%%%%%%%%%%%%%%%%%%%%%%%%%%%%%%%%%%%%%%%%%%%%%%%%%%%%%%%%%%%%%%%%%%%%%%%%%%%%%%%%%%%%%
\section{Motivations}

\cite{rayvlite_wired, rayvlite_fraktal}

%%%%%%%%%%%%%%%%%%%%%%%%%%%%%%%%%%%%%%%%%%%%%%%%%%%%%%%%%%%%%%%%%%%%%%%%%%%%%%%%%%%%%%%%%%%%%%%
\section{Objectives}

%%%%%%%%%%%%%%%%%%%%%%%%%%%%%%%%%%%%%%%%%%%%%%%%%%%%%%%%%%%%%%%%%%%%%%%%%%%%%%%%%%%%%%%%%%%%%%%
\section{Manuscript outline}

This work is segmented in seven chapters, the first being this introduction.

Chapter~\ref{chapter:soa} presents the state of the art about Information Flow Tracking (IFT) by explaining how they work, the different types of existing IFT, this chapter presents what is physical attacks, the literature about it and presents the two principle type of physical attacks: Side-Channel Attacks and Fault Injection Attacks. Finally, the chapter presents an overview of the literature about countermeasures against Fault Injection Attacks

Chapter~\ref{chapter:dift_assessment} presents the background of this work with the presentation of the RISC-V ISA, the architecture of the D-RI5CY core and the DIFT works. Then, the use cases used in this work are going to be presented. Finally, a vulnerability assessment will be done to show how these use case are vulnerable against FIA and where.

Chapter~\ref{chapter:fissa} introduces a new tool, FISSA, to automatise fault injection campaigns in simulation. This tool allows a designer to assess his design during the conception phase. This chapter presents how it works and how to use it, and compares it to others tools available in the literature.

Chapter~\ref{chapter:countermeasures} details the different implementation of countermeasures to protect the D-RI5CY core against FIA and evaluate these protections in terms of area, performance, and efficiency.

Chapter~\ref{chapter:exp_setup_results}

Chapter~\ref{chapter:conclusion}

%%%%%%%%%%%%%%%%%%%%%%%%%%%%%%%%%%%%%%%%%%%%%%%%%%%%%%%%%%%%%%%%%%%%%%%%%%%%%%%%%%%%%%%%%%%%%%%